\documentclass[a4paper,10pt]{article}
\pdfoutput=1

\usepackage[utf8]{inputenc}

\usepackage{enumitem}
\setlist{nosep}


\textheight = 24cm
\textwidth = 16.5cm
\oddsidemargin = -5pt
\topmargin = -1.5cm
\parskip = 10pt

\usepackage{hyperref}
\usepackage{xcolor}
\hypersetup{
    colorlinks=true,
    allcolors={blue!60!black}
    %linkcolor={red!50!black},
    %citecolor={blue!50!black},
    %urlcolor={blue!80!black}
}

\title{\vspace{-2cm}Proposal for Cumulative Thesis}
\date{\vspace{-2ex}June 19, 2019}
\author{\vspace{-1ex}Vasily Sotnikov}


\newcommand{\paper}[1]{\vspace{3ex}\noindent\large\textbf{#1}\newline}

\begin{document}

\maketitle


\subsection*{Relevant Publications}
\vspace{-3ex}

\paper{1. NLO QCD predictions for $Wb\bar b$ production in association with up to three light jets at the LHC}
  F.~R.~Anger, F.~Febres Cordero, H.~Ita and V.~Sotnikov\\
  Phys.\ Rev.\ D {\bf 97}, no. 3, 036018 (2018), doi:10.1103/PhysRevD.97.036018

  In this article the next-to-leading QCD corrections to the associated production of $W$-bosons, a $b\bar{b}$ pair, and up to three light jets are presented.
  The computation is done in the four-flavor-number scheme, thus including effects due to the non-vanishing bottom-quark mass.
  The results obtained in this work provide a prediction for an important irreducible background to $H(\to b\bar{b})W$ studies.

  The computation of one-loop matrix elements required for this project became possible
  due to my work on the major upgrade of the \textsc{BlackHat} library to handle massive fermions in automated numerical unitarity.
  The key developments are:
  \begin{itemize}
    \item unitarity cuts for a complete set of diagrams with internal massive particles in the loop
    \item an algorithm for the extraction of the rational part through a modified spectrum of particles in the loop
    \item in-house multi-precision implementation of master integrals with internal masses.
  \end{itemize}
  I also have carried out the study of numerical stability of the new version of the \textsc{BlackHat} library.
  My main contribution to the writeup of this publication is in the description of formal aspects of virtual matrix element computation.


\paper{2. Planar Two-Loop Five-Parton Amplitudes from Numerical Unitarity}
  S.~Abreu, F.~Febres Cordero, H.~Ita, B.~Page and V.~Sotnikov\\
JHEP {\bf 1811}, 116 (2018), doi:10.1007/JHEP11(2018)116

  In this work we computed for the first time all two-loop five-parton helicity amplitudes relevant
  for the computation of NNLO QCD corrections to the production of tree jets at hadron colliders in the leading color approximation.
  The computation is carried out with the recently developed two-loop numerical unitarity methods, which we extend for use with finite field arithmetics.
  We exploit decomposition of the integrand into master and surface terms that is independent of the parton type.
  We present reference values for helicity amplitudes, which are obtained by combining the master-integral coefficients with the known master integrals.

  For this project I have developed and implemented the formalism for
  consistent definition of helicity amplitudes with quarks in dimensional regularization,
  suitable for application in the framework of numerical computations with multi-loop $D$-dimensional unitarity.

  Other technical contributions which has been crucial for this project include:
  \begin{itemize}
    \item extension of the $D$-dimensional tree-level amplitude generator to handle fermions
    \item a general solution to the compatibility of fermions with evaluations over finite fields
    \item extensive validation of the new results (divergent structure, comparison to literature, etc.\@)
  \end{itemize}
  As a result of my work I became a main developer of the \texttt{C++} framework \textsc{Caravel} for exploring multi-loop amplitudes
  with numerical unitarity.

  I have contributed substantially to the preparation of all parts of the manuscript,
  with the focus on the definition of helicity amplitudes and presenting the main results of the paper.

\paper{3. Analytic Form of the Planar Two-Loop Five-Parton Scattering Amplitudes in QCD}
S.~Abreu, J.~Dormans, F.~Febres Cordero, H.~Ita, B.~Page and V.~Sotnikov\\
JHEP {\bf 1905}, 084 (2019), doi:10.1007/JHEP05(2019)084

In this work we present the analytic form of all two-loop five-parton helicity amplitudes
required for the computation of NNLO QCD corrections to the production of three jets at hadron colliders
in the leading-color approximation.
We demonstrate how a combination of numerical evaluation of amplitudes through numerical unitarity with a
dedicated algorithm of multivariate rational function reconstruction allows to obtain
the analytic form of the amplitudes from exact numerical evaluations over finite fields.
Their systematic simplification using multivariate partial-fraction decomposition
leads to a particularly compact form. We show that with these methods, a sample with a size comparable to that
required for numerical Monte-Carlo integration is sufficient to produce analytic expressions.

  Here I have designed and implemented an algorithm which significantly optimizes the
  treatment of $D$-dimensional particles in the loops and allows to carry out all numerical
  evaluations with particle states taken to be in six dimensions. This replaces reconstruction
  of the dependence on the dimensional regulator from sample values.
  This development made the analytical reconstruction of amplitudes with quarks from numerical evaluations possible,
  which was not feasible before.

  I have also contributed to many other parts of the project, such as
  building a framework for automated evaluation and reconstruction of large number of amplitudes.
  My main contribution to the writeup of the publication is in the description of the new algorithms employed for the evaluation
  of amplitudes, as well as presentation of the results.


\subsection*{Ongoing Work}

%I have presented the results of all the referenced papers through seminar talks in multiple universities and at a number of major international conferences.
I am currently actively involved in the ongoing research projects which will results in further significant results and publications in the near future.

Some of the details of my work on the dimensional regularization of helicity amplitudes in the framework of numerical unitarity can be found
in the (yet) unpublished article\\
\textit{``On the Dimensional Regularization of QCD Helicity Amplitudes With Quarks,''} F.~R.~Anger and V.~Sotnikov, arXiv:1803.11127 [hep-ph].








\end{document}
