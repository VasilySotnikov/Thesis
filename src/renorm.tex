In order to arrive at physical predictions, we need to
\textit{renormalize} QCD \ola s. Renormalization refers to the process of
connecting the initial Lagrangian parameters, so called bare
parameters, to physical quantities that can be measured in experiment. In a renormalizable theory, UV divergencies stemming from
virtual corrections drop out once the bare parameters are connected to
physical ones. The SM, and thus QCD, is a renormalizable theory \cite{tHooft:1971qjg,tHooft:1972tcz}.


We renormalize \ola s for processes involving massive quarks in a
mixed scheme. The strong coupling is renormalized in the
$\MSb$ scheme
and we apply a shift to decouple massive quark loops from the running of
$\alpha_s$. The wave-function and mass renormalization is done in the
on-shell scheme. The mass renormalization is performed at the
level of each color-ordered amplitude. Previously, we explicitly broke gauge invariance of the amplitudes by
removing external leg self-energy Feynman diagrams from tree
amplitudes that appeared in double cuts on \olb~topologies
(cf.~Sec.~\ref{sec:massivebubble}). After recombination with the corresponding mass counterterms, gauge
invariance of the \ola s is restored. The coupling renormalization and massive quark
decoupling as well as wave-function renormalization lead to a shift proportional to the tree amplitude. We provide all renormalization constants in FDH to
be consistent with the computation of the \ola s. 

Furthermore, in Sec.~\ref{sec:scaledep} we work out the correct \root~\cite{ROOT} \ntuple{} weights \cite{BH:Ntuples} for massive
one-loop amplitudes. Using \ntuple{} files allows for an efficient a-posteriori reevaluation of
matrix elements with different renormalization and factorization
scales as well as couplings and parton distribution functions.

\section{Multiplicative Renormalization}
\label{sec:stdrenorm}
We will briefly review the standard multiplicative renormalization
procedure (more details can be found in many textbooks \cite{PeskinS,Schwartz:2013pla,Bohm:2001yx}). In particular, we will see that mass counterterms have to
be computed at the level of each color-ordered amplitude whereas the
other renormalizations result in a shift proportional to the tree amplitude. We
reparameterize the bare parameters in the Lagrangian by 
\begin{align}
  \psi_0 & = \sqrt{Z_2}\psi_R, & A_0^\mu & =  \sqrt{Z_3} A_R^\mu, \\
m_0 & = Z_m m_R, & g_0 & = Z_g g_R, \notag
\end{align}
where $\psi_0$ is the bare fermion field, $A^\mu_0$ the bare QCD gauge field, $m_0$
the bare mass and $g_0$ the bare strong coupling. In perturbation
theory, we expand the renormalization constants $Z_i$ and get to first order
\begin{align}\label{eq:renconst}
  Z_2 & \equiv 1 + \delta_2, &  Z_3 &\equiv 1 + \delta_3,  \\
  Z_m & \equiv 1 + \delta_m, &  Z_g &\equiv 1 + \delta_g, \notag 
\end{align}
which leads to a split up of the Lagrangian. We get the
original Lagrangian in which bare parameters are replaced by
renormalized ones, and in addition, the so called \textit{counterterm}
Lagrangian that collects all terms in the renormalization constants $\delta_i$
and generates counterterm Feynman rules. The renormalization constants
multiplying the counterterm Feynman rules for QCD vertices are given by
\begin{align}\label{eq:vertqcdct}
\begin{split}
  Z_{3g}-1 = Z_3^{3/2}Z_g-1 &\equiv \frac{3}{2} \delta_3 +
  \delta_g, \\
  Z_{4g}-1 = Z_3^{4/2}Z_g^2 -1&\equiv  2\delta_3 + 2 \delta_g, \\
Z_{2q1g} -1= Z_2 Z_3^{1/2}Z_g -1 &\equiv  \delta_2 +
  \frac{1}{2}\delta_3 + \delta_g,
\end{split}
\end{align}
for the three- and four-valent gluon interaction $Z_{3g}$ and $Z_{4g}$
as well as the quark-gluon interaction $Z_{2q1g}$. The Feynman
rules for two-particle counterterms in Feynman gauge are given in Fig.~\ref{fig:qcdct}.
\begin{eqnarray*}
\begin{tikzpicture}[baseline=(m)]
  \begin{feynman}[inline=(m)]
    \vertex (m) at (0,0);
    \vertex (v1) at (1,0);
    \vertex (v2) at (2,0);
    \diagram* {
      (m)  -- [fermion,insertion={[size=8pt]0.99}] (v1) -- [fermion] (v2), 
    };
  \end{feynman}
\end{tikzpicture}
 & \qquad i\left((\slashed{p}-m_R)\delta_2 -
\delta_mm_R\right) &=P^q_{\text{ct}}(p)_{|\delta_2} +P^q_{\text{ct}}(p)_{|\delta_m}\\
\vspace{1cm}
\begin{tikzpicture}[baseline=(m)]
  \begin{feynman}[inline=(m)]
    \vertex (m) at (0,0);
    \vertex (v2) at (2,0);
    \diagram* {
      (m)  -- [gluon,insertion={[size=8pt]0.5}] (v2), 
    };
  \end{feynman}
\end{tikzpicture}& \qquad -ip^2\delta_3 g^{\mu\nu}&=P^{g,\mu\nu}_{\text{ct}}(p)\\
\end{eqnarray*}
\vspace{-1.5cm}
\captionedequationset{Feynman rules for QCD two-particle counterterms in Feynman gauge. The renormalization constants
  $\delta_i$ are defined in Eq.~\eqref{eq:renconst}.\label{fig:qcdct}} 
In principle, one has to compute all possible
counterterms as specified by the counterterm Feynman rules. However,
the simple counting exercise of Sec.~\ref{sec:renorm-shifts-prop} reveals that wave-function and coupling renormalization of QCD one-loop amplitudes lead to a
renormalization shift proportional to the tree. The remaining mass
renormalization counterterms have to be computed explicitly, as
spelled out in Sec.~\ref{sec:mrenorm}.

\section{Coupling and Wave-Function Renormalization}
\label{sec:renorm-prop-tree}

\subsection{Renormalization Shifts Proportional to Tree Amplitudes}
\label{sec:renorm-shifts-prop}

We first observe that the combination of two propagators of the QCD spectrum with a counterterm
insertion leads, for the parts independent of $\delta_m$, to the original propagator
multiplied by a renormalization constant. In particular, the counterterm
insertion on fermion propagators is given by
\begin{align}\label{eq:repropq}
\begin{split}
  P^q(p)P_{\text{ct}}^q(p)_{|\delta_2} P^q(p)& =  \frac{i(\slashed{p}+m_R)}{p^2-m_R^2}\left(i(\slashed{p}-m_R)\delta_2\right) \frac{i(\slashed{p}+m_R)}{p^2-m_R^2}\\
  &=-\delta_2\frac{i(\slashed{p}+m_R)}{p^2-m_R^2} = -\delta_2 P^q(p),
\end{split}
\end{align}
and that on gluon propagators by
\begin{align}\label{eq:repropg}
\begin{split}
  P^g_{\mu\nu}(p) P^{g,\nu\rho}_{\text{ct}} P^g_{\rho\sigma}(p)& =  \frac{-ig_{\mu\nu}}{p^2} \left(-ip^2\delta_3 g^{\nu\rho}\right)  \frac{-ig_{\rho\sigma}}{p^2}\\
  &=-\delta_3\frac{-ig_{\mu\sigma}}{p^2} = -\delta_3P^g_{\mu\sigma}(p).
\end{split}
\end{align}
It remains to show that the multiplicative renormalization factor of each Feynman diagram is
purely determined by the number and species of external particles. In
order to do so, we count vertices and
propagators of QCD tree amplitudes and multiply them with the corresponding
renormalization factors,
cf.~Eqs.~\eqref{eq:vertqcdct}-\eqref{eq:repropg}. We start with pure gluon amplitudes with $n$
external particles. For each Feynman diagram contributing to the
amplitude, we have
\begin{align}\label{eq:gvert}
  n = n_3 + 2 n_4 + 2,
\end{align}
where $n_3$ and $n_4$ are the number of three- and four-valent gluon
interactions. The number of gluon propagators $n_{\text{p,g}}$ is
related to the number of vertices by
\begin{align}\label{eq:gprop}
  n_{\text{p,g}} = n_3+ n_4 -1.
\end{align}
Therefore, the renormalization constants multiplying each Feynman
diagram are given by
\begin{align}
\begin{split}
  \text{R}(n,n_3,n_4,n_{\text{p,g}}) &= n_4 \left( 2\delta_3 + 2\delta_g \right) +n_3 \left(
    \frac{3}{2} \delta_3 + \delta_g \right)-n_{\text{p,g}} \delta_3\\
%&= n_4 \left( 2\delta_3 + 2\delta_g \right) +( n - 2 n_4 - 2) \left(
%    \frac{3}{2} \delta_3 + \delta_g \right)-(n - n_4 -3)\delta_3 \\
&= \frac{n}{2}\delta_3 +(n-2)\delta_g \equiv \text{R}(n),
\end{split}
\end{align}
where we used Eqs.~\eqref{eq:gvert} and \eqref{eq:gprop} as well as
the counterterm Feynman rules. Thus the renormalization constants
multiplying each Feynman diagram are only dependent on the number of
external legs. Wave-function
and coupling renormalization of pure gluon amplitudes are therefore
proportional to the corresponding tree
amplitude. 

We perform a similar counting exercise for processes involving both quarks and gluons. For a Feynman diagram of $n$ particles, with $N_g$ gluons, $N_q$ quarks, $n_3$ and $n_4$ gluon vertices and
$n_{\text{qg}}$ quark-gluon vertices, we get the relation
\begin{align}\label{eq:rqgen}
  n=N_g+N_q = n_3 + 2n_4 + n_{\text{qg}} +2.
\end{align}
The number of both gluon ($n_{\text{p,g}}$) and quark ($n_{\text{p,q}}$) propagators is given by
\begin{align}\label{eq:propqre}
  n_{\text{p,g}} &= n_3+n_4 + \left(\frac{N_q}{2} - 1\right), &   n_{\text{p,q}} &=
  n_{\text{qg}}-1- \left( \frac{N_q}{2} - 1\right).
\end{align}
The multiplicative renormalization factor $\text{R}(N_g,N_q,n_3,n_4,n_{\text{qg}},n_{\text{p,g}},n_{\text{p,q}})$ of each Feynman diagram with $N_g$ gluons and
$N_q$ quarks therefore reads
\begin{align}\label{eq:shiftfd}
\begin{split}
 \text{R}(\cdots) &= n_4 \left( 2\delta_3 + 2\delta_g \right) +n_3 \left(
    \frac{3}{2} \delta_3 + \delta_g \right)+n_{\text{qg}}\left(\delta_2+\frac{1}{2}\delta_3+\delta_g\right)-n_{\text{p,g}} \delta_3-n_{\text{p,q}} \delta_2\\
%&= n_4 \left( 2\delta_3 + 2\delta_g \right) +( n - 2 n_4 -n_{\text{qg}}+ (N_q-2)) \left(\frac{3}{2} \delta_3 +\delta_g \right)+n_{\text{qg}}(\delta_2+\frac{1}{2}\delta_3+\delta_g)\\
%&\phantom{m}-\left(n -n_4-n_{\text{qg}} + \frac{1}{2}(m - 2)\right) \delta_3-\left(n_{\text{qg}}-\frac{m}{2} \right) \delta_2\\
&=\frac{N_g}{2}\delta_3
+\frac{N_q}{2}\delta_2+\left(n-2\right)\delta_g \\
&\equiv \text{R}(N_g,N_q),
\end{split}
\end{align}
where we have used Eqs.~\eqref{eq:rqgen} and \eqref{eq:propqre} and
the counterterm Feynman rules. As above, the wave-function and coupling renormalization shift for
amplitudes containing quarks and gluons is proportional to a
tree amplitude.

\subsection{Renormalization Constants}
\subsubsection{Gluon Wave-Function Renormalization}
We renormalize the gluon wave function in the on-shell scheme. The
renormalization constant is fixed by the requirement that the residue
of the gluon propagator is one. As a side effect, we do not have
to calculate self-energy corrections on external gluon legs. The
renormalization constant in the on-shell scheme calculated in FDH is
given by
\begin{align}
Z_3^{\text{os}}=1+\delta_{3}=1-g_s^2c_\Gamma\frac{2}{3}\left(\frac{1}{\epsilon} +
  \log\left(\frac{\mu^2}{ m_{\text{os}}^2}\right)\right)+\mathcal{O}(g_s^4,\epsilon),
\end{align}
with a heavy quark with on-shell mass $m_{\text{os}}$ running in the
closed loop and the prefactor $\displaystyle
c_\Gamma=(4\pi)^{-(2-\epsilon)}\Gamma(1+\epsilon)\Gamma^2(1-\epsilon)/\Gamma(1-2\epsilon)$
appearing in all integrals and renormalization constants. As we saw
in Eq.~\eqref{eq:shiftfd}, each of the $N_g$ external gluons contributes a
factor of $\frac{1}{2}\delta_3$ and we get contributions from each
flavor in closed massive quark loops. Therefore the
renormalization shift to a \ola~involving $n$ particles and $N_g$ gluons is given by
\begin{align}
\begin{split}
  \text{R}_{\text{wf, gluon}} &= -g_s^{2}c_\Gamma
N_g\sum_{i=1}^{N_{hf}}\left(\frac{1}{3\epsilon} + 
  \frac{1}{3}\log\left(\frac{\mu^2}{ m_{i,\text{os}}^2}\right)\right)
\Ampt(1,\dots,n)\\
&=-g_s^{2}c_\Gamma
N_g \Delta_3 \Ampt(1,\dots,n),
\end{split}
\end{align}
where $N_{hf}$ denotes the number of heavy-quark flavors with mass $m_{i,\text{os}}$.
%There is no contribution from massless quarks to the gluon
%self-energy, since the corresponding integrals vanish in dimensional regularization.
\subsubsection{Massive Quark Wave-Function Renormalization}
 We use the on-shell scheme to renormalize the massive quark wave
 function and the renormalization constant in FDH is given by
\begin{align}
 Z_2^{\text{os}}=1+\delta_2&=1-g_s^2c_\Gamma C_F\left(\frac{3}{\epsilon}+3\log{\left(\frac{\mu^2}{m_{\text{os}}^2}\right)}+5\right)+\mathcal{O}(g_s^4,\epsilon),
\end{align}
with $C_F=\frac{N_C^2-1}{2N_C}$. 
%The constant $Z_2^{\text{os}}$ is fixed from requirements on the fermion propagator Eq.~\eqref{eq:osfermprop}.
%\begin{align}
% \lim_{p^2\rightarrow m^2_{\text{os}}}\left[\frac{i(\slashed{p}+m_{\text{os}})}{p^2-m^2_{\text{os}}}\left(1-{\hat\Sigma}_0(\slashed{p})-\frac{m_{\text{os}}(\slashed{p}+m_{\text{os}})\tilde{\Sigma}_1(\slashed{p})}{p^2-m_{\text{os}}^2}\right)u(p)\overset{!}{=}\frac{i(\slashed{p}+m_{\text{os}})}{p^2-m^2_{\text{os}}}u(p)\right],
%\end{align}
%which can be rephrased as
%\begin{align}
%  \hat\Sigma_0(m_{\text{os}}^2)+2m^2_{\text{os}}\tilde{\Sigma}^\prime_1(m_{\text{os}}^2)=0,
%\end{align}
%with the self-energies $\Sigma_i$ defined as in Sec.~\ref{sec:mrenorm} and
%$\Sigma^\prime(\slashed{p})=\pdv{\Sigma(\slashed{p})}{\slashed{p}}$. 
Each massive external fermion
contributes a factor of $\frac{1}{2}\delta_2$, see Eq.~\eqref{eq:shiftfd}. Therefore, the
renormalization shift to a \ola~involving $n$ particles including external massive quarks is given by
\begin{align}
\begin{split}
  \text{R}_{\text{wf, quark}} &= -g_s^{2}c_\Gamma
\sum_{i=1}^{N_{hf}}N_{Q_i}\frac{1}{2}C_F\left(\frac{3}{\epsilon}+3\log{\left(\frac{\mu^2}{m_{i,\text{os}}^2}\right)}+5\right)\Ampt(1,\dots,n)\\
&= -g_s^{2}c_\Gamma
\sum_{i=1}^{N_{hf}}N_{Q_i}\frac{\Delta_{2,i}}{2}\Ampt(1,\dots,n),
\end{split}
\end{align}
where $N_{hf}$ denotes the number of heavy-quark flavors and $N_{Q_i}$ the number of external quarks of flavor $i$ with on-shell mass $m_{i,\text{os}}$.

%See for example:\\
%KST, \url{https://arxiv.org/pdf/hep-ph/9305239.pdf}\\
%\url{https://arxiv.org/pdf/0807.4424.pdf}\\
%\url{https://arxiv.org/pdf/hep-ph/9310301.pdf}\\
%Weinzierl, \url{https://arxiv.org/pdf/hep-ph/0207055.pdf}

\subsubsection{Coupling Renormalization}
The coupling renormalization constant is fixed by a
calculation of the
vacuum polarization. We use the $\MSb$ scheme, where the
renormalization constant for the coupling computed in FDH is given by
\begin{align}
 Z_g^{\MSb}=1+\delta_g=1-g_s^2c_\Gamma\frac{1}{2}\left(\frac{11\NC-2(\NF+N_{hf})}{3\epsilon}-\frac{\NC}{3}\right)
 + \mathcal{O}(\epsilon,g_s^4),
\end{align}
with $\NF$ denoting the number of light and $N_{hf}$ the
number of heavy-quark flavors. The finite shift $\frac{\NC}{3}$ stems
from the translation of the gauge coupling from standard $\MSb$ to the
FDH variant. As we saw in Eq.~\eqref{eq:shiftfd}, each power $N_{g_s}$ of the strong coupling $g_s$ contributes to the renormalization shift
\begin{align}
\begin{split}
 \text{R}_{\text{coupling}} &= -g_s^{2}c_\Gamma \frac{N_{g_s}}{2}
 \left(\frac{11\NC-2(\NF+N_{hf})}{3\epsilon}-\frac{\NC}{3}\right)\Ampt(1,\dots,n)\\
 &= -g_s^{2}c_\Gamma N_{\alps} \Delta_{\alps}\Ampt(1,\dots,n).
\end{split}
\end{align}
For pure QCD amplitudes, the power of $g_s$ is given by
$N_{g_s}=n-2$ and that of $\alps$ by $N_{\alps}=N_{g_s}/2$. For amplitudes
involving \ew~gauge bosons, the counting is reduced accordingly.
\subsubsection{Heavy-Quark Decoupling}
We work in the decoupling scheme \cite{Collins1978a}. That is, we require
that heavy quarks decouple from the running of
$\alpha_s$ at energies $E\ll m_{hf}$. Consequently, the appropriate $\MSb$ coefficients for the
running of $\alpha_s$ should be the same as in absence of the heavy
quarks. Therefore, diagrams with a heavy quark loop
are subtracted at zero momentum transfer and the decoupling shift for an
$n$-particle amplitude with $\frac{n-2}{2}$ powers of the strong
coupling $\alps$ and $N_{hf}$ decoupled heavy-quark flavors is
given by
\begin{align}
\begin{split}
\text{R}_{\text{decoupling}}&= g_s^{2}c_\Gamma
N_{\alps}\sum_{i=1}^{N_{hf}}
\frac{2}{3}\log(\frac{\mu^2}{m_{i,\text{os}}}) \Ampt(1,\dots,n)\\
&= -g_s^{2}c_\Gamma N_{\alps}\sum_{i=1}^{N_{hf}}
\Delta_i\Ampt(1,\dots,n),
\end{split}
\end{align}
where the sum runs over heavy quark flavors $N_{hf}$ and $N_{\alps}$ denotes the power of the coupling $\alps$. For processes involving \ew~gauge bosons, the counting is
reduced accordingly.
%\url{http://lanl.arxiv.org/pdf/hep-ph/0508242}
%\url{https://ac.els-cdn.com/0550321382900384/1-s2.0-0550321382900384-main.pdf?_tid=3fc95d9e-a468-11e7-b8d4-00000aacb362&acdnat=1506615511_0873854316f0722bef4a45c5ee57e85d}

\section{Mass Renormalization}
\label{sec:mrenorm}
For processes involving massive quarks, we need to renormalize the
bare mass parameter. Its renormalization cannot be represented
as a contribution proportional to the tree amplitude. For convenience we combine the computation of these
contributions with the bubble diagrams. We explicitly
compute mass-counterterm contributions using a dedicated recursive
tree-like computation at the level of primitive loop amplitudes. The only source to the $\delta_m$ counterterm is the two-particle QCD counterterm interaction, cf.~Fig.~\ref{fig:qcdct} and Eq.~\eqref{eq:vertqcdct} 
\begin{align}\label{eq:fermionpropins}
  P^q_{\text{ct}}(p)_{|\delta_m} = -im_R \delta_m.
\end{align}
For double cut topologies with a gluon and a massive quark
cut line, the mass counterterm is computed by replacing the two cut lines of the bubble with fermion propagators and the above insertion of
Eq.~\eqref{eq:fermionpropins} as shown in Fig.~\ref{fig:massct}. The counterterm for a double cut topology
with legs $1_g,\dots,j_{\bar{q}}$ joining in vertex one and
$(j+1)_g,\dots,n_q$ in vertex two is thus computed by
\begin{align}
  \text{CT}_{1_g,\dots,j_{\bar{q}};(j+1)_g,\dots,n_q} =
  Q(1_g,\dots,j_{\bar{q}}) P^q_{\text{ct}}(P_{1j})_{|\delta_m} \bar{Q}((j+1)_g,\dots,n_q),
\end{align}
where we denoted the momentum sum $P_{ij}=\sum_{k=i}^jp_k$ and $Q$ and
$\bar{Q}$ denote quark and anti-quark Berends-Giele currents respectively. By adding these
counterterms to each color-ordered \ola, gauge invariance is
restored at the level of primitive amplitudes, after initially being broken by the removal of external leg self-energy insertions in \olb~cuts, cf.~Sec.~\ref{sec:massivebubble}.
\begin{equation*}
\begin{tikzpicture}[baseline=(m1)]
  \def\leglength{1}
  \def\blobpos{2.5}
  \def\cutshift{0.3}
  \def\cutlength{1.0}

  %The double cut lines
  \pgfmathsetmacro\vertcut{(\blobpos-1)/2+1}
  \draw[very thick] (\vertcut,\cutshift) -- (\vertcut,\cutshift+\cutlength);
  \draw[very thick] (\vertcut,-\cutshift) -- (\vertcut,-\cutshift-\cutlength);

  %\node at (-0.3,0) {\(1_{\bar{q}}\)};
  \begin{feynman}[inline=(m1)]
    \vertex[blob] (m) at (\blobpos, 0) {};
    \vertex[blob] (m1) at (1, 0){};
    \vertex (i) at (0, 0){\(\vdots\)};
    \vertex (i1) at (0, \leglength){\(j_{\bar{q}}\)};
    \vertex (i2) at (0, -\leglength){\(1_g\)};
    \vertex (a) at (\blobpos+\leglength,0){\(\vdots\)};
    \vertex (b) at (\blobpos+ \leglength,\leglength){\( (j+1)_{g} \)};
    \vertex (c) at (\blobpos+ \leglength,- \leglength){\(n_{q}\)};
    \diagram* {
     (m1) -- [gluon, half left,out=90,in=90] (m)
       -- [half left, thick,out=85,in=95] (m1),
       (m) -- [gluon] (b),
       (m1) -- [anti fermion] (i1),
       (m1) -- [gluon] (i2),
      (m) -- [fermion,  thick] (c),
    };
  \end{feynman}
\end{tikzpicture}
\hspace{0.5cm}\longrightarrow\hspace{0.5cm}
\begin{tikzpicture}[baseline=(m1)]
  \def\leglength{1}
  \def\blobpos{3.5}
  \def\cutshift{0.3}
  \def\cutlength{1.0}

  %\node at (-0.3,0) {\(1_{\bar{q}}\)};
  \begin{feynman}[inline=(m1)]
    \vertex[blob] (m) at (\blobpos, 0) {};
    \vertex[blob] (m1) at (1, 0){};
    \vertex (mi) at (2.25, 0);
    \vertex (i) at (0, 0){\(\vdots\)};
    \vertex (i1) at (0, \leglength){\(j_{\bar{q}}\)};
    \vertex (i2) at (0, -\leglength){\(1_g\)};
    \vertex (a) at (\blobpos+\leglength,0){\(\vdots\)};
    \vertex (b) at (\blobpos+ \leglength,\leglength){\( (j+1)_{g} \)};
    \vertex (c) at (\blobpos+ \leglength,- \leglength){\(n_{q}\)};
    \diagram* {
      (m1) -- [fermion,insertion={[size=8pt]0.99}] (mi)   -- [fermion] (m),
       (m) -- [gluon] (b),
       (m1) -- [anti fermion] (i1),
       (m1) -- [gluon] (i2),
      (m) -- [fermion,  thick] (c),
    };
  \end{feynman}
\end{tikzpicture}
\end{equation*}
\vspace{-1.2cm}
\captionedequationset{Generation of mass counterterms. For double cut
  topologies with a gluon and fermion cut line, we compute the
  corresponding counterterm by joining the two currents
  $Q(1_g,\dots,j_{\bar{q}})$ and $\bar{Q}((j+1)_g,\dots,n_q)$ with the
  fermion two-particle counterterm interaction.\label{fig:massct}} 


We renormalize the mass in the \textit{on-shell scheme}. The renormalization
constant $Z_m$, computed in Feynman gauge and expressed in FDH is given by
\begin{align}\label{eq:rcm}
\begin{split}
  Z_m^{\text{os}}=1+\delta_m&=1-C_Fg_s^2c_\Gamma
  \left(\frac{\mu^2}{m_{\text{os}}^2}\right)^\epsilon
  \left(\frac{3}{\epsilon}+5\right)+\mathcal{O}(g_s^4)\\
&=1-C_Fg_s^2c_\Gamma \left(\frac{3}{\epsilon}+3\log{\left(\frac{\mu^2}{m_{\text{os}}^2}\right)}+5\right)+\mathcal{O}(g_s^4,\epsilon).
\end{split}
\end{align}
The color factor $C_F$ evaluates to $C_F=1$ if we compute the counter
terms for color-ordered particles. In the
on-shell scheme, the renormalization constant
$Z_m^{\text{os}}$ is fixed such that the pole of the
propagator is described by the renormalized mass
$m_{\text{os}}$, thereby fixing the finite terms. 

%Summing up the one-particle irreducible contributions
%to the fermion propagator up to first order, we get
%\begin{align}\label{eq:osfermprop}
%\begin{split}
%  \hat{P}^q(p)&=\frac{i(\slashed{p}+m_R)}{p^2-m^2_R}\left(1-\frac{(\slashed{p}+m_R)\hat\Sigma(\slashed{p})}{p^2-m_R^2}\right)\\
%&=\frac{i(\slashed{p}+m_R)}{p^2-m^2_R}\left(1-{\hat\Sigma}_0(\slashed{p})-\frac{m_R(\slashed{p}+m_R)\tilde{\Sigma}_1(\slashed{p})}{p^2-m_R^2}\right),
%\end{split}
%\end{align}
%where we denote renormalized quantities with a hat. The renormalized
%self-energy $\hat\Sigma(\slashed{p})$ can be split into vectorial and scalar part
%according to $\hat\Sigma(\slashed{p})=\slashed{p}{\hat\Sigma}_0(\slashed{p})+m{\hat\Sigma}_1(\slashed{p})$ and the sum of the
%parts is denoted by
%$\tilde{\Sigma}_1(\slashed{p})={\hat\Sigma}_0(\slashed{p})+{\hat\Sigma}_1(\slashed{p})$. If we then require
%to have a simple pole in Eq.~\eqref{eq:osfermprop} for $p^2\rightarrow m_R\equiv m_{\text{os}}$, we obtain the renormalization condition
%\begin{align}\label{eq:rnc1}
%  \lim_{p^2\rightarrow m^2_{\text{os}}}\tilde{\Sigma}_1(p^2) = \tilde{\Sigma}_1(m_{\text{os}}^2)  = 0.
%\end{align}
%The sum of renormalized self-energies
%${\tilde\Sigma}_1(m_{\text{os}}^2)$ written in terms of unrenormalized
%self-energies is given by
%\begin{align}
%\begin{split}
%  \tilde{\Sigma}_1(m_{\text{os}}^2) &=
%  {\hat{\Sigma}}_0(m_{\text{os}}^2)+{\hat \Sigma}_1(m_{\text{os}}^2) \\
%&=  {\Sigma}_0(m_{\text{os}}^2)+{\Sigma}_1(m_{\text{os}}^2)-\delta_{m}^{\text{os}}=0,
%\end{split}
%\end{align}
%which fixes the finite part of $Z_m^{os}\equiv
%1+\delta_{m}^{\text{os}}$. It evaluates in FDH to Eq.~\eqref{eq:rcm}, given the expressions for the unrenormalized quark
%self-energies in QCD.



%\subsection{$\gamma_5$ renormalization}
%In the HV scheme we do have a finite shift for $gamma_5$.
%\begin{align}
%  \gamma^\mu\gamma_5 \rightarrow   \frac{1}{2} Z_{axial}(\gamma^\mu\gamma_5 -  \gamma_5\gamma^\mu)\notag
%\end{align}
%with $Z_{axial}=1-2C_F 1_{HV}$\\
%\url{https://arxiv.org/pdf/hep-ph/9903380.pdf}

\section{Summary of Renormalization}
\label{sec:summary}
In summary, we renormalize \ola s in a mixed scheme. For external states we use the on-shell
scheme and the gluon wave-function renormalization receives contributions from all
active heavy quark flavors. We renormalize the mass in the on-shell scheme
and compute mass renormalization counterterms
explicitly at the level of each color-ordered amplitude. The QCD coupling is renormalized in $\MSb$, where we decouple massive quarks from the
running of $\alpha_s$. The renormalized \ola~is obtained by combining all tree-like renormalization shifts with the mass renormalized \ola
\begin{align}
\begin{split}
  \Amp_{(ren)}&=
  \Amp_{(mass\
    ren)}+\text{R}_{\text{wf,quark}}+\text{R}_{\text{wf,gluon}}+\text{R}_{\text{coupling}}+\text{R}_{\text{decoupling}}\\
&= \Amp_{(mass\
    ren)}-g_s^2 c_\Gamma \left( \sum_{i=1}^{N_{hf}} N_{Q_i} \frac{\Delta_{2,i}}{2} + N_{g}\Delta_3 +
    N_{\alps}\left(\Delta_{\alps} + \sum_{i=1}^{N_{hf}}\Delta_i\right) \right)  \mathcal{A}^{\text{tree}},
\end{split}
\end{align}
where $N_{hf}$ denotes the number of heavy flavors, $N_{Q_i}$ the number of external heavy quarks of flavor $i$, $N_g$ the number of external
gluons and $N_{\alps}$ the power of $\alps$ (at Born level) . The
renormalization constants in the FDH scheme are summarized in Table \ref{tab:renorm}.
\begin{table}[h]
  \caption{Renormalization constants used in this thesis. Here $\mu$ is the renormalization
  scale, $m_{i}$ are the masses for heavy quarks, $\NF$ is the number of light flavors,
  $N_{hf}$ the number of heavy flavors, and $\NC$ the number of colors.
  A common factor of $-g_s^2 c_\Gamma$ has been factored out.}
      \vskip 4mm
  \centering
    \begin{tabularx}{\textwidth}{lcll}
      \hline\hline
      \noalign{\vskip 4mm}
      \textbf{Renormalization} & \textbf{Scheme} & \textbf{Constant}\\
      \noalign{\vskip 3mm}
      \hline
      \noalign{\vskip 2mm}
      %\rule{0pt}{1ex}\\
    %\small
      %\toprule
      Heavy-quark wave function   & on-shell & $\displaystyle
      \Delta_{2,i} ~=~ \frac{\NC^2-1}{2\NC} \left( \frac{3}{\epsilon}
        + 3 \log{\frac{\mu^2}{m_i^2} + 5} \right)$\\
      \noalign{\vskip 1mm}
      Light-quark wave function   & on-shell & 0\qquad(UV+IR
      cancellation) \\
      \noalign{\vskip 3mm}
      Quark mass            & on-shell & $\displaystyle \Delta_{m_i}
      ~=~ \Delta_{2,i}\quad\text{}$\\
      \noalign{\vskip 1mm}
      Gluon wave function   & on-shell & $\displaystyle \Delta_3 ~=~ \sum_{i=1}^{N_{hf}}\left(\frac{1}{3 \epsilon} +
      \frac{1}{3}\log{\frac{\mu^2}{m_i^2}}\right)$\\
      \noalign{\vskip 1mm}
      QCD coupling & $\MSb$ & $\displaystyle \Delta_{\alps} ~=~  \frac{1}{\epsilon} \left( \frac{11}{3}\NC - \frac{2}{3}(\NF+N_{hf}) \right) - \frac{\NC}{3}$\\
      \noalign{\vskip 2mm}
      \hline
      \noalign{\vskip 2mm}
      Decoupling shift & --- & $\displaystyle     \Delta_i ~=~-
      \frac{2}{3}\log{\frac{\mu^2}{m_i^2}} $\\
      \noalign{\vskip 1mm}
      \hline\hline
    \end{tabularx}
  \label{tab:renorm}
\end{table}

\section{Scheme Shift From FDH to 't Hooft-Veltman}
\label{sec:schemeshift}
We use the FDH variant of dimensional regularization in intermediate steps to
regularize UV and IR divergences. At the end we convert the renormalized
amplitude to the HV scheme~\cite{tHooft:1972tcz} which is often more convenient for comparisons and interfacing with Monte-Carlo generators. The conversion of a renormalized \ola~$\Amp_{(ren)}$ is performed by a finite shift, see
e.g.\ \cite{Signer:2008va}
\begin{align}
  \mathcal{A}_{(ren)}^{\text{1-loop,HV}} =
  \mathcal{A}_{(ren)}^{\text{1-loop,FDH}} - g_s^{2}c_\Gamma\left(N_g\frac{\NC}{6}+N_q\frac{\NC^2-1}{4\NC}\right)\Ampt,
\end{align}
where $N_g$ denotes the number of gluons and $N_q$ the number of light
quarks in the respective amplitude.



\section{Scale Variation Using \Ntuple{} Data}
\label{sec:scaledep}

We provide NLO results in the form of \root~\cite{ROOT} \ntuple{} files \cite{BH:Ntuples}. The \ntuple{} files allow to reevaluate
generic IR-safe observables with different renormalization and
factorization scales as well as different PDFs. They store the relevant information to
reevaluate observables without recomputing the short-distance matrix
elements, which is possible due to the particular simple form of the parts of the virtual matrix element that depend on $\mu_R$. This possibility for a-posteriori variations of scales and PDFs
make the efficient estimation of uncertainties associated to NLO
predictions possible. 

In this section, we analyze the dependence of virtual cross sections on the unphysical renormalization scale. We neglect the implicit dependence
via the strong coupling for this discussion, since it amounts to a global prefactor and can be
treated as described in \cite{BH:Ntuples}. In particular, we identfiy additional
renormalization terms with $\mu_R$ dependence which are present in calculations with massive external particles and have to be considered in order to supply the correct data for \ntuple{} files.

The dependence of the virtual cross section on the unphysical scale
$\mu_R$ is introduced in
dimensional regularization to enforce a mass dimension of one for the
gauge 
couplings $g\rightarrow \mu_R^{\epsilon}g$. As a consequence, there is
an explicit dependence on $\mu_R$ in virtual NLO matrix elements. We can write an unrenormalized, one-loop
amplitude, with couplings stripped off, as
\begin{align}\label{eq:mudep}
  \Amp(\mu_R) =  \tilde{\mu}_R^{2\epsilon}\left[\frac{a_{2}}{\epsilon^2}+\frac{a_{1}}{\epsilon}
  + a_0\right],
\end{align}
where we use the dimensionless scale $\tilde{\mu}_R =
\frac{\mu_R}{\text{GeV}}$ to simplify the argument in the following. We generically have ratios of
renormalization scale $\mu_R$ and involved physical scales, from which
we build dimensionless quantities by
\begin{align}
 \left(\frac{\mu_R}{s}\right)^{2\epsilon} =\tilde{\mu}_R^{2\epsilon}\left(\frac{\text{GeV}}{s}\right)^{2\epsilon}.
\end{align}
Note, that Eq.~\eqref{eq:mudep} is valid for
any number of involved physical scales.
%\footnote{Assume we had two scales, the initial expression expression
%would be $ \sigma_V =
%(\frac{\mu_R}{sc_1})^{2\epsilon}\left[\frac{b_{2}}{\epsilon^2}+\frac{b_{1}}{\epsilon}
%+ b_0\right]+
%(\frac{\mu_R}{sc_2})^{2\epsilon}\left[\frac{c_{2}}{\epsilon^2}+\frac{c_{1}}{\epsilon}
%+ c_0\right]$. After identifying two identical dimensionless scales,
%we see that a redefinition of the parameters $b_i$ and $c_i$ leads to
%the form in Eq.~\eqref{eq:mudep}.}. 
We then expand $\left(\tilde{\mu}_R\right)^{2\epsilon}$ and obtain
\begin{align}
   \tilde{\mu}_R^{2\epsilon} = 1 +
   \log(\murt)\epsilon+\frac{1}{2}\log(\murt)^2\epsilon^2
   + \mathcal{O}(\epsilon^3),
\end{align}
such that the explicit $\tilde \mu_R$ dependence of
Eq.~\eqref{eq:mudep} becomes transparent
\begin{align}\label{eq:unrenam}
    \Amp(\mu_R)  = \left[\frac{a_{2}}{\epsilon^2}+\frac{a_{1}+a_{2}\log(\murt)}{\epsilon}
  + a_0+a_{1}\log(\murt)+\frac{1}{2}a_{2}\log(\murt)^2\right].
\end{align}
This formula contains double and single poles due to infrared
divergencies as well as single poles due to ultraviolet
divergencies. The difference of the finite part of the amplitude computed at two distinct scales $\mu_{R,1}$ and $\mu_{R,2}$ is then given by
\begin{align}\label{eq:extrapolscale}
\begin{split}
\Amp_{|\epsilon^{0}}({\mu}_{R,2}) -\Amp_{|\epsilon^{0}}({\mu}_{R,1})
  &=  a_1\log(\frac{\tilde{\mu}_{R,2}^2}{\tilde{\mu}_{R,1}^2})
+\frac{1}{2} a_2 \left[
    \log(\tilde{\mu}_{R,2}^2)^2-
    \log(\tilde{\mu}_{R,1}^2)^2\right]\\
&=  \left[a_1+a_2\log(\tilde{\mu}_{R,1}^2)\right]\log(\frac{\tilde{\mu}_{R,2}^2}{\tilde{\mu}_{R,1}^2})
+\frac{1}{2} a_2
\log(\frac{\tilde{\mu}_{R,2}^2}{\tilde{\mu}_{R,1}^2})^2,
\end{split}
\end{align}
which we can parameterize in terms of the weights $w_1$ and $w_2$
\begin{align}\label{eq:unrenweights}
  w_1 &\equiv a_{1}+a_2
  \log(\tilde{\mu}_{R,1}^2)=\Amp_{|\epsilon^{-1}},& w_2 &\equiv a_{2} = \Amp_{|\epsilon^{-2}}. 
\end{align}
The above parameterization has the advantage, that only ratios of
renormalization scales have to be considered and no explicit
dimensionless scales have to be constructed. The finite part of an unrenormalized \ola~at scale
$\tilde{\mu}_{R,2}$ can thus be extrapolated from that at scale
$\tilde{\mu}_{R,1}$ by adding the terms in Eq.~\eqref{eq:extrapolscale}
\begin{align}\label{eq:extrapolscale2}
\begin{split}
  \Amp_{|\epsilon^{0}}({\mu}_{R,2})   &= \Amp_{|\epsilon^{0}}({\mu}_{R,1}) +
  w_1\log(\frac{\tilde{\mu}_{R,2}^2}{\tilde{\mu}_{R,1}^2})+
\frac{1}{2} w_2
\log(\frac{\tilde{\mu}_{R,2}^2}{\tilde{\mu}_{R,1}^2})^2.
\end{split}
\end{align}
The above considerations are independent
of whether the involved particles are massive or not. In the
renormalization procedure, see the previous sections, additional
sources of $\mu_R$ dependence are introduced. Whereas mass and wave-function
renormalization lead to terms in which the logarithmic dependence in
the finite part has the same prefactor as the single pole, the shift
due to charge renormalization has no dependence on $\mu_R$ but
contributes to the single pole. Its
schematic contributions in an $\epsilon$ expansion
$(\epsilon^{-1},\epsilon^0)$ is given by
\begin{align}
   \text{R}_{\text{charge}} & = (a_{1,c},a_{0,c}),
\end{align} 
with coefficients $a_i$ that do not contain any logarithms. The decoupling
shift for massive quarks however has finite logarithms but no contribution
to the single pole
\begin{align}
   \text{R}_{\text{decoupling}} & = (0,a_{0,dec}\log(\murt)).
\end{align}
The weight $w_1$ should correctly parameterize simple logs of $\mu_R$ in
the finite part. To account for the absence of $\mu_R$ dependence in the charge renormalization and the fact that the decoupling shift
does not have a single-pole contribution, the weights for renormalized
amplitudes are defined as 
\begin{align}\label{eq:reweights}
 w_1 &=\Ampr_{|\epsilon^{-1}} -a_{1,c}+ a_{0,dec}, &  w_2 &= \Ampr_{|\epsilon^{-2}}.
\end{align}
With these weights, one can use Eq.~\eqref{eq:extrapolscale} to
extrapolate the virtual cross section to different values of $\mu_R$
for processes involving massive particles.

We provide the weights in Eq.~\eqref{eq:reweights} for each
phase space point together with the finite part of the matrix element squared. They
are stored in \ntuple{} files alongside information like the phase-space
point, the involved partons, the factorization and renormalization
scales, the Bjorken-x and the corresponding PDF
weights. We make extensive use of a-posteriori scale variations in the phenomenological
results presented in the following Chapters \ref{chap:wbb_intro}-\ref{chap:vjet_result}.
