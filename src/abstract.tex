%A significant part of the physics program of the ongoing Large Hadron Collider experiment at CERN
%is in precise measurements. Confronted with precise theoretical predictions, these measurements allow 
%to scrutinize the Standard Model of particle physic and potentially uncover hidden signals of new physics. 
%The computations of theoretical predictions from the first principles of QFT are very complex and
%require development of sophisticated computational techniques and algorithms.
In this thesis, we develop and apply the $D$-dimensional numerical unitarity method for
the computation of multi-loop scattering amplitudes, which
are the key building blocks for precise theoretical predictions for the ongoing physics program of the Large Hadron Collider experiment at CERN.
We formulate a framework for consistently implementing dimensional regularization in numerical approaches, which in particular
can be applied for efficient numerical evaluations of dimensionally-regularized helicity amplitudes with external fermions.
We present the next-to-leading order QCD predictions for $Wb\bar{b}+n$-jet ($n=0,1,2,3$) production at the Large Hadron Collider,
which is an irreducible background to $H(\rightarrow b{\bar b})W$ studies.
We include all the effects of the non-vanishing bottom-quark mass. 
To obtain the one-loop matrix elements required for the process,
we implement an extension of the numerical unitarity method to massive particles in the loop.
We present the first benchmark values and analytic form of all two-loop five-parton helicity amplitudes in QCD in the leading-color approximation.
The analytic expressions are reconstructed from exact numerical evaluations with the numerical unitarity method.
Our results provide all two-loop
amplitudes required for the calculation of next-to-next-to-leading order
QCD corrections to the production of three jets at hadron colliders in
the leading-color approximation.

%subprocesses at leading electroweak order as well as all heavy-fermion-loop
%effects. We show the impact of QCD corrections for total as well as
%differential cross sections and make an assessment of theoretical
%uncertainties of \Wbb{} production viewed as an irreducible background to $H(\rightarrow b{\bar b})W$ studies. For the calculations we have
%employed an upgraded version of the \BlackHat{}
%library which can handle massive fermions.

