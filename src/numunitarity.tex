In this chapter we introduce the main computational methods we used to obtain results of
\cref{chap:wbb_pheno,chap:5parton}. 

We will describe a complete generalization of unitarity techniques to multi-loop level.

With aim for automation and fully numerical.

Some partial attempts can be found in 


In this chapter we do not refer to a particular theory.


Based on the papers

I want to take the pragmatic perspective. And in practice it is not even essential:
\begin{itemize}
    \item       the most practical and flexible way to compute them in $D$ dimensions is with off-shell recursions (see references e.g. in tails of 1001 gluons)
    \item Actually it is not even always the case as we see for massive particles.
\end{itemize}

LHS can be just an expression obtained from Feynman diagrams

However in perturbation theory the fact that the corners are tree amplitudes can be seen explicitly.
Because we collect together all graphs containing a particular singularity.


\section{Ansatz}
\label{sec:ansatz_integrand}

\subsection{Basis of Numerator Insertions}
Numerator tensors, span of function space. Master/Surface decomposition.

\todo{classification of tensors here or in chapter \ref{chap:stdtech}?}




\subsection{Traceless Completion}
\label{sec:traceless_completion}

Starting from two loops will have $D$-dependence in the numerator from IBP identities, so
we can choose to abandon all-together the idea of not having it there.

Also can be obtained with through a specially constructed IBP identities \cite{Ita:2015tya}.
%vectors of the form 
%\begin{equation}
  %v^\mu = (\sp(n_\epsilon^i,\ell))~\ell^\mu - (\sp(\tau,\ell))~n_\epsilon^{i\,\mu}
%\end{equation}


\subsection{Unitarity-Compatible IBP Identities}

\section{Cut Equations}

\subsection{On-Shell Loop-Momenta}

\todo{In $D$ dimensions we don't have to worry about branches of solutions, and also loop momenta have to be real}

\subsection{Evaluation of Cuts}

\subsection{Off-Shell Recurcion}
\label{sec:BG_recursion}

$D$-dimensional off-shell recursion, 


weyl and Dirac fermions, 
tracking of representations?


\subsection{Dimensional Reconstruction}
 In $D$ and $D_s$ (dimensional reduction later)


\section{Subleading Poles}
Here discussion about \textit{subleading poles}



\section{Examples}

\subsection{One-Loop Hierarchy}

Once the automated approach is set up, the one-loop case is so simple that we can
do the full one-loop hierarchy.

Compare to ``normal'' decomposition, we allow $D$-dependence in numerators. This might be not the best
idea for NLO computations since terms of order $\mathcal{O}(\epsilon)$ can be dropped.
It might be reasonable to do for higher-order computations where higher orders of $\epsilon$ are required.


Take 5-point hierarchy.

\subsubsection{Pentagon}

Take $\mu^2$ insertion.

\subsubsection{Box}

\subsubsection{Triangle}

Here also show that 1 and 2 mass triangles can be reduced

\subsubsection{Bubble}

See what happens with the light-like external momentum from the IBP perspective?
Should work out.

\subsubsection{Tadpole}

\subsection{Some Two-Loop Topology}

\todo{select a topology which is not too big to present} 

