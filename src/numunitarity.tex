In \cref{chap:stdtech} we discussed how a generic loop amplitude \eqref{eq:general_amplitude} can be reduced
to a sum of master integrals \eqref{eq:amplitude_integrated} in two stages.
First all reducible numerators are maximally canceled against corresponding denominators.
And after that the remaining numerators are reduced with IBP identities.
We indicated that the former can be rather challenging in some cases,
and the latter is one of the main bottlenecks in computations of
multi-scale two-loop amplitudes.
In this chapter we introduce a generalization of one-loop integrand-level unitarity-based methods \cite{Ossola:2006us,Giele:2008ve,Ellis:2008ir}
to two loops, which offers an alternative path to reduction of amplitudes and addresses the mentioned bottlenecks in an elegant way.
The foundations of this approach have been laid out in works \cite{Ita:2015tya,Abreu:2017xsl,Abreu:2017hqn}.
Notably it is designed to be amenable for automation and efficient numerical implementation, which is indispensable
for addressing the problem of large expressions in multi-scale problems.

Like in \cref{chap:stdtech}, here we will not refer to any particular theory.
%We also postpone the discussion of helicity amplitudes until \cref{chap:dshel}
%and assume that 
From now on we will also specialize to two loops whenever necessary,
although most results of this chapter
can be applied to any number of loops.

The computational methods described in this chapter are essential for obtaining the results we present in \cref{chap:wbb_pheno,chap:5parton}.

\hrule

I want to take the pragmatic perspective. And in practice it is not even essential:
\begin{itemize}
    \item       the most practical and flexible way to compute them in $D$ dimensions is with off-shell recursions (see references e.g. in tails of 1001 gluons)
    \item Actually it is not even always the case as we see for massive particles.
\end{itemize}

LHS can be just an expression obtained from Feynman diagrams

However in perturbation theory the fact that the corners are tree amplitudes can be seen explicitly.
Because we collect together all graphs containing a particular singularity.

\section{Ansatz}
\label{sec:ansatz_integrand}

We start by again considering a generic $L$-loop amplitude in \cref{eq:general_amplitude},
and  the classification of numerators from \cref{sec:classification_numerators}.
We organize all topologies hierarchically as described around \cref{eq:topology_order,eq:topology_sequence},
but we \emph{do not} cast integrals into families by allowing exponents $\gamma_j$ of propagators $\vb*{\rho}$ to be non-positive.
Instead we aim to construct a special basis of numerator functions of the form 
  
In this section we discuss the reduction at the integrand level and thus we drop the explicit integral symbol from all expressions.


for each topology independently.

Note that 





\subsection{Basis of Numerator Insertions}
Numerator tensors, span of function space. Master/Surface decomposition.

\todo{classification of tensors here or in chapter \ref{chap:stdtech}?}




\subsection{Traceless Completion}
\label{sec:traceless_completion}

Starting from two loops will have $D$-dependence in the numerator from IBP identities, so
we can choose to abandon all-together the idea of not having it there.

Also can be obtained with through a specially constructed IBP identities \cite{Ita:2015tya}.
%vectors of the form 
%\begin{equation}
  %v^\mu = (\sp(n_\epsilon^i,\ell))~\ell^\mu - (\sp(\tau,\ell))~n_\epsilon^{i\,\mu}
%\end{equation}


\subsection{Unitarity-Compatible IBP Identities}

\section{Cut Equations}

\subsection{On-Shell Loop-Momenta}

\todo{In $D$ dimensions we don't have to worry about branches of solutions, and also loop momenta have to be real}

\subsection{Evaluation of Cuts}

\subsection{Off-Shell Recurcion}
\label{sec:BG_recursion}

$D$-dimensional off-shell recursion, 


weyl and Dirac fermions, 
tracking of representations?


\subsection{Dimensional Reconstruction}
 In $D$ and $D_s$ (dimensional reduction later)


\section{Subleading Poles}
Here discussion about \textit{subleading poles}



\section{Examples}

\subsection{One-Loop Hierarchy}

Once the automated approach is set up, the one-loop case is so simple that we can
do the full one-loop hierarchy.

Compare to ``normal'' decomposition, we allow $D$-dependence in numerators. This might be not the best
idea for NLO computations since terms of order $\mathcal{O}(\epsilon)$ can be dropped.
It might be reasonable to do for higher-order computations where higher orders of $\epsilon$ are required.


Take 5-point hierarchy.

\subsubsection{Pentagon}

Take $\mu^2$ insertion.

\subsubsection{Box}

\subsubsection{Triangle}

Here also show that 1 and 2 mass triangles can be reduced

\subsubsection{Bubble}

See what happens with the light-like external momentum from the IBP perspective?
Should work out.

\subsubsection{Tadpole}

\subsection{Some Two-Loop Topology}

\todo{select a topology which is not too big to present} 

