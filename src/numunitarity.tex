\section{Unitarity Cuts, Optical Theorem and Factorization}

\section{Ansatz for the Integrand}

\section{Cut Equations}

In the van Neerven-Vermaseren (NV) construction
\cite{Neerven1984a}, the $D$-dimensional
space-time is decomposed into a physical space and its complement, the
transverse space. The NV construction is used for integrand
parameterizations \cite{Ellis:2007br,Ita:2011hi}. Consider $r$ inflow momenta $p_1,\dots,p_r$. The
physical space is then spanned by these $r$ momenta and has dimensions
$D_p=\min(D,r-1)$, since the $r$ momenta are in general linearly dependent and one degree of freedom is absorbed by momentum
conservation $\sum_{i=1}^rp_i = 0$. Whenever $r>D$, one can find
additional relations to reduce the higher point scalar integrals to
lower point ones \cite{Melrose1965}. For example, if $D=4$, one finds
a basis with a maximal rank of $4$. The physical space therefore forms
a lower dimensional subspace whenever $r\leq D$. The complent to the
physical space is called
transverse space and has dimension $D_t=\max(0,D-r+1)$, such that
\begin{align}
  D=D_p+D_t.
\end{align}
We assume that the momenta $p_1,\dots,p_r$ are ordered, such that the
first $D_p$ vectors are linearly independent. The corresponding dual
vectors are called $v_{1},\dots,v_{D_p}$. The transverse basis
vectors are denoted $n_{1},\dots,n_{D_t}$ and we have the following properties
\begin{align}
    n_i\cdot n_j &= \delta_{ij}, & n_i\cdot p_j &= 0,\\
   v_j \cdot n_i  &= 0, & v_j\cdot p_i &= \delta_{ij}.\notag
\end{align}
For the case that $r > D$, the space
is parameterized solely in terms of linearly independent vectors
$p_i$. 
 The metric tensor is thus decomposed in the NV basis as 
\begin{align}
  g^{\mu\nu} = \sum_{i=1}^{D_p}p_i^\mu v_i^\nu+\sum_{i=1}^{D_t}n_i^\mu n_i^\nu,
\end{align}
with the projector into the transverse space consquently given by
\begin{align}\label{eq:metricpron}
  (g_\perp)^{\mu\nu} \equiv  \sum_{i=1}^{D_t}n_i^\mu n_i^\nu = g^{\mu\nu}- \sum_{i=1}^{D_p}p_i^\mu v_i^\nu.
\end{align}
One can thus decompose the loop momentum in the NV basis as 
\begin{align}
  \ell^\mu = \sum_{i=1}^{D_p}(\ell\cdot p_i)
  v_i^\mu+\sum_{i=1}^{D_t}(\ell\cdot n_i) n_i^\mu,
\end{align}
with the projections of the loop momentum on the transverse space
$t_i(\ell) = (\ell \cdot n_i)$.  In actual computations with $D>4$, only a specific combination of the
projection on the transverse space is relevant. Since external
particles remain in $4$-dimensions, the additional components of the
loop momentum are only relevant in contractions of the loop momentum
with itself
\begin{align}
  \mu^2 = -\ell_{[D-4]}^2 \equiv \sum_{i=5}^D(\ell \cdot n_i)^2,
\end{align}
for explicit integer $D$-dimensions. Only this combination is of
relevance in $D\neq 4$-dimensional calculations. The $n_{i>4}$ are the transverse
vectors of the $D-4$-dimensional space. One can split up the
decomposition into a $4$-dimensional part and a $D-4$-dimensional part
and gets for the projector into the transverse space
\begin{align}\label{eq:metricprond}
\begin{split}
  (g_\perp)^{\mu\nu} &=  \sum_{i=1}^{D_t}n_i^\mu n_i^\nu=\sum_{i=1}^{4-r+1}n_i^\mu n_i^\nu +  \sum_{i=5}^{d}n_i^\mu n_i^\nu.
\end{split}
\end{align}


\section{Doubled Propagators}
Here discussion about \textit{subleading poles}

\section{Basis of Numerator Insertions}
Numerator tensors, span of function space. Master/Surface decomposition.

\todo{classification of tensors here or in chapter \ref{chap:stdtech}?}


\section{Dimensional Reconstruction}
 In $D$ and $D_s$ (dimensional reduction later)

\section{Evaluation of Cuts}
$D$-dimensional off-shell recursion, weyl and Dirac fermions, tracking of representations?



