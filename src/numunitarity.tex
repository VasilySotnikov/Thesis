In \cref{chap:stdtech} we sketched how a generic loop amplitude \eqref{eq:general_amplitude} can be reduced
to a sum of master integrals \eqref{eq:amplitude_integrated} in two stages.
First all reducible numerators are maximally canceled against corresponding denominators.
And after that the remaining numerators are reduced with IBP identities.
We indicated that the former can be rather challenging in some cases,
and the latter is one of the main bottlenecks in computations of
multi-scale two-loop amplitudes.
In this chapter we introduce a generalization of one-loop integrand-level unitarity-based methods \cite{Ossola:2006us,Giele:2008ve,Ellis:2008ir}
to two loops, which offers an alternative path to reduction of amplitudes and addresses the mentioned bottlenecks in an elegant way.
The foundations of this approach have been laid out in works \cite{Ita:2015tya,Abreu:2017xsl,Abreu:2017hqn,Abreu:2017idw}.
Notably it is designed to be amenable for automation and efficient numerical implementation, which is indispensable
for addressing the problem of large expressions in multi-scale problems.
\todo{Having fast numerical algorithm at hand allows in many cases to use numerical evaluations to exactly reconstruct analytic expressions.}

\todo{References for other partial developments in multi-loop generalized unitarity}

The approach we discuss in this chapter broadly speaking consists of two major ingredients.
This first is a special choice of integrand parametrization\footnote{sometimes also called \emph{ansatz}},
which we discuss in \cref{sec:ansatz_integrand}.
The second is exploiting integrand-level factorization limits of amplitudes originating from generalized unitarity (see \cref{sec:unitarity}),
which we discuss in \cref{sec:cut_equations}.

%Like in \cref{chap:stdtech}, here we will not refer to any particular theory.
%We also postpone the discussion of helicity amplitudes until \cref{chap:dshel}
%and assume that
%From now on we will also specialize to two loops whenever necessary,
%although most results of this chapter
%can be applied to any number of loops.

\todo{The computational methods described in this chapter are essential for obtaining the results we present in \cref{chap:wbb_pheno,chap:5parton}.}

\section{Master/Surface Integrand Parametrization}
\label{sec:ansatz_integrand}

We start by again considering a generic $L$-loop amplitude in \cref{eq:general_amplitude},
and  the classification of numerators from \cref{sec:classification_numerators}.
We organize all topologies hierarchically as described around \cref{eq:topology_order,eq:topology_sequence}.
but we \emph{do not} cast integrals into families by allowing exponents $\gamma_j$ of propagators $\vb*{\rho}$ to be non-positive.

Instead we will construct a special basis of numerator functions for each topology independently,
\begin{equation} \label{eq:master_surface_basis}
  \mathcal{N}_\Gamma = \sum_{i \in M_\Gamma \cup S_\Gamma}^{} c_{\Gamma, i}(\vb*{x},D) \; m_{\Gamma, i} (\vb*{x},D;\vb*{\rho},\vb*{\alpha}),
\end{equation}
where $S_\Gamma$ is a set of numerators which vanish upon integration, which we call \emph{surface terms}, $M_\Gamma$ is a set of numerators associated to master integrals.
Here we introduced a notation for distinguishing the arguments of $m_{\Gamma, i} (\vb*{x},D;\vb*{\rho},\vb*{\alpha})$,
which we will use throughout this chapter. 
It means that $m_{\Gamma, i}$ is a rational function of arguments on the left of ``;'', and \emph{polynomial} in the arguments on the right. 

We will refer to this basis as \emph{master/surface} basis or decomposition.
The advantage of having a master/surface basis is that
reducing an amplitude to it achieves
a complete reduction by construction,
and the coefficients $c_{\Gamma,i}$ for $i\in M_\Gamma$ are the coefficients
of master integrals in \cref{eq:amplitude_integrated}.
Also we do not need to consider topologies with $N_p > 4L + \frac{1}{2}L(L+1)$ since as we discussed in \cref{sec:classification_numerators}
they are reducible to topologies lower in the hierarchy.

Of course, this basis can be obtained from the standard approach of \cref{chap:stdtech}, which would not be a very illuminating exercise.
However the construction of master/surface decomposition can be cast into a much simpler computational problem than solving all IBP relations.
In what follows we review how this is accomplished.


\subsection{Transverse ISPs}
\label{sec:traceless_completion}

Consider a numerator $\mathcal{N}_\Gamma$ of any topology $\Gamma$ from $\Delta$ with $N_p <  4L + \frac{1}{2}L(L+1)$.

What is meant by a set of numerator insertions $m_{\Gamma,i}$ being a basis of numerator insertions is that
$\spn{m_{\Gamma,i}(\vb*{x},D,\vb*{\alpha};\vb*{\rho})}$ is equivalent as a linear vector space
modulo RSPs (i.e.\ with imposed the on-shell constraints $\rho_j = 0\;\forall j\in P_\Gamma$)
to a ring of polynomials $\mathcal{K}_{\vb*{x},D}[\vb*{\alpha}]$.

Recall that anything proportional to $\rho_j \in P_\Gamma$ is reducible to topologies lower in the hierarchy,
so the coefficients of the polynomials in $\vb*{\alpha}$ need not to depend on $\vb*{\rho}$ in order

\footnote{
  %more precisely: $\mathcal{K}_{\vb*{x},D;\vb*{\rho}}[\vb*{\alpha}]$ has coefficients from a ring $\mathcal{K}_{\vb*{x},D}[\vb*{\rho}]$,
  %which itself has coefficients from a field of rational functions $Q(\vb*{x},D)$
}, 

The linear vector space of monomials $\mathcal{K}_{\vb*{x},D;\vb*{\rho}}[\vb*{\alpha}]$ is of course infinite,
so we need to impose power counting constraints.


\todo{
  Somehow bring in that all $\rho$ are reduced (can be put to zero).
  Perhaps need to explain beforehand that we do not need to consider any RSPs.
  This requires to have cut equations before this section? Maybe it is better.
}

So it is convenient to start constructing the master/surface basis from a natural basis of monomials in $\mathcal{K}_{\vb*{x},D;\vb*{\rho}}[\vb*{\alpha}]$:
\begin{equation}
  \mathcal{N}_\Gamma = \sum_{\va{i}}^{} c_{\Gamma, \va{i}}(\vb*{x},D;\vb*{\rho}) \; m_\Gamma^{\va{i}}(\vb*{\alpha}), \qquad
  m_\Gamma^{\va{i}}(\vb*{\alpha}) = \vb*{\alpha}^{\va{i}} \equiv 
  %\rho_1^{i_1}\,\ldots\,\rho_{N_p}^{i_{N_p}} ~
  %\alpha_1^{i_{N_p+1}}\,\ldots{}\,\alpha_{N_\text{ISP} }^{i_{N_p+1 +N_\text{ISP} }}.
  \alpha_1^{i_1}\,\ldots{}\,\alpha_{N_\text{ISP} }^{i_{N_\text{ISP} }}.
\end{equation}

\todo{Important note here: coefficients $c_{\Gamma,i}(\vb*{x},D;\rho)$ are required to have support on the cut, i.e.\ $\eval{c_{\Gamma,\va{i}}}_{\rho_j \to 0} \neq 0$ } 



\todo{potentially mention here on how the transverse vectors are obtained}

Recall that some of the scalar products from $\vb*{\alpha}$ have a form $\sp(\ell_i,\tau_j)$ with $\tau_j$ transverse to all external momenta $p_i$.
We will denote these as $\vb*{\alpha}^\tau$.
From Lorentz symmetry an $L$-loop integral of an arbitrary rational function $f(\vb*{\vb*{x},\vb*{\rho}},D)$ is
\begin{equation} \label{eq:rank_1_tensor_integral}
  \Dmeasure{}~ f(\vb*{\vb*{x},\vb*{\rho}},D) ~ \ell_k^{\mu} = \sum_{i}^{}c_i(\vb*{x},D)\;p^\mu_i.
\end{equation}
Contracting both sides with $\tau^\mu_j$ and using the orthogonality condition we obtain
\begin{equation}
  \Dmeasure{}~ f(\vb*{\vb*{x},\vb*{\rho}},D) ~ \alpha^{\tau}_{k j}( \equiv \sp(\ell_k,\tau_j)) \quad = 0.
\end{equation}
Following the same argument it is easy to show that all monomials $\vb*{\alpha}^{\va{i}}$ with at least
one of the transverse ISPs taken to an odd power are vanishing upon integration and thus can be taken as surface terms as is.

Next we consider a rank 2 tensor integral
\begin{equation}
  \Dmeasure{}~ f(\vb*{\vb*{x},\vb*{\rho}},D) ~ \ell_k^{\mu}\ell_k^{\nu} = c_0\,g^{\mu\nu}_{[D]} + \sum_{i,j}^{}c_{i,j}\;p^\mu_i p^\nu_j,
\end{equation}
where the arguments of all coefficient are as in \cref{eq:rank_1_tensor_integral}, and we explicitly specified the dimensions of the metric tensor.
Upon contraction with $g^{\mu\nu}_{[D-4]}$ or $\tau_j^\mu\tau_j^\nu$, and taking into an account
\[
  g^{\mu\nu}_{[D]}g^{\phantom{\mu\nu}}_{[D-4]\nu\mu} = D-4,
\]
only $c_0$ survives.
We can then easily find a combination of two contractions which make the right-hand side vanish. Applying it to the left-hand side, we have
\begin{equation}
  \qty(\tau_j^\mu\tau_j^\nu - \frac{g^{\mu\nu}_{[D-4]}}{D-4}) ~ \ell_k^{\mu}\ell_k^{\nu} = (\alpha_j^\tau)^2 - \frac{\mu_{kk}(\vb*{\rho},\vb*{\alpha})}{D-4} \longrightarrow 0,
\end{equation}
where $\mu_{kj} \equiv g^{\phantom{\mu\nu}}_{[D-4]\mu\nu} \ell_k^\mu \ell_j^\nu$, so the expression on the right is a surface term and
we append it to our basis.

\todo{
  Here it is important that $\rho$ in $\mu_{ij}$ should be kept, although they are zero on the topology.
  This will be the same for surface terms.
}

Similar relations can be found for any product of even powers of $\alpha_j \in \vb*{\alpha}^\tau$.
They can all be generated with the projector
\begin{equation}
  \mathrm{P}^{\mu_1\ldots{}\mu_n}_j =  \tau^{\mu_i}_j\ldots{}\tau^{\mu_n}_j - \frac{1}{n}\frac{1}{D-4} \sum_{\sigma \in \mathcal{G}_{2;n}}^{} g^{\mu_{\sigma(1)}\mu_{\sigma(2)}}_{[D-4]}~\tau^{\mu_{\sigma(3)}}_j\ldots{}\tau^{\mu_{\sigma(n)}}_j,
\end{equation}
where the sum is over all inequivalent ways to distribute indices $\{\mu_1,\ldots{},\mu_n\}$ in the summand.
We will refer to such relations as \emph{traceless completions}.

As an example, we list all traceless completions sufficient for the computation of two-loop amplitudes in renormalizable theories:
\begin{align} \label{eq:traceless_completion_replacements}
  (\alpha_{kj}^{\tau})^2 &\to (\alpha_{kj}^{\tau})^2 - \frac{\mu_{kk}}{D-4}, \\
  \nonumber
  (\alpha_{kj}^{\tau})^2(\alpha_{k^\prime j}^{\tau})^2 &\to
  (\alpha_{kj}^{\tau})^2(\alpha_{k^\prime j}^{\tau})^2 - \frac{1}{D-4} \qty(
    \frac{1}{2}(\alpha_{kj}^{\tau})^2 \mu_{k^\prime k^\prime} ~+~ \alpha_{kj}^{\tau}\alpha_{k^\prime j}^{\tau} \mu_{k k^\prime} + \frac{1}{2}(\alpha_{k^\prime j}^{\tau})^2 \mu_{k k}
  ).
\end{align}

To summarize, all transverse ISPs are either trivially surface terms or can be turned into surface terms with the replacements of \cref{eq:traceless_completion_replacements}.
After this is done we are left to consider monomials from the ring $\mathcal{K}[\vb*{\alpha}\setminus\vb*{\alpha}^\tau]$ only.


We conclude this section with two remarks.
First, at one loop the master/surface decomposition is hereby complete since all ISPs are transverse.
Second, the traceless completions of transverse ISPs alternatively
can be obtained through a specially constructed IBP identities \cite{Ita:2015tya}.

%vectors of the form
%\begin{equation}
  %v^\mu = (\sp(n_\epsilon^i,\ell))~\ell^\mu - (\sp(\tau,\ell))~n_\epsilon^{i\,\mu}
%\end{equation}


\subsection{Unitarity-Compatible IBP Identities}
So far our set of numerator insertions $m_{\Gamma,i}$ consists of monomials with odd powers of transverse ISPs and traceless-completions of
monomials with even powers of transverse ISPs, which are all surface terms.
This leaves us to handle the span of polynomial ring $\mathcal{K}_{\vb*{x},D;\rho}[\hat{\vb*{\alpha}}\equiv \vb*{\alpha}\setminus\vb*{\alpha}^\tau]$ truncated by the imposed
power counting condition,
which is the most non-trivial part of the master/surface decomposition.

The reason why not all terms from $\mathcal{K}_{\vb*{x},D;\rho}[\hat{\vb*{\alpha}}]$ are master terms is that they are
related through IBP identities \eqref{eq:ibps}. So what we need to do is
\begin{enumerate}
  \item Generate IBP identities which relate
    \emph{only} monomials from $\mathcal{K}_{\vb*{x},D;\rho}[\hat{\vb*{\alpha}}]$, i.e.\ those which do not increase powers of denominators (and do not add new denominators).
    These identities can all be taken as surface terms, but they are not independent. 
    \label{item1}
  \item  Viewing each surface term as a vector in the linear space of monomials $\hat{\vb*{\alpha}}^{\va{i}}$, take
    a basis of this space as the set $\hat{S}_\Gamma$
    \label{item2}
  \item The vectors from the complement $\mathcal{K}[\hat{\vb*{\alpha}}]\setminus \hat{S}_\Gamma$ are then master terms $M_\Gamma$
    \label{item3}
\end{enumerate}

In principle the described procedure can be carried out with standard IBP reduction techniques discussed in \cref{sec:ibp}.
However this would be rather wasteful
since the vast majority of IBP identities involve relations between different topologies and doubling of denominators.
And indeed currently publicly available reduction programs
\cite{Studerus:2009ye,vonManteuffel:2012np, Smirnov:2008iw,Smirnov:2014hma, Lee:2012cn,Lee:2013mka, Maierhoefer:2017hyi,Maierhofer:2018gpa}
have problems with reducing high degrees of ISPs in many variables.

Instead we can generate from the start only the IBP identities described in \cref{item1},
known as \emph{unitarity-compatible} IBP identities \cite{Gluza:2010ws,Schabinger:2011dz,Ita:2015tya}, by
restricting the IBP vectors $v_i^\mu$ in \cref{eq:ibps} to satisfy
\begin{equation} \label{eq:non_doubling_ibp_vectors}
  \sum_i v_i^\mu(\vb*{x},D;\vb*{\rho},\vb*{\alpha}) ~ \pdv{\ell^\mu_i} \rho_k = f(\vb*{x},D;\vb*{\rho},\vb*{\alpha}) ~ \rho_k, \qquad \forall k\in P_\Gamma,
\end{equation}
where it is crucial that both the arbitrary function $f$ and the vector components $v_i^\mu$ are polynomial in $\vb*{\rho}$ and $\vb*{\alpha}$, not rational.

{
  \def\coeff{c^{(\text{loop, ext})}_{ij,\va{k}_\rho \va{k}_{\hat{\alpha}}}}
  A naive way to find all vectors satisfying \cref{eq:non_doubling_ibp_vectors}
  is by inserting a simple ansatz
  \begin{equation}
    v_i^\mu = \sum_{j}^{}c^{(\text{loop})}_{ij}\ell^\mu_j ~+~  \sum_{j}^{}c^{(\text{ext})}_{ij} p^\mu_j, \qquad
    c_{ij}^{(\text{loop,ext})} = \sum_{\va{k}_\rho, \va{k}_{\hat{\alpha}}} \coeff (\vb*{x},D)\, \vb*{\rho}^{\va{k}_\rho}\vb*{\hat{\alpha}}^{\va{k}_\alpha}
  \end{equation}
  into \cref{eq:non_doubling_ibp_vectors}.
  The linear constraints for coefficients $\coeff$ can be then solved with standard linear algebra.
}%
In practice much more powerful methods of solving \cref{eq:non_doubling_ibp_vectors}  based on computational
algebraic geometry can be employed \cite{Larsen:2015ped,Boehm:2018fpv,Zhang:2016kfo,Abreu:2017xsl,Bendle2019}.
Having unitarity-compatible IBP vectors at hand trivializes the generation of an overcomplete set of surface terms.
The spanning set of surface terms is then obtained as outlined in \cref{item2,item3}.

\todo{power counting remarks: 1. in general had to be increased before bringing it down to final}

\todo{note: $M_\Gamma$ can be and frequently is empty}

\todo{define clearly on-shell of topology so I can use it everywhere}

\todo{surface terms themselves need to be off-shell, can be checked with standard IBP reduction} 

\todo{linear independence can be checked on-shell}

\section{Cut Equations}
\label{sec:cut_equations}

I want to take the pragmatic perspective. And in practice it is not even essential:
\begin{itemize}
    \item       the most practical and flexible way to compute them in $D$ dimensions is with off-shell recursions (see references e.g.\ in tails of 1001 gluons)
    \item Actually it is not even always the case as we see for massive particles.
\end{itemize}

LHS can be just an expression obtained from Feynman diagrams

However in perturbation theory the fact that the corners are tree amplitudes can be seen explicitly.
Because we collect together all graphs containing a particular singularity.


We have constructed the master/surface basis in \cref{eq:master_surface_basis}. Now we discuss how
to reduce amplitudes to it.


\subsection{On-Shell Loop-Momenta}

\todo{In $D$ dimensions we don't have to worry about branches of solutions, and also loop momenta have to be real}

\subsection{Evaluation of Cuts}

\subsection{Off-Shell Recurcion}
\label{sec:BG_recursion}

$D$-dimensional off-shell recursion,


weyl and Dirac fermions,
tracking of representations?


\subsection{Dimensional Reconstruction}
 In $D$ and $D_s$ (dimensional reduction later)


\section{Subleading Poles}
Here discussion about \textit{subleading poles}


\section{Discussion}
The two parts of our method are largely independent of each other and can be used separately if deemed beneficial.

Can be used to fit any integrand parametrization. For example one
can want to not have $D$-dependent coefficient, and fit to simple monomials.
Or one can eliminate transverse ISPs only, and do the IBP reduction later.
Or one can choose master/surface basis and do the full reduction in place.

Since the hardest problem is of IBP reduction it is beneficial to choose master/surface basis.

\section{Examples}

\subsection{One-Loop Hierarchy}

Once the automated approach is set up, the one-loop case is so simple that we can
do the full one-loop hierarchy.

Compare to ``normal'' decomposition, we allow $D$-dependence in numerators. This might be not the best
idea for NLO computations since terms of order $\mathcal{O}(\epsilon)$ can be dropped.
It might be reasonable to do for higher-order computations where higher orders of $\epsilon$ are required.

Starting from two loops will have $D$-dependence in the numerator from IBP identities, so
we can choose to abandon all-together the idea of not having it there.


Take 5-point hierarchy.

\subsubsection{Pentagon}

Take $\mu^2$ insertion.

\subsubsection{Box}

\subsubsection{Triangle}

Here also show that 1 and 2 mass triangles can be reduced

\subsubsection{Bubble}

See what happens with the light-like external momentum from the IBP perspective?
Should work out.

\subsubsection{Tadpole}

\subsection{Some Two-Loop Topology}

\todo{select a topology which is not too big to present}

