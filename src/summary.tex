In this thesis, we have demonstrated the cutting-edge applications of the numerical unitarity method \cite{Abreu:2017hqn,Abreu:2017idw,Abreu:2017xsl,Ita:2015tya,Ellis:2008ir,Ellis:2007br,Giele:2008ve}
that advance the state-of-the-art computations of loop amplitudes.
Compared to more canonical approaches, the bottleneck of inversion of IBP systems is avoided in this method.
With the help of a targeted set of IBP relations \citep{Gluza:2010ws,Abreu:2017hqn}
a complete parametrization of the integrand in terms of master and surface insertions can be obtained.
The unitarity cuts are then matched to this parametrization, therefore achieving the reduction to master integrals.
The approach is formulated in a numerical context, thus mitigating increase of complexity with the addition of scales.


We have formulated a framework for efficient numerical evaluation of helicity amplitudes with external fermions in dimensional regularization.
This framework was an essential ingredient required to obtain our results for the two-loop 
two-quark three-gluon and four-quark one-gluon helicity amplitudes.
It is based on the dimensional reconstruction, but dimensionally
reduces the kinematically-independent degrees of freedom.
In this way, computations of two-loop amplitudes can be performed in six dimensions.
This substantially improves the efficiency of the extraction of the analytic dependence on the
number of dimensions $D_s$, compared to the numerical dimensional-reconstruction approach \cite{Giele:2008ve,Ellis:2008ir, Boughezal:2011br}, as applied
in \cite{Abreu:2017xsl,Abreu:2017hqn,Badger:2018gip,Abreu:2018jgq}.
This framework has been also useful to clarify some issues with the computation of one-loop helicity amplitudes
with massive external fermions \cite{Anger:2018ove}.

We have presented the first NLO QCD predictions for \Wbbjj{} and \Wbbjjj{} production in \cref{chap:wbb_pheno}.
The computation was done in the four-flavor-number scheme, thus including effects due to the non-vanishing bottom-quark mass.
To obtain the one-loop matrix elements with massive fermions for high-multiplicity processes,
we have carried out a major upgrade of the \BlackHat{} library, following the developments of \cite{Ellis:2008ir}.
We have looked into multiple differential distributions depending on the number of additional light jets.
We found that observables constructed to include $\geq2$ additional light jets at LO
show excellent convergence of NLO QCD corrections.
In contrast to this, NLO QCD corrections to inclusive $Wb\bar{b}$ production display large K factors and shape distortions.
Based on this observation, we employed exclusive sums to improve predictions for observables associated to $WH(\rightarrow b{\bar b})$
production, such as $p_T^{b\bar b}$, $p_T^W$, and $M_{b\bar b}$.


%Our approach is based . We take advantage of the ideas of dimensional reduction
%to precomute the full $D_s$ dependence by contracting all kinematically-independent indices.
%Thus the whole numerical computation  of two-loop amplitudes can be performed in six dimensions.
%This leads to
%significant efficiency improvements as compared to a numerical dimensional-reconstruction
%approach~\cite{Giele:2008ve,Ellis:2008ir, Boughezal:2011br} as applied earlier
%in refs.~\cite{Abreu:2017xsl,Abreu:2017hqn,Badger:2018gip,Abreu:2018jgq}.
%Enhancing the dimensional reconstruction by dimensional reduction 
%is not a new idea and has been applied to gluon amplitudes in
%the original work~\cite{Giele:2008ve}.
%Here we apply this improvement~\cite{Anger:2018ove}
%starting from the five-point amplitudes including external fermions~\cite{Abreu:2018jgq}.  


%The apply method for the computation of two-loop amplitudes with external fermions,
%an extension 


We have carries out extensive validation of our computational framework.

We have presented high presicion benchmark values for helicity amplitudes.

 we give compact
analytic expressions for five-gluon amplitudes as well as amplitudes with two and four
external quarks, including in all cases the contributions of closed light-quark loops.
These
results have been obtained employing a functional-reconstruction approach to promote
numerical unitarity  finite-field evaluations of the amplitude to analytical expressions.

With these methods, a sample with a size comparable to that required for
numerical Monte-Carlo integration
can be used to produce analytic expressions.

 Given our results, the complete set of two-
 loop amplitudes required for a NNLO QCD calculation of three-jet production at hadron
 colliders in the leading-color approximation are now available.
 Fast and efficient evaluation of coefficients.


 The techniques developed in this paper show the potential for the automation of two-
 loop multi-particle amplitude calculations in the Standard Model. Our numerical approach
 is relatively insensitive to the addition of scales. Having already implemented vector and
 spinor fields, we are now ready to explore processes of phenomenological relevance that
 include jets, (massive) gauge bosons and leptons in the final state. While techniques for
 computing two-loop master integrals progress and new methods appear for handling infrared
 divergent terms in real-real and real-virtual contributions, we expect to provide a program
 that can deliver one- and two-loop matrix elements necessary for computing precise QCD
 predictions for the LHC.



Outlook
\begin{itemize}
  \item potential extension to other processes
  \item analytic continuation
  \item in a framework for handling subtractions of IR divergences and real radiation contributions,
    and combined with a Monte-Carlo integration program.
  \item explore floating point computations
\end{itemize}


