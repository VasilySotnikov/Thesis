The processes of electroweak-gauge-boson production in association
with many light- and $b$-jets have diverse final states and are abundantly produced at the LHC. They therefore constitute a
natural testing ground for perturbative QCD and more generally the Standard Model
over several orders of magnitude. The study of these processes at the
LHC thus helps to improve the understanding of both theoretical and experimental
techniques, e.g.\ our understanding of heavy-jet production. Furthermore, these processes are background to many interesting signatures, for example $HW$ associated production with a subsequent decay of the Higgs into a
bottom-quark pair as well as many searches for new physics with
multi-particle final states. 

The theoretical description of these
high-multiplicity processes represents a
challenging task. In particular, the computation of the required QCD
$\mathcal{O}(\alps)$ corrections represents a difficult task due to the
many scales that are present such as the bottom mass. We used the numerical unitarity approach \cite{Bern:1994zx,Bern:1995db,Ossola:2006us,Ellis:2007br,Giele:2008ve,Ellis:2008ir} to compute virtual matrix elements, which required several extensions of the method, and which allowed us to make a first computation for massive final states in association with many jets and multiple quark lines. In general, NLO QCD corrections help to considerably reduce the spurious dependence on unphysical renormalization and factorization scales. Associated to it is the question which scale to use in a fixed-order calculation. New ideas for scale setting such as the parton-shower-inspired \MINLO{} method \cite{MINLO} are an interesting alternative to traditional approaches to scale setting. The application and comparison of different functional forms of dynamical scales for high-multiplicity processes helps to improve the understanding of uncertainties associated to NLO QCD calculations. 


In this thesis, we presented NLO QCD predictions for \Wbb~production
in association with two and three light jets \cite{wbbpaper}, and we also included results with zero or
one extra light jet for completeness. We find that the NLO QCD
corrections for the high-multiplicity processes of \Wbbnj[2]{}
and \Wbbnj[3]{} production are mild and the remaining renormalization and factorization scale 
dependence reduces from LO to NLO. In contrast, the $K$-factor for NLO QCD corrections is large in inclusive
\Wbb{}
production~\cite{Ellis:1998fv,FebresCordero:2006sj,Cordero:2009kv},
which led us to explore possible
improvements to observables for the \Wbb{} process by using exclusive
sums~\cite{ESums}. We focused on the context of this process being background to $H(\rightarrow b\bar b)W$ associated production and showed that exclusive-sum predictions for key observables give reduced uncertainties
associated to missing higher-order corrections. We
found that uncertainties associated to PDFs are subleading in the generic kinematical regimes we studied. We provide \ntuple{}
sets for the predictions obtained by a new \BlackHat{} version in
combination with \SHERPA{} \cite{Sherpa}, which can be used in future analyses of the 
$Wb\bar b\,\!+\!\,n$-jet
signatures.


To achieve these state-of-the-art NLO QCD results for \Wbbn~production, we needed to extend
previous applications of the numerical unitarity method to the case of
multiple massive fermion lines. For that we presented all relevant
ingredients that we used and developed. In particular, the previous application of the numerical unitarity method with massive quarks by Melnikov et al \cite{Ellis:2008ir} did not reach this level of complexity and we provide a complete set of methods for the first time. In order to deal with explicitly
divergent double cuts, we used the prescription of
Ref.~\cite{Ellis:2008ir}. In the application of single cuts we also encountered divergent contributions associated to tadpole Feynman diagrams. We extended the
prescription of Ref.~\cite{Ellis:2008ir} to this case and removed the divergent expressions by adjusting the Berends-Giele tree
generation. 


Furthermore, we efficiently evaluated $D$-dimensional unitarity cuts
by reducing the $D_s$-dimensional Dirac algebra and states to lower
dimensions \cite{angerds}. In order to do so, we found a correct numerical implementation of the HV/FDH scheme for massive external fermions, which builds on a consistent embedding of the external fermion states. The resulting decomposition of one-loop helicity amplitudes by particle content allowed us to avoid the overhead of computing in
higher-dimensional representations of the Dirac algebra. We have implemented the above in a new version of the \BlackHat{} library.


Moreover, we have performed a dedicated study of weak-vector-boson production in
association with light jets at the $\sqrt{s} = 13$ TeV
LHC~\cite{Anger:2017nkq} that we presented in this thesis. We have provided NLO QCD predictions for
$W^\pm+n$-jet and $Z+m$-jet production in association with $n\leq 5$
and $m\leq 4$ light jets. We used available matrix elements from \BlackHat{} \cite{Berger:2008sj} in combination with \SHERPA{} and extended previous predictions~\cite{BH:W5j,BH:Z4j} to the energy configuration of LHC Run-II. We provided a study of uncertainties associated to renormalization
and factorization scales by employing several functional forms of
dynamical scales and by conventional variations of
the central scale. To this end, we used the fixed-order scale $\HTpartonicp/2$ and compared the results with the more
physically motivated \MINLOp{} reweighting procedure (adaption of the original formulation in \cite{MINLO}). This is the first
time such a comparison is carried out in the context of processes with
the high jet multiplicity of four and five jets. We find that NLO results obtained with $\HTpartonicp/2$
and \MINLOp{} are largely consistent in both shapes and normalization for total cross
sections and differential distributions. We also computed several observable ratios and found that jet production ratios have an increased stability compared to the
results at $\sqrt{s} = 7$ TeV. In general,
the good agreement between the two scale choices confirms that NLO
QCD predictions in high multiplicity processes give the first reliable
predictions. It will be interesting to see more comparisons of our results with LHC data such as the ones recently shown in Ref.~\cite{Aaboud:2017hbk}.



Finally, we discuss possible extensions of the general approach of numerical unitarity with massive quarks presented in this thesis. We have matrix elements for several interesting phenomenological processes available in our new version of the
\BlackHat{} library. They allow for the computation of NLO QCD corrections to challenging processes such as on-shell
$t\bar{t}+b\bar{b}+$jet production, on-shell $t\bar{t}$
production in association with up to four light jets, inclusive $Wb$+jets production and other applications with massive quarks
and vector bosons in the future. Also $Zb\bar{b}$ production in association with two
light jets should be accessible with some minor modifications of our new
matrix elements. In this thesis, we presented a comparison of 4FNS and 5FNS for \Wbb~and \Wbbnj[1]~production. This comparison could easily be extended to the
higher-multiplicity cases of \Wbbnj[2]~and \Wbbnj[3]~production. In
particular, $b$-initiated subprocesses start to play a more
dominant role for these high-multiplicity cases. It would be interesting to see how this affects the comparison of 4FNS and 5FNS and whether we can confirm our conclusions for the comparison in the higher-multiplicity cases. 

On the conceptual side, the correct numerical implementation of the HV/FDH scheme for massive quarks and our resulting algorithm for a decomposition of QCD
one-loop amplitudes by particle content \cite{angerds} can be beneficial for the computational
efficiency of one- and two-loop amplitudes in the future. In particular, an
application in the numerical unitarity approach at two-loops seems to
be feasible and well in line with the ongoing research activities here
in Freiburg. This might open the route to two-loop amplitudes with massive external quarks. More generally, in order to verify both theory predictions and experimental
techniques at the LHC, it will be interesting to
study and compare data with predictions for high-multiplicity processes
including $b$ jets, like the ones presented in this work. For example, the results based on exclusive sums for \Wbb{} production presented in this thesis might help to further constrain the coupling of the Higgs to bottom quarks.


