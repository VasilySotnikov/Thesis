In the following two chapters, we present a variant of the
$D$-dimensional numerical unitarity method~\cite{Bern:1997sc,Britto:2004nc,Ellis:2007br,Giele:2008ve,Ellis:2008ir} applied to QCD \ola s with massive quark pairs. Unitarity methods are based on both
analyticity and unitarity of one-loop scattering amplitudes. Analyticity allows to reconstruct amplitudes as a function of kinematical invariants and mass parameters while unitarity guarantees that residues at the
singular points factorize into tree amplitudes. The term numerical here refers to the variant of the unitarity approach suitable for numerical evaluation as opposed to analytic computations. Numerical unitarity
with massless particles has been successfully applied at one loop to high-multiplicity processes \cite{BH:W5j,BH:Z4j,FebresCordero2017} and recently using
the techniques described in \cite{Ita2016} also to the four- and five gluon
amplitude at two loops~\cite{Abreu:2017xsl,Abreu:2017idw}. Nevertheless, implementations of
unitarity with massive external and internal particles remain few. Analytic formulae
have been obtained for \Wbb~\cite{Badger:2010mg} and $t\bar{t}$ production
\cite{Badger:2011yu} and a numerical approach was applied to on-shell $t\bar{t}(+1$
jet) production \cite{Melnikov:2009dn,Melnikov:2010iu}. The work
presented in this thesis applies the method of numerical unitarity to
high-multiplicity processes with massive quark flavors. In Chapter
\ref{chap:wbb_results}, we will for the
first time present phenomenological results
for $Wb\bar{b}+2$ and $3$ light jets using the here described
methods. This is possible through a combination of the methods of numerical unitarity
\cite{BlackHatI,Ellis:2007br,Ellis:2008ir}, together with an important extension of the algorithm, which concerns the careful treatment of internal and external polarization states in dimensional regularization. All details presented here
have been implemented in C++ libraries which are the basis for a new
version of the \BlackHat{} library~\cite{BlackHatI}. We overview the aspects of numerical
unitarity most relevant for this thesis, and more details can be found
in the reviews \cite{Ita:2011hi,Ellis:2011cr} and references therein. The next Chapter
\ref{chap:fdf}, is devoted to the way we compute unitarity cuts in
$D$-dimensions with massive particles.


\section{Preliminaries}
\label{sec:unitarity}
\subsection{A Generic One-Loop Amplitude in $D$-Dimensions}
\label{sec:genloop}
We need to regularize \ola s to keep track of divergencies. Following
the approach of dimensional regularization, we move away from $D=4$
dimensional space-time and continue momenta and polarization vectors
of loop particles to $D\neq 4$-dimensional space time. Without
specifying \textit{how} to choose the space-time dimensionality for the moment, we can write any generic $n$-particle one-loop
amplitude in $D$-dimensional space-time as
\begin{align}\label{eq:genericola}
\Am_n \sim \int \frac{\dd[D]{\ell}}{(2\pi)^{D}} \frac{\mathcal{N}(p_1,p_2,\dots,p_N;\ell)}{D_1D_2\dots D_N},
\end{align}
with the numerator function $\mathcal{N}$ depending on the
momenta of the external particles as well as on the loop momenta $\ell$. The inverse propagators 
\begin{align}\label{eq:invprop}
  D_i=\left(\ell_i^2-m_i^2\right)=\left((\ell+q_i)^2-m_i^2\right)=\left((\ell+\sum_{j=1}^ip_j-q_0)^2-m_i^2\right),
\end{align}
build up the denominator, see Fig.~\ref{fig:genericloop}, with region
momenta $q_i$ defined up to an arbitrary shift $q_0$. All external
momenta are defined to be outgoing throughout this thesis.

\begin{equation*}
\begin{tikzpicture}[very thick]
\coordinate (O) at (0,0);
\def\ri{1.8cm}
\def\ris{2.0cm}
\def\ro{3.5cm}
\def\ar{2.5cm}
\def\lr{3.3cm}

\def\bd{292}
\def\ed{76}
%\draw[decoration={text along path,reverse path,text
%align={align=center},text={{$l_1$}}},decorate] (2.5,0) arc
%(0:180:2.5);
 \draw [dotted,very thick,domain=\bd:360] plot ({\ris*cos(\x)}, {\ris*sin(\x)});
 \draw [dotted,very thick,domain=0:\ed] plot ({\ris*cos(\x)}, {\ris*sin(\x)});
 \draw [very thick,domain=\ed:\bd] plot ({\ris*cos(\x)}, {\ris*sin(\x)});
%\draw (O) circle (\ri);
\node (i1) at (220:\ri){};
\node (i2) at (292:\ri){};
\node (i3) at (4:\ri){};
\node (i4) at (76:\ri){};
\node (i5) at (148:\ri){};
\node (o1) at (220:\ro) {$p_1$};
\node (o2) at (292:\ro){$p_N$};
\node (o3) at (4:\ro){};
\node (o4) at (76:\ro){$p_3$};
\node (o5) at (148:\ro){$p_2$};
\draw [->,very thick] (128:\ar) arc[x radius=\ar, y radius =\ar, start
angle=128, end angle=96];
\node (t2) at (112:\lr){$\ell+q_2$};
\draw [->,very thick] (200:\ar) arc[x radius=\ar, y radius =\ar, start
angle=200, end angle=168];
\node (t1) at (180:\lr){$\ell+q_1$};
\draw [->,very thick] (272:\ar) arc[x radius=\ar, y radius =\ar, start
angle=272, end angle=240];
\node (t3) at (252:\lr){$\ell+q_N$};
\begin{scope}[very thick,decoration={
    markings,
    mark=at position 0.6 with {\arrow{>}}}
    ] 
\draw[postaction={decorate}] (i1) -- (o1);
\draw[postaction={decorate}] (i2) -- (o2);
\draw[postaction={decorate}] (i4) -- (o4);
\draw[postaction={decorate}] (i5) -- (o5);
\end{scope}
\end{tikzpicture}
\end{equation*}
\vspace{-1.2cm}
\captionedequationset{A generic one-loop amplitude, with region
  momenta $q_i$ defined up to a shift $q_0$. The momenta of all
  external particles are defined to be
  outgoing.\label{fig:genericloop}} 


\subsection{Dimensional Regularization}
\label{sec:dimregfdh}
In dimensional regularization, we set
\begin{align}
  D\equiv 4-2\epsilon,
\end{align}
which results in a representation of the UV and IR divergent integrals
in terms of poles in $1/\epsilon^n$. The final results we
provide are regularized in the standard 't
Hooft-Veltman scheme (HV) \cite{tHooft:1972tcz}\footnote{Here $D_s=D$. For more on dimensional regularization schemes, see Sec.~\ref{sec:FDH-def}.}. For intermediate calculations however, we employ the four-dimensional
helicity scheme (FDH)\footnote{Here, we set $D_s=4$ after performing
  the algebra of Dirac matrices.}, in which results are related to the HV scheme by a finite shift
that we apply at the end of our calculation
(cf.~Sec.~\ref{sec:schemeshift}). For the purpose of this chapter, it
suffices to define $D$-dimensional loop momenta $\ell$ and a
decomposition into a
$4$-dimensional part and an orthogonal $(D-4)$-dimensional part
\cite{Bern:1995db}, which we refer to as higher-dimensional part,
defined by
\begin{align}\label{eq:lsplit}
  \ell^\mu_{[D]}=\ell_{[4]}^\mu + \ell_{[D-4]}^\mu,
\end{align}
with $\ell_{[4]}\cdot \ell_{[D-4]} = 0$. The corresponding on-shell conditions read
\begin{align}\label{eq:ddimos}
\begin{split}
  \ell^2_{[D]}&=\ell_{[4]}^2+\ell^2_{[D-4]}=m^2\\
&=\ell_{[4]}^2-\mu^2=m^2,
\end{split}
\end{align}
with the invariant mass of the higher-dimensional part
$\ell_{[D-4]}^\mu$ conventionally
named \linebreak {$\ell_{[D-4]}^2= -\mu^2$}. In FDH, external momenta are kept in $D=4$ and contractions of those with
the higher-dimensional part of the loop momentum are zero
$\ell_{[D-4]}\cdot p_{i\,[4]} = 0$, i.e.\ there is no preferred direction in the
$(D-4)$ subspace of the $D$-dimensional momenta. Thus the only dependence of the numerator
function $\mathcal{N}$ in Eq.~\eqref{eq:genericola} on the higher
dimensional part of the loop momentum comes through contractions of
the form
\begin{align}
  \ell^2_{[D-4]}=-\mu^2.
\end{align}
In particular, this
implies a rotational invariance in the higher-dimensional subspace of
the loop momenta in the integration since only the absolute square of it enters the
numerator function\footnote{If one chooses a formulation in integer
  dimensions it thus suffices to use a $5$-dimensional loop
  momentum}. For numerical purposes, we require finite dimensional representations of both metric tensor and Dirac
algebra in order to compute \ola s in FDH. This is the topic of Chapter \ref{chap:fdf}.



\subsection{Exploiting Unitarity as a Computational Method}
\label{sec:unimeth}
The unitarity approach builds on the observation that integrands of one-loop amplitudes $A_n^{\text{1-loop}}(\ell)$ factorize into products of tree amplitudes in the
factorization or on-shell limit:
\begin{align}\label{eq:faclimit}
\begin{split}
 \text{Res}_{i_1\dots i_k}[A_n^{\text{1-loop}}(\ell)]&\equiv\lim_{\ell\rightarrow
    \ell_{i_1\dots i_k}}\left(D_{i_1}\dots
    D_{i_k}A_n^{\text{1-loop}}(\ell)\right)\\
&=\sum_{\text{states}}\left(\Amt_{n_1}(\ell_k,\dots,-\ell_1) \times \dots
    \times \Amt_{n_k}(\ell_{k-1},\dots,-\ell_k) \right),
\end{split}
\end{align}
where the set of propagator momenta $\ell_{i_1\dots
  i_k}=\{\ell_{i_1},\dots,\ell_{i_k}\}$ in the loop simultaneously solves the
on-shell conditions for the inverse propagators
\begin{align}\label{eq:simloop}
D_{i_1}(\ell_{i_1}) = \dots
= D_{i_k} (\ell_{i_k}) = 0.
\end{align}
The propagator momenta are thereby restricted to the on-shell phase spaces
and the sum runs over the states associated to the full spectrum of the
theory for each on-shell momentum. The systematic application of
these unitarity cuts allows a computation of the coefficients of the
expansion of \ola s in terms of master integrals, as we will
see in the following sections. The above is connected to the
unitarity of the $S$-matrix, as worked out in \cite{Eden:1966dnq}.

\subsection{Master Integrals}
\label{sec:scmi}
Any \ola~in $D$-dimensional space time can be written as a sum of scalar master integrals \cite{Melrose1965,Passarino:1978jh,Neerven1984a,Bern1993}
\begin{align}\label{eq:mi}
\Am_n = \sum_i \tilde{e}_i I_5^i+\sum_i \tilde{d}_i  I_4^i+ \sum_i \tilde{c}_i I_3^i+ \sum_i \tilde{b}_i I_2^i+ \sum_i \tilde{a}_i  I_1^i,
\end{align}
where the summation runs over all scalar integral functions of the
corresponding topology, and $I_5,I_4,I_3,I_2,I_1$ - pentagon, box,
triangle, bubble and tadpole integrals respectively - are known. The pentagon functions $I_5$ can be expressed in terms of lower point functions $I_{n<5}$ in one-loop computations, however, we require pentagon integrands and keep pentagons in intermediate steps of the computation. The master integrals on the right-hand side of Eq.~\eqref{eq:mi} are defined as 
\begin{align}
I_n^i = \frac{\mu^{2\epsilon}}{i c_\Gamma} \int \frac{\dd[D]{\ell}}{(2\pi)^{D}} \frac{1}{D_{i_1}D_{i_2}\dots D_{i_n}},
\end{align}
with $c_\Gamma=(4\pi)^{-(2-\epsilon)}\Gamma^2(1-\epsilon)\Gamma(1+\epsilon)/\Gamma(1-2\epsilon)$. In the following, we suppress the prefactor appearing in the integrals for simplicity. A set
of the IR and / or collinearly divergent scalar integrals for general masses can for example be found in \cite{Ellis:2007qk}. Compared to a computation
with massless external particles, the basis we employ contains more integrals. These are integrals with massive
internal propagators as well as a new topology, the tadpole integral
$I_1$. For massless particles, the latter is scaleless and vanishes in dimensional
regularization. 

The extraction of the coefficients for the tadpole
integral and the bubble integral with
a single massive leg in one corner $I_{2,m^2}$, see Fig.~\ref{fig:newmi}, require a special
treatment in the unitarity method. This is the topic of
Section~\ref{sec:massivebubble}. Explicit formulae for these two integrals can
be found in App.~\ref{sec:massivebasisint}. 

\begin{eqnarray*}
\begin{tikzpicture}[baseline=(m1)]
  \def\leglength{0.75}
  \def\blobpos{1.0}
  \def\secondb{\blobpos+2}
 
  \node at (\blobpos+1,0) {$I_{2,m^2}$};
  \node at (\blobpos+1,1.2) {$0$};
  \node at (\blobpos+1,-1.2) {$m^2$};
  \node at (\blobpos-1.35,0) {$m^2$};
  \begin{feynman}[inline=(m1)]
    \vertex (m1) at (\blobpos, 0);
    \vertex(m2) at (\secondb, 0);
    \vertex (a) at (\secondb+\leglength*1.4,0){\vdots};
    \vertex (b) at (\secondb+ \leglength,\leglength);
    \vertex (c) at (\secondb+ \leglength,- \leglength);
    \diagram* {
      i -- [ultra thick] (m1)
       -- [half left] (m2)
       -- [ultra thick, half left] (m1),
       (m2) --  (b),
      (m2) -- [ultra thick] (c),
    };
  \end{feynman}
\end{tikzpicture}
& &\hspace{1.5cm}
\begin{tikzpicture}[baseline=(m1)]
  \def\leglength{0.75}
  \def\blobpos{1.0}
  \def\secondb{\blobpos+2}

  \node at (\blobpos+1,1.2) {$m^2$};
  \node at (\blobpos+1,0) {$I_{1}$};
  \begin{feynman}[inline=(m1)]
    \vertex (m1) at (\blobpos, 0);
    \vertex(m2) at (\secondb, 0);
    \vertex (a) at (\secondb+\leglength*1.4,0);
    \diagram* {
      (m1) -- [ultra thick,half left] (m2)
       -- [ultra thick, half left] (m1),
       (m2) --  (a),
    };
  \end{feynman}
\end{tikzpicture}
\end{eqnarray*}
\vspace{-1.2cm}
\captionedequationset{The extraction of the coefficients for the additional scalar integrals $I_{2,m^2}$ and
  $I_1$ require a special treatment in the unitarity method. Thick lines refer to loop momenta with
  $\ell^2=m^2$.\label{fig:newmi}}

The number of inflow momenta for each integral in the basis of scalar master integrals is
bounded from above by the space-time dimensionality $D$. Since we take
external momenta and polarization vectors to be four dimensional, we maximally
include a five-point function, the pentagon. Any higher-point function
can be reduced to this set of integral functions. The knowledge of the
coefficients $\tilde{e}_i,\tilde{d}_i,\tilde{c}_i,\tilde{b}_i,\tilde{a}_i$ completely defines the \ola.


\section{The Loop Integrand}
We consider loop amplitudes before integration,
that is the integrand of \eqref{eq:genericola}. In the construction due
to OPP~\cite{Ossola:2006us} and Ellis, Giele, Kunszt and Melnikov (EGKM) \cite{Ellis:2007br,Giele:2008ve}, the
generic expression for the integrand of a $D$-dimensional \ola~is
\begin{align}
\begin{split}
 A_n^{\text{1-loop}}(\ell)\equiv\frac{\mathcal{N}(\ell)}{D_1\dots D_n} =&\sum_{1\le i_1 < \dots < i_5 \le
   n}\frac{\Delta^5_{i_1i_2i_3i_4i_5}(\ell)}{D_{i_1}D_{i_2}D_{i_3}D_{i_4}D_{i_5}}+\sum_{1\le
   i_1 < \dots < i_4 \le
   n}\frac{\Delta^4_{i_1i_2i_3i_4}(\ell)}{D_{i_1}D_{i_2}D_{i_3}D_{i_4}}\\
&+\sum_{1\le i_1 < \dots < i_3 \le n}\frac{\Delta^3_{i_1i_2i_3}(\ell)}{D_{i_1}D_{i_2}D_{i_3}}+\sum_{1\le i_1 < \dots < i_2 \le n}\frac{\Delta^2_{i_1i_2}(\ell)}{D_{i_1}D_{i_2}}+\sum_{1\le i_1 \le n}\frac{\Delta^1_{i_1}(\ell)}{D_{i_1}},
\end{split}
\end{align}
where inverse propagators are defined in Eq.~\eqref{eq:invprop}. We
suppress the dependence of the numerator tensors $\Delta^i(\ell)$ on
external particles and polarization states. As a next step, we want to
find a suitable
parameterization of the numerator tensors $\Delta^i(\ell)$. It should
satisfy the following criteria: all possible tensor
insertions up to a certain rank should be covered and it should be easy to relate
the basis to the coefficients $\tilde{e}_i,\tilde{d}_i,\tilde{c}_i,\tilde{b}_i,\tilde{a}_i$ of the scalar master integral
decomposition (cf.~Eq.~\ref{eq:mi}). Such a basis of possible
tensors can be found using the van Neerven - Vermaseren construction (NV)
\cite{Neerven1984a}, for more detail see also App.~\ref{sec:vnvm}. In
the NV construction, the $D$-dimensional
space-time is decomposed into a physical space, spanned by the linearly
independent external momenta flowing into the parent diagram, and its complement, the
transverse space. The vectors $n_i$ form an orthonormal
basis that spans the space transverse to the
physical momenta, $p_i\cdot n_j = 0$ and $n_i
\cdot n_j = \delta_{ij}$. Together with the external momenta
$p_i$ the $n_i$ form a basis of the four dimensional
momentum space.


A suitable numerator parameterization for $D$-dimensional \ola s involving massive particles is given by \cite{Ellis:2007br,Ellis:2008ir}
\begin{align}\label{eq:intparam}
\begin{split}
  \Delta_{i_1i_2i_3i_4i_5}^5(\ell) &= e^0_{i_1i_2i_3i_4i_5}\mu^2\\
  \Delta_{i_1i_2i_3i_4}^4(\ell) &=
  d^0_{i_1i_2i_3i_4}+d^1_{i_1i_2i_3i_4}t_1\\
&+(d^2_{i_1i_2i_3i_4}+d^3_{i_1i_2i_3i_4}t_1)\mu^2+d^4_{i_1i_2i_3i_4}\mu^4\\
  \Delta_{i_1i_2i_3}^3(\ell) &=
  c^0_{i_1i_2i_3}+c^1_{i_1i_2i_3}t_1+c^2_{i_1i_2i_3}t_2+c^3_{i_1i_2i_3}(t_1^2-t^2_2)\\
&+t_1t_2(c^4_{i_1i_2i_3}
  +c^5_{i_1i_2i_3}t_1+c^6_{i_1i_2i_3}t_2)\\
&+(c^7_{i_1i_2i_3}t_1+c^8_{i_1i_2i_3}t_2+c^9_{i_1i_2i_3})\mu^2\\
  \Delta_{i_1i_2}^2(\ell) &=
  b^0_{i_1i_2}+b^1_{i_1i_2}t_1+b^2_{i_1i_2}t_2+b^3_{i_1i_2}t_3+b^4_{i_1i_2}(t_1^2-t_3^2)\\
&
+b^5_{i_1i_2}(t_2^2-t_3^2)+b^6_{i_1i_2}t_2t_3+b^7_{i_1i_2}t_1t_2+b^8_{i_1i_2}t_1t_3\\
&+b^9_{i_1i_2}\mu^2\\
  \Delta_{i_1}^1(\ell) &=
  a^0_{i_1}+a^1_{i_1}t_1+a^2_{i_1}t_2+a^3_{i_1}t_3+a^4_{i_1}t_4,
\end{split}
\end{align}
where $t_i= (n_i\cdot \ell)$. The terms $\mu^2$ are the extra dimensional
scalar product of the loop momentum and originate in a split-up of the
loop momentum into four and $(D-4)$-dimensional part
(cf.~Eqs.~\eqref{eq:lsplit} and \eqref{eq:ddimos}).


The terms in Eq.~\eqref{eq:intparam} that contain
projections of the loop momentum into the transverse space integrate
to zero (spurious terms). Lorentz-invariance\footnote{And parity invariance to
  exclude the epsilon tensor.} restricts the possible results of the
integration to tensors formed by external momenta $p_i^\mu$ and metric tensors
$g^{\mu\nu}$. The contraction of the resulting tensor with any
symmetric tensor of transverse basis vectors therefore yields a vanishing result. As an explicit example, we can integrate the
spurious part of the box numerator $\Delta^4_{1234}(\ell)$ over a $4$-dimensional loop momentum $\ell^\mu$ ($\mu^2=0$) and get
\begin{align}
\begin{split}
  \int\frac{\dd[4]{\ell}}{(2\pi)^{4}}\frac{t_1}{D_1D_2D_3D_4}=\int\frac{\dd[4]{\ell}}{(2\pi)^{4}}\frac{\ell\cdot
    n_1}{D_1D_2D_3D_4}\sim T(p_i)\cdot n_{1}=0,
\end{split}
\end{align}
with the tensor $T^\mu(p_i)$ being formed by linear combinations of external
momenta. Since the transverse vectors are orthogonal to the physical
space, the scalar product of $T^\mu(p_i)$ with the transverse
vector $n_1^\mu$ must vanish. A similar reasoning holds for general
tensors, see e.g.\ \cite{Ita:2011hi}. The remaining terms of the basis
in Eq.~\eqref{eq:intparam} either plainly multiply scalar master
integrals or scalar integrals with $\mu^n$ insertions, which we will treat later. The above basis is therefore naturally
connected to the basis of scalar master integrals Eq.~\eqref{eq:mi}. 

 

It remains to justify, why the above parameterization in
Eq.~\eqref{eq:intparam} covers all possible tensor insertions. The
tensors with components pointing along the physical space are
dispensable. They lead to integrands of lower-point and scalar
integrands as can be seen from the simplest case of 
\begin{align}
  \ell\cdot q_2 = \frac{1}{2}\left(\ell^2-(\ell-q_2)^2+q_2^2\right)=\frac{1}{2}\left(D_1-D_2+q_2^2\right).
\end{align}
Inserting this numerator into the integrand, the inverse propagators
cancel and we get
\begin{align}\label{eq:physdir}
  \frac{\ell\cdot q_2}{D_1D_2D_3\cdots D_n} = \frac{1}{2}\left(\frac{1}{D_2D_3\cdots D_n}-\frac{1}{D_1D_3\cdots D_n}+\frac{q_2^2}{D_1D_2D_3\cdots D_n}\right),
\end{align}
with lower-point integrands and a scalar integrand. The
Passarino-Veltman reduction algorithm \cite{Passarino:1978jh} exploits
the above, namely that at one-loop, every scalar product of loop and
external momenta can be expressed as a combination of propagators. 

In consequence, the numerator tensors can only have components pointing along the
transverse direction. Out of those, only a subset is linearly
independent. As an example, the box numerator tensor could contain terms of
the form 
\begin{align}
   \Delta_{i_1i_2i_3i_4}^4(\ell) = \cdots + \ell^\mu\ell^\nu n_{1\mu}n_{1\nu}+\ell^\mu\ell^\nu\ell^\rho
n_{1\mu}n_{1\nu}n_{1\rho}+\cdots,
\end{align}
where the maximal rank of the tensor in Standard Model processes is equal to the
number of inverse propagators in the diagram. Each of the above terms can be rewritten
using the projection of the metric tensor into the transverse space
(cf.~App.~\ref{sec:vnvm}~, Eq.~\eqref{eq:metricpron}). The terms then
have the following form
\begin{align}
    \ell^\mu\ell^\nu n_{1\mu}n_{1\nu} = \ell^2 - \sum_{i=1}^3(\ell
    \cdot p_i)(\ell \cdot v_i),
\end{align}
where $v_i$ are the so called dual vectors (see
Appendix~\ref{sec:vnvm}) and the above terms contain inverse propagators and projections of the loop momentum onto the physical space $\ell
\cdot p_i$. The latter again reduce, as in
Eq.~\eqref{eq:physdir}, to lower-point and scalar
integrands. With the projection of the metric into the transverse
space, one obtains an additional constraint on the numerator tensors
for each topology. The number of tensors in the numerator of
each topology is then determined by power counting and the
additional constraint from the projection of the trace into the
transverse space.

With the parameterization of Eq.~\eqref{eq:intparam}, the integration
over the loop momentum, and thereby the connection to the basis of
master integrals, can be trivially performed. As an example, we do get for the
integration over the box numerator in $D$-dimensions
\begin{align}
\begin{split}
  \int\frac{\dd[D]{\ell}}{(2\pi)^{D}}\frac{\Delta^4_{ijkl}(\ell)}{D_iD_jD_kD_l}&= \int\frac{\dd[D]{\ell}}{(2\pi)^{D}}\frac{d^0_{ijkl}+d^1_{ijkl}t_1+(d^2_{ijkl}+d^3_{ijkl}t_1)\mu^2+d^4_{ijkl}\mu^4
}{D_iD_jD_kD_l}\\
&= \int\frac{\dd[D]{\ell}}{(2\pi)^{D}}\frac{d^0_{ijkl}+d^2_{ijkl}\mu^2+d^4_{ijkl}\mu^4
}{D_iD_jD_kD_l}\\
&= d^0_{ijkl}I_{4,ijkl}+d^2_{ijkl}I_{4,ijkl}[\mu^2]+d^4_{ijkl}I_{4,ijkl}[\mu^4],
\end{split}
\end{align}
where $I[\mu^2]$ denotes an insertions of $\mu^2$ from the higher-dimensional part of the loop
momentum. 


By splitting the integration into the $4$ and $D-4$-dimensional part,
one can rewrite scalar master integrals with $\mu^2$ insertions in
terms of higher dimensional scalar integrals \cite{Giele:2008ve,Badger:2008cm} (see also appendix of \cite{Bern:1995db}) 
\begin{align}\label{eq:intdim}
\begin{split}
  I_{i\dots j}^{(D)}[\mu^2]=\int\frac{\dd[D]{\ell}}{(2\pi)^{D}}\frac{\mu^2}{D_i\dots
    D_j} &= -\frac{(D-4)}{2} I_{i\dots j}^{(D+2)}[1]\\
  I_{i\dots j}^{(D)}[\mu^4]=\int\frac{\dd[D]{\ell}}{(2\pi)^{D}}\frac{\mu^4}{D_i\dots
    D_j}&= \frac{(D-4)(D-2)}{4} I_{i\dots j}^{(D+4)}[1].
\end{split}
\end{align}

With the identities in Eq.~\eqref{eq:intdim}, different choices for
the tensor basis are possible. We use the specific form of the pentagon numerator
$\Delta^5(\ell)$ given in \cite{Badger:2008cm}. It is
obtained by a dimensional shift identity on the $D$-dimensional scalar
pentagon. As a result, one gets a $\mu^2$ insertion on the pentagon
numerator and a contribution to the scalar box coefficient, which we
already absorbed into the coefficient $d^0$. In total, we thus have five non-vanishing integrals with $\mu^2$ insertions. They are
multiplied by the coefficients $e^0,d^2,d^4,c^9,b^9$. The six-dimensional
scalar box and pentagon integrals are finite \cite{Bern1993}. The
integrals therefore do not contribute to the amplitude in the limit
$D=4-2\epsilon \rightarrow 4$
\begin{align}\label{eq:vanint}
\begin{split}
\lim_{D\rightarrow 4}\frac{(D-4)}{2} I^{(6)}_{4,ijkl}[1] = 0\\
\lim_{D\rightarrow 4}\frac{(D-4)}{2} I^{(6)}_{5,ijklm}[1] = 0.
\end{split}
\end{align}
The 6-dimensional 2-point and 3-point as well as the 8-dimensional
4-point scalar integrals are UV divergent. Each of those integrals is
multiplied by a factor $(D-4)$, cf.~Eq.~\eqref{eq:intdim}, that
cancels the pole $1/\epsilon$. Therefore, these integrals produce finite,
$\epsilon$-independent contributions that are given by
\begin{align}\label{eq:finiterint}
\begin{split}
  \lim_{D\rightarrow 4} I_{4,ijkl}^{(D)}[\mu^4]&= \lim_{D\rightarrow 4}\frac{(D-4)(D-2)}{4}I_{4,ijkl}^{(D+4)}[1] \rightarrow -\frac{1}{6}\\
   \lim_{D\rightarrow 4} I_{3,ijk}^{(D)}[\mu^2]&= \lim_{D\rightarrow 4}-\frac{(D-4)}{2}I_{3,ijk}^{(D+2)}[1] \rightarrow -\frac{1}{2}\\
    \lim_{D\rightarrow 4} I_{2,ij}^{(D)}[\mu^2]&= \lim_{D\rightarrow
      4}-\frac{(D-4)}{2}I_{2,ij}^{(D+2)}[1] \rightarrow
    \frac{(q_i-q_j)^2}{6} - \frac{m_i^2+m_j^2}{2},
\end{split}
\end{align}
where we used Eq.~\eqref{eq:intdim}. Historically, one splits \ola s into a \textit{cut constructable}
part and a \textit{rational} part when doing unitarity
calculations \cite{Bern:1994zx,Bern:1994cg}. However, in this thesis we adopt the general $D$-dimensional picture as used in the relations in this section. Rational terms are generated by multiplying higher order terms (in
$\epsilon$) of the coefficients with the integrals, producing finite
contributions. In a calculation in $D=4$, all $\mathcal{O}(\epsilon)$
terms in the expansion of the coefficients are set to zero and the
rational part cannot be reconstructed. The
non-vanishing finite contributions to the rational part of \ola s are given by the integrals in
Eq.~\eqref{eq:finiterint} multiplied by the corresponding
coefficients and can only be reconstructed by $D$-dimensional
unitarity cuts. 

The $D=4-2\epsilon$-dimensional representation of the \ola~is thus given by
\begin{align}
\begin{split}
 A_n^{\text{1-loop}}=&\sum_{1\le
   i_1 < \dots < i_4 \le
   n}  \left( d^0_{i_1i_2i_3i_4} I_{4,i_1i_2i_3i_4}[1]-\frac{d^4_{i_1i_2i_3i_4}}{6}\right)\\
&+\sum_{1\le i_1 < \dots < i_3 \le n}\left(c^0_{i_1i_2i_3}
  I_{3,i_1i_2i_3}[1]-\frac{c^9_{i_1i_2i_3}}{2} \right)\\
&+\sum_{1\le i_1
  < \dots < i_2 \le n}\left(b^0_{i_1i_2}
  I_{2,i_1i_2}[1]+\left(\frac{(q_{i_1}-q_{i_2})^2}{6}-\frac{m_{i_1}^2+m_{i_2}^2}{2}\right)b^9_{i_1i_2}
\right)\\
&+\sum_{1\le i_1 \le n} a^0_{i_1}I_{1,i_1} + \mathcal{O}(\epsilon),
\end{split}
\end{align}
where we eliminated all vanishing integrals
(cf.~Eq.~\eqref{eq:vanint}). Furthermore, we used the explicit expressions in
Eq.~\eqref{eq:finiterint} for the non-vanishing integrals with $\mu^2$
insertions. The rational coefficients $d_i^0,d_i^4,c_i^0,c_i^9,b_i^0,b^9_i,a_i^0$ are multiplying box, triangle, bubble and tadpole scalar integrals
respectively and are functions of the external kinematics. Knowing these coefficients is tantamount to having
computed the full color-ordered \ola. The coefficients multiplying
spurious numerator terms are relevant only at intermediate steps. Also
those with $\mu^2$ insertions with vanishing integrals in the limit
$D=4-2\epsilon$ do not contribute to the final result, like the pentagon
coefficient $e^0$ and the box coefficient $d^2$, but are nevertheless
needed for a consistent computation. We will next see how the systematic application of multiple unitarity
cuts allows to determine all tensor coefficients and thereby
reconstruct the full \ola.


\section{Generalized Unitarity Method}
We employ the numerical unitarity method, which is based on a prescription of how to reconstruct the full \ola~by the consecutive
application of pentuple, quadruple, triple, double and single cuts
\cite{Bern:1997sc,Britto:2004nc}. We apply cuts at the level of
the integrand of a \ola, with the numerator parameterization given in
Eq.~\eqref{eq:intparam}. Central to this is the behavior of the
one-loop integrand in the factorization limit of
Eq.~\eqref{eq:faclimit}, i.e.\ it factorizes into the product of tree
amplitudes. The numerator tensors $\Delta^i(\ell)$ in
Eq.~\eqref{eq:intparam}, are then given in terms of residues of
multi-particle unitarity cuts, by
\begin{align}\label{eq:unitarityrel}
\begin{split}
  \Delta_{i_1i_2i_3i_4i_5}^5(\ell_{i_1\dots i_5}) &=
  \text{Res}_{i_1i_2i_3i_4i_5}\left[\Am_n(\ell_{i_1\dots i_5})\right]\
  ,\\
  \Delta_{i_1i_2i_3i_4}^4(\ell_{i_1\dots i_4}) &=
  \text{Res}_{i_1i_2i_3i_4}\left[\Am_n(\ell_{i_1\dots i_4})\right]-\sum_{j\neq
      i_1,i_2,i_3,i_4}\frac{\Delta_{i_1i_2i_3i_4j}^5(\ell_{i_1\dots
        i_4})}{D_j}\ ,\\
 \Delta_{i_1i_2i_3}^3(\ell_{i_1\dots i_3}) &=
  \text{Res}_{i_1i_2i_3}\left[\Am_n(\ell_{i_1\dots i_3})\right]-\sum_{j\neq
        i_1,i_2,i_3}\frac{\Delta^4_{i_1i_2i_3j}(\ell_{i_1\dots
          i_3})}{D_j} \\
     &\phantom{=} -\sum_{\substack{j,k\neq
          i_1,i_2,i_3\\j<k}}\frac{\Delta^5_{i_1i_2i_3jk}(\ell_{i_1\dots
          i_3})}{D_jD_k}\ ,\\
 \Delta_{i_1i_2}^2(\ell_{i_1i_2}) &=
  \text{Res}_{i_1i_2}\left[\Am_n(\ell_{i_1i_2})\right]-\sum_{j\neq
        i_1,i_2}\frac{\Delta^3_{i_1i_2j}(\ell_{i_1i_2})}{D_j}\\
&\phantom{=}-\sum_{\substack{j,k\neq
        i_1,i_2\\j< k}}\frac{\Delta^4_{i_1i_2jk}(\ell_{i_1i_2})}{D_jD_k} -
    \sum_{\substack{j,k,m\neq
        i_1,i_2\\j<k<m}}\frac{\Delta^5_{i_1i_2jkm}(\ell_{i_1i_2})}{D_jD_kD_m}\
    ,\\
 \Delta_{i_1}^1(\ell_{i_1})& =
  \text{Res}_{i_1}\left[\Am_n(\ell_{i_1})\right]-\sum_{j\neq
        i_1}\frac{\Delta^2_{i_1j}(\ell_{i_1})}{D_j}-\sum_{\substack{j,k\neq
        i_1\\j<k}}\frac{\Delta^3_{i_1jk}(\ell_{i_1})}{D_jD_k}\\
&\phantom{=}-\sum_{\substack{j,k,m\neq
        i_1\\j< k<m}}\frac{\Delta^4_{i_1jkm}(\ell_{i_1})}{D_jD_kD_m} -
    \sum_{\substack{j,k,m,n\neq
        i_1\\j<k<m<n}}\frac{\Delta^5_{i_1jkmn}(\ell_{i_1})}{D_jD_kD_mD_n}\
    .
\end{split}
\end{align}
The above equations are evaluated for a set of on-shell propagator momenta
$\ell_{i_1\dots i_k}$ as defined
in Eq.~\eqref{eq:simloop}. In order to isolate the contribution to a
coefficient $\Delta^i$ from Eq.~\eqref{eq:unitarityrel}, care has to be taken to correctly subtract the
contributions from topologies with more loop propagators to each
residue. For example, the residue $\text{Res}_{i_1i_2i_3i_4}\left[\Am_n(\ell_{i_1\dots i_4})\right]$ contains contributions from five-point topologies, that have to be
subtracted. These \textit{subtractions} rely on a hierarchical
dependency between all topologies contributing to a specific primitive
amplitude represented by a parent diagram. One can build the hierarchical
dependence by considering all possibilities of
\textit{pinching} propagators of a parent diagram. As an explicit
example of Eq.~\eqref{eq:unitarityrel}, we give some of the numerators of a $4$-point amplitude in
terms of the residues and subtractions from higher point
topologies\footnote{There is no $\Delta^5$ for a $4$-point amplitude.}
\begin{align}
\begin{split}
   \Delta_{1234}^4(\ell_{1234}) &=
  \text{Res}_{1234}\left[\Am_n(\ell_{1234})\right]\ ,\\
   \Delta_{234}^3(\ell_{234}) &=
  \text{Res}_{234}\left[\Am_n(\ell_{234})\right]-\frac{\Delta_{1234}^4(\ell_{234})}{D_1}\
  ,\\
\Delta_{23}^2(\ell_{23}) &=
  \text{Res}_{23}\left[\Am_n(\ell_{23})\right]-\frac{\Delta_{123}^3(\ell_{23})}{D_1}-\frac{\Delta_{234}^3(\ell_{23})}{D_4}-\frac{\Delta_{1234}^4(\ell_{23})}{D_1D_4}\
  ,\\
   \Delta_{3}^1(\ell_{3}) &=\text{Res}_{3}\left[\Am_n(\ell_{3})\right]-\frac{\Delta_{13}^2(\ell_{3})}{D_1}-\frac{\Delta_{23}^2(\ell_{3})}{D_2}-\frac{\Delta_{34}^2(\ell_{3})}{D_4}\\
&\hspace{3.6cm}-\frac{\Delta_{123}^3(\ell_{3})}{D_1D_2}-\frac{\Delta_{134}^3(\ell_{3})}{D_1D_4}-\frac{\Delta_{234}^3(\ell_{3})}{D_2D_4}-\frac{\Delta_{1234}^4(\ell_{3})}{D_1D_2D_4}\
.
\end{split}
\end{align}
For the sake of brevity, we do not give
explicit expressions for the loop momenta satisfying the cut
conditions. Depending on
the topology, the on-shell conditions in general do not fully
constrain the loop-momentum but lead to a variety of loop momentum
solutions, and moreover, require complex momenta to be satisfied. Parameterizations valid for massive particles and $D$-dimensional loop momenta can, for
example, be found in \cite{Kilgore2007,Giele:2008ve,Ita:2011hi}.

\subsubsection{Inverting the Unitarity Relations}
\label{sec:invunirel}
Starting with the pentagon numerator $\Delta^5$, the unitarity
relations Eq.~\eqref{eq:unitarityrel} give a clear prescription of how to
obtain the coefficients in the numerators $\Delta^i$. In a numerical application of the unitarity method, we sample the loop momentum and obtain $\Delta^i(\ell_{j_1,\ldots,j_i})$ computed
for a specific choice of propagator momenta. In order to identify the scalar
coefficients, by
which the final \ola~is completely defined, we need to perform an inversion
on the linear relations in Eq.~\eqref{eq:intparam}. In particular, for a
numerical implementation this
implies evaluating the cuts in Eq.~\eqref{eq:unitarityrel} multiple
times depending on the tensor-rank. 


The above is valid for general dimensionality $D$. In order to
simplify the construction and reduce computing time, we do a
\textit{two-step procedure} \cite{Giele:2008ve} to obtain all
coefficients. As a first step, we compute with a $D=4$ dimensional
on-shell phase-space, with all higher-dimensional contributions proportional to $\mu^2$ vanishing. The
numerator tensors in Eq.~\eqref{eq:intparam} simplify and start with
$\Delta^4$. Note that this corresponds to a particular way of choosing
the loop momentum. Depending on the rank, we then evaluate the
unitarity relations on multiple on-shell momenta. The resulting system of equations
can be solved by matrix inversion or the application of a discrete
Fourier transform as for example described in \cite{BlackHatI}. As a
second step, we compute the remaining numerator tensors on $D$-dimensional on-shell
phase spaces and reuse the coefficients obtained for $D=4$ in the
inversion. The latter is possible since the coefficients $e^i,d^i,c^i,b^i,a^i$ are independent of the loop
momentum.

As an explicit example, we take the computation of the box
coefficients $d^i$ for a fixed arrangement of momenta flowing into the
vertices of the parent diagram. We drop the indices associated to the
external momenta and get
\begin{align}
  \pmqty{\Delta^4(\ell_1)\\\Delta^4(\ell_2)\\\Delta^4(\ell_3)\\\Delta^4(\ell_4)\\\Delta^4(\ell_5)}=
\pmqty{1 & t_{1;1}&0&0&0\\1 & t_{1;2}&0&0&0\\1 & t_{1;3}&\mu^2_3&t_{2;3}\mu^2_3&\mu^4_3\\1 & t_{1;4}&\mu^2_4&t_{2;4}\mu^2_4&\mu^4_4\\1 & t_{1;5}&\mu^2_5&t_{2;5}\mu^2_5&\mu^4_5\\} \pmqty{d^0\\d^1\\d^2\\d^3\\d^4\\},
\end{align}
where the $\ell_i$ are solutions to the on-shell conditions in
$D$-dimensions and the associated variables $t_{j,i}$ and $\mu_i$
calculated from loop momentum $\ell_i$. We take the loop momentum solutions $\ell_1$
and $\ell_2$ being restricted to the $4$-dimensional subspace that is with
$\mu^2=0$. As described above, our \textit{two-step procedure} gives us a
lower block diagonal matrix by construction. As a first step, we
compute the cuts $\Delta^4(\ell_1)$ and $\Delta^4(\ell_2)$. Those are
cuts in $D=4$-dimensional space-time, which makes their computation
numerically cheaper. We then invert the upper part of the matrix and
obtain $d^0$ and $d^1$. In particular, we can reconstruct all tadpole
coefficients by working in $D=4$, since they do not have $\mu^2$
insertions. The second step then requires the evaluation of
$\Delta^4(\ell_3)$, $\Delta^4(\ell_4)$ and $\Delta^4(\ell_5)$, where
the loop momenta are $D$-dimensional momenta with $\mu^2\neq 0$. We
then invert the lower block of the matrix. The next Chapter
\ref{chap:fdf} is devoted to the computation of the cuts in $D$-dimensions.


The advantage of this two-step process is twofold. It simplifies the
inversion of the unitarity relations by turning them into a lower
block diagonal matrix. And more importantly, it simplifies the calculation of the cuts
$\Delta^i(\ell)$ for those loop momenta $\ell$ restricted to
$4$-dimensions by not having to extend the spinor algebra and momenta to higher
dimensional space-time.


\section{Massive Bubble and Tadpole Cuts}
\label{sec:massivebubble}
Among the new scalar integrals contained in the basis of master
integrals for massive particles (cf.~Sec.~\ref{sec:scmi}) are the
tadpole integral $I_1$ and the
bubble integral with a single massive leg in one corner $I_{2,m^2}$, which we
refer to as massive one-leg bubble integral. The coefficients
multiplying these integrals appear in the corresponding bubble and tadpole numerators in
Eq.~\eqref{eq:intparam}. Double and
single cuts on the corresponding topologies are required to
reconstruct these coefficients. In this section, we will
see that some care has to
be taken in order to treat these two new topologies
in the numerical unitarity framework described above.

\subsection{Massive One-Leg Bubble Cuts}
\label{sec:1legbub}
In a direct application of numerical unitarity, external
leg self-energy corrections would be included in double cuts on bubble
topologies with one massive leg. We shall call these massive \olb s. The
double cut on a massive \olb~leads to explicit
divergent factors, as shown in Fig.~\ref{fig:doublecut}. In traditional approaches to renormalization, Feynman diagrams with
self-energy insertions on massive external legs are discarded. They
are accounted for in wave function and mass renormalization. For a massless external leg, the
corresponding scalar integral is scaleless and therefore vanishing in
dimensional regularization. Double cuts on \olb~topologies are
therefore not considered for massless particles.


The one-loop amplitude on the massive \olb~double cut factorizes into two tree amplitudes~$\Amt(-\ell_{2,q},1_{\bar
  q},\ell_{1,g})$ and $\Amt(-\ell_{1,g},\dots,\ell_{2,{\bar{q}}})$, where the latter is
not well defined, since it contains a divergent propagator. We split it into those parts multiplying the divergent
propagator $R$ and
those free from it $F$
\begin{align}\label{eq:defolbc}
  \Amt(-\ell_{1,g},\dots,\ell_{2,{\bar{q}}})=\frac{R(-\ell_{1,g},\dots,\ell_{2,{\bar{q}}})}{(\ell_{1,g}-\ell_{2,{\bar{q}}})^2-m^2_{\bar{q}}}+F(-\ell_{1,g},\dots,\ell_{2,\bar{q}}),
\end{align}
with the invariant mass of the momenta flowing through the intermediate
propagator being fixed by momentum conservation
$(\ell_{1,g}-\ell_{2,{\bar{q}}})^2=1_{\bar{q}}^2=m^2_{\bar{q}}$. Since it is equal to the quark mass,
the first term in Eq.~\eqref{eq:defolbc} is singular.


Three different possibilities on how to reconcile generalized
unitarity and cuts of massive \olb s are known in the literature:
First, the
explicit removal of the unwanted contributions due to EGKM
\cite{Ellis:2008ir} and their inclusion by wave function and mass renormalization. Second, the introduction of a regulator for the divergent tree which
has to be accounted for throughout the calculation, due to Britto and
Mirabella \cite{Britto:2011cr}. Third, a recent proposal by Badger et
al.~\cite{Badger:2017gta} to match both infrared and ultraviolet poles in
both $4-2\epsilon$ and $6-2\epsilon$ dimensions and thereby avoid the
calculation of both one-leg massive bubbles and tadpoles. The
latter comes at the cost of constructing an effective Lagrangian for the
higher-dimensional calculation.

\begin{equation*}
\begin{tikzpicture}[baseline=(m1)]
  \def\leglength{1}
  \def\blobpos{2.5}
  \def\cutshift{0.3}
  \def\cutlength{1.0}

  %The double cut lines
  \pgfmathsetmacro\vertcut{(\blobpos-1)/2+1}
  \draw[very thick] (\vertcut,\cutshift) -- (\vertcut,\cutshift+\cutlength);
  \draw[very thick] (\vertcut,-\cutshift) -- (\vertcut,-\cutshift-\cutlength);

  \node at (-0.3,0) {\(1_{\bar{q}}\)};
  \begin{feynman}[inline=(m1)]
    \vertex[blob] (m) at (\blobpos, 0) {};
    \vertex (m1) at (1, 0);
    \vertex (i) at (0, 0);
    \vertex (a) at (\blobpos+\leglength*1.4,0);
    \vertex (b) at (\blobpos+ \leglength,\leglength);
    \vertex (c) at (\blobpos+ \leglength,- \leglength);
    \diagram* {
      i -- [fermion, thick] (m1)
       -- [gluon, half left,momentum={[arrow shorten=0.3]\(\ell_1\)}] (m)
       -- [half left, thick,out=85,in=100,momentum={[arrow shorten=0.3]\(\ell_2\)}] (m1),
       (m) -- [gluon] (a),
       (m) -- [gluon] (b),
      (m) -- [fermion,  thick] (c),
    };
  \end{feynman}
\end{tikzpicture}
=
\begin{tikzpicture}[baseline=(m1)]
  \def\leglength{0.75}
  \def\blobpos{2.5}
  \def\secondb{\blobpos+1}
  \def\cutshift{0.2}
  \def\cutlength{1.0}
 
  %The double cut lines
  \pgfmathsetmacro\vertcut{(\blobpos-1)/2+1}
  %\def\vertcut{1.75}
  \draw[very thick] (\vertcut,\cutshift) -- (\vertcut,\cutshift+\cutlength);
  \draw[very thick] (\vertcut,-\cutshift) --
  (\vertcut,-\cutshift-\cutlength);

  %Cross over whole contribution
  \def\hg{1.3}
  \def\wg{0.5}
  \def\bg{3.5}
  \draw[ultra thick,red] (\wg,\hg) -- (\wg+\bg,-\hg);
  \draw[ultra thick,red] (\wg,-\hg) -- (\wg+\bg,\hg);
 
  \begin{feynman}[inline=(m1)]
    \vertex (m1) at (1, 0);
    \vertex (m) at (\blobpos, 0);
    \vertex[blob] (m2) at (\secondb, 0){};
    \vertex (a) at (\secondb+\leglength*1.4,0);
    \vertex (b) at (\secondb+ \leglength,\leglength);
    \vertex (c) at (\secondb+ \leglength,- \leglength);
    \diagram* {
      i -- [fermion, thick] (m1)
       -- [gluon, half left] (m)
       -- [half left, thick,out=85,in=95] (m1),
       (m) -- [fermion, thick] (m2),
       (m2) -- [gluon] (a),
       (m2) -- [gluon] (b),
      (m2) -- [fermion,  thick] (c),
    };
  \end{feynman}
\end{tikzpicture}+\sum
\begin{tikzpicture}[baseline=(m1)]
  \def\leglength{0.75}
  \def\blobpos{2.5}
  \def\secondb{\blobpos+1}
  \def\cutshift{0.2}
  \def\cutlength{1.0}
  \def\sblob{0.6}
 
  %The double cut lines
  \pgfmathsetmacro\vertcut{(\blobpos-1)/2+1}
  %\def\vertcut{1.75}
  \draw[very thick] (\vertcut,\cutshift) -- (\vertcut,\cutshift+\cutlength);
  \draw[very thick] (\vertcut,-\cutshift) -- (\vertcut,-\cutshift-\cutlength);
 
  \begin{feynman}[inline=(m1)]
    \vertex (m1) at (1, 0);
    \vertex[blob] (m) at (\blobpos, \sblob){};
    \vertex[blob] (m2) at (\blobpos, -\sblob){};
    \vertex (a) at (\blobpos+ \leglength,-\sblob+ \leglength*0.9);
    \vertex (b) at (\blobpos+ \leglength*1.2,\sblob);
    \vertex (c) at (\blobpos+ \leglength,-\sblob- \leglength*0.9);
    \diagram* {
      i -- [fermion, thick] (m1)
       -- [gluon, quarter left] (m),
      (m2) -- [quarter left, thick] (m1),
      (m)  -- [gluon] (m2),
       (m) -- [gluon] (b),
       (m2) -- [gluon] (a),
      (m2) -- [fermion,  thick] (c),
    };
  \end{feynman}
\end{tikzpicture}
\end{equation*}
\vspace{-1.2cm}
\captionedequationset{Double cut on a massive one-leg bubble topology. The
  blobs denote tree-like contributions and the sum in the right term runs over the possible ways to distribute the external
  particles on the two tree amplitudes. We discard contributions from
  external leg self-energy insertions.\label{fig:doublecut}} 

We follow the proposition by EGKM \cite{Ellis:2008ir}
to remove the wave-function graph contribution to the double cut
appearing in \olb s. This is particularly well suited for
calculations employing Berends-Giele recursions for the tree
generation. Computing the double cut trees, we interrupt the recursion
whenever a wave-function graph situation is identified. We thereby
remove any divergent tree contribution. In particular, this implies
that the two cut particles are not allowed to join in a vertex without
previous interactions. Due to this
removal, gauge invariance of the tree amplitude is broken and only restored
after combination with the appropriate counterterms (cf.~Sec.~\ref{sec:mrenorm}). This requires in
particular to consistently compute in the same gauge, which for simplicity we choose to be Feynman gauge, i.e.\ the sum over polarization states is given by
\begin{align}\label{eq:gluonpolsum}
  \sum_{s=1}^{D_s}\epsilon_s^\mu\epsilon_s^\nu = - g^{\mu\nu}.
\end{align}
The above polarization sum includes unphysical degrees of
freedom. In the discussed double-cut of Fig.~\ref{fig:doublecut} however, there is always a quark
in the loop and therefore no additional ghost particles need to be introduced.

\subsection{Massive Tadpole Cuts}
\label{sec:massivetadpoles}
The master integral basis for massive particles contains the scalar
one-point integral $I_1$, a tadpole. Its coefficients can be reconstructed by the application of single
cuts. Note that the scalar one-point integral is not to be confused with
\textit{tadpole Feynman diagrams}. They have numerator insertions depending on
the theory and vanish for both QED and QCD. In an application of
unitarity single-cuts, vanishing tadpole Feynman diagrams would contribute. This section is concerned with the proper handling of
single cuts. For massless particles, scalar one-point integrals are scaleless and vanish in dimensional
regularization. Therefore, for massless processes, one does not have
to deal with the described issues.

\begin{equation*}
\begin{tikzpicture}[baseline=(m)]
  \def\leglength{1}
  \def\blobpos{2.5}
  \def\cutshift{0.5}
  \def\cutlength{1.0}

  %The double cut lines
  \draw[very thick] (\cutshift,0) -- (\cutshift+\cutlength,0);
  %\draw[very thick] (\vertcut,-\cutshift) -- (\vertcut,-\cutshift-\cutlength);

  \begin{feynman}[inline=(m1)]
    \vertex[blob] (m) at (\blobpos, 0) {};
    \vertex (m1) at (1, 0);
    \vertex (a) at (\blobpos+\leglength*1.4,\leglength*0.5);
    \vertex (b) at (\blobpos+ \leglength,\leglength);
    \vertex (c) at (\blobpos+ \leglength*1.4,- \leglength*0.5);
    \vertex (d) at (\blobpos+ \leglength,- \leglength);
    \diagram* {
       (m1) -- [fermion, half left,momentum={[arrow shorten=0.3]\(\ell_1\)},thick] (m)
       -- [fermion,half left,thick] (m1),
       (m) -- [gluon] (a),
       (m) -- [fermion,thick] (b),
      (m) -- [gluon] (c),
      (m) -- [anti fermion,  thick] (d),
    };
  \end{feynman}
\end{tikzpicture}
=
\begin{tikzpicture}[baseline=(m1)]
  \def\leglength{1}
  \def\blobpos{3.5}
  \def\cutshift{0.5}
  \def\cutlength{0.9}

  %The double cut lines
  \draw[very thick] (\cutshift,0) -- (\cutshift+\cutlength,0);
  %\draw[very thick] (\vertcut,-\cutshift) -- (\vertcut,-\cutshift-\cutlength);
  %Cross over whole contribution
  \def\hg{1.1}
  \def\wg{0.5}
  \def\bg{3.2}
  \draw[ultra thick,red] (\wg,\hg) -- (\wg+\bg,-\hg);
  \draw[ultra thick,red] (\wg,-\hg) -- (\wg+\bg,\hg);
  \begin{feynman}[inline=(m1)]
    \vertex (m1) at (1, 0);
    \vertex (m2) at (2.5, 0);
    \vertex[blob] (m) at (\blobpos, 0){};
    \vertex (a) at (\blobpos+\leglength*1.4,\leglength*0.5);
    \vertex (b) at (\blobpos+ \leglength,\leglength);
    \vertex (c) at (\blobpos+ \leglength*1.4,- \leglength*0.5);
    \vertex (d) at (\blobpos+ \leglength,- \leglength);
    \diagram* {
       (m1) -- [fermion, half left,thick] (m2)
       -- [fermion,half left,thick] (m1),
       (m2) -- [gluon] (m),
       (m) -- [gluon] (a),
       (m) -- [fermion,thick] (b),
      (m) -- [gluon] (c),
      (m) -- [anti fermion,  thick] (d),
    };
  \end{feynman}
\end{tikzpicture}+\sum
\begin{tikzpicture}[baseline=(m1)]
  \def\leglength{1}
  \def\blobpos{2.0}
  \def\cutshift{0.5}
  \def\cutlength{1.0}
  \def\by{0.6}

  %The double cut lines
  \draw[very thick] (\cutshift,0) -- (\cutshift+\cutlength,0);
  %\draw[very thick] (\vertcut,-\cutshift) -- (\vertcut,-\cutshift-\cutlength);

  \begin{feynman}[inline=(m1)]
    \vertex (m1) at (1, 0);
    \vertex[blob] (m) at (\blobpos,\by ) {};
    \vertex[blob] (m2) at (\blobpos, -\by) {};
    \vertex (a) at (\blobpos+\leglength*1.4,\leglength*0.5);
    \vertex (b) at (\blobpos+ \leglength,\leglength);
    \vertex (c) at (\blobpos+ \leglength*1.4,- \leglength*0.5);
    \vertex (d) at (\blobpos+ \leglength,- \leglength);
    \diagram* {
       (m1) -- [fermion, quarter left,thick] (m),
       (m1) -- [anti fermion, quarter right,thick] (m2),
       (m) -- [gluon] (m2),
       (m) -- [gluon] (a),
       (m) -- [fermion,thick] (b),
      (m2) -- [gluon] (c),
      (m2) -- [anti fermion,  thick] (d),
    };
  \end{feynman}
\end{tikzpicture}
\end{equation*}
\vspace{-0.5cm}
\captionedequationset{Single cut on a massive tadpole topology, with
  the first term on the left, $R$, denotes tadpole Feynman diagram
  contributions and the second one non-tadpole
  like terms $N$. The
  blobs denote tree-like contributions and the sum runs over all possibilities
  to distribute the external particles amongst the two blobs. We
  discard contributions from tadpole Feynman diagrams.\label{fig:singlecut}} 

We use a direct approach to obtain tadpole coefficients by following the unitarity prescription,
with explicit loop momentum parameterization given for example in
\cite{Ellis:2007br}. An explicit prescription of how to deal with
divergent contributions that appear in single
cuts is not explicitly described in the literature. In this section, we will
illustrate the source of these divergent contributions and propose a
consistent scheme to compute single cuts based on the same
principle that we applied for the double cut
(cf.~Sec.~\ref{sec:1legbub} and \cite{Ellis:2008ir}). For processes with the particles in the loop having an identical mass $m^2$, a second
possibility exists to obtain the single coefficient of the scalar tadpole
integral by matching the UV pole structure \cite{Bern:1995db,Badger2008,Badger:2017gta}. The
only source of UV divergencies of \ola s are bubble and
tadpole integrals, which are either independent of kinematical
invariants or depend on the mass $m^2$ of loop particles. Knowing the coefficients of the bubble integrals
and the UV pole structure of the full amplitude, one can deduce the coefficient of the
scalar tadpole integral. For more details, see App.~\ref{sec:uvmatch}.

There are two conceptual
challenges in the direct unitarity approach when dealing with single cuts and both are
related to tree-like contributions that diverge. The one-loop amplitude in
Fig.~\ref{fig:singlecut} on the massive single cut corresponds to a single
tree amplitude $\Amt(-\ell_{1,\bar{q}},\dots,\ell_{1,q})$. This tree amplitude is not well
defined, since it contains singular terms in \textit{several}
factorization channels. The first type corresponds to tadpole Feynman
diagrams. These are not considered in traditional calculations based
on Feynman diagrams since they vanish in dimensional regularization. By Lorentz invariance they must be proportional to the momentum $p^\mu$ of the
gluon, which is zero $p^\mu=0$ by momentum conservation. We can
split the tree into the part $R$ multiplying the apparently divergent
propagator and the non-tadpole like term $N$ and obtain
\begin{align}\label{eq:scparts}
  \Amt(-\ell_{1,\bar{q}},\dots,\ell_{1,q})=\frac{R(-\ell_{1,\bar{q}},\dots,\ell_{1,q})}{p^2}+N(-\ell_{1,\bar{q}},\dots,\ell_{1,q}),
\end{align}
where momentum conservation fixes the momentum $p$ to be
zero $p^\mu=0$. We discard the tadpole Feynman diagram contributions appearing in
the single cut. Again, the tree generation with a Berends-Giele
recursion is beneficial. Whenever a
situation is identified where the two cut quarks directly join into a
vertex without any previous interaction, we stop the recursion and the
tadpole Feynman diagram is discarded. The appearance of a tadpole
Feynman diagram in the single cut has not been explicitly discussed in
the literature, but can be seen as being contained in the treatment of
the double cut introduced by EGKM in \cite{Ellis:2008ir}.

\begin{minipage}{1.0\linewidth}
\begin{eqnarray*}
\sum
\begin{tikzpicture}[baseline=(m1)]
  \def\leglength{1}
  \def\blobpos{2.0}
  \def\cutshift{0.5}
  \def\cutlength{1.0}
  \def\by{0.6}

  %The double cut lines
  \draw[very thick] (\cutshift,0) -- (\cutshift+\cutlength,0);
  %\draw[very thick] (\vertcut,-\cutshift) -- (\vertcut,-\cutshift-\cutlength);

  \begin{feynman}[inline=(m1)]
    \vertex (m1) at (1, 0);
    \vertex[blob] (m) at (\blobpos,\by ) {};
    \vertex[blob] (m2) at (\blobpos, -\by) {};
    \vertex (a) at (\blobpos+\leglength*1.4,\leglength*0.5);
    \vertex (b) at (\blobpos+ \leglength,\leglength);
    \vertex (c) at (\blobpos+ \leglength*1.4,- \leglength*0.5);
    \vertex (d) at (\blobpos+ \leglength,- \leglength);
    \diagram* {
       (m1) -- [fermion, quarter left,thick] (m),
       (m1) -- [anti fermion, quarter right,thick] (m2),
       (m) -- [gluon] (m2),
       (m) -- [gluon] (a),
       (m) -- [fermion,thick] (b),
      (m2) -- [gluon] (c),
      (m2) -- [anti fermion,  thick] (d),
    };
  \end{feynman}
\end{tikzpicture}
&=&
\begin{tikzpicture}[baseline=(m1)]
  \def\leglength{1}
  \def\blobpos{2.0}
  \def\cutshift{0.5}
  \def\cutlength{1.0}
  \def\by{0.6}

  %The double cut lines
  \draw[very thick] (\cutshift,0) -- (\cutshift+\cutlength,0);
  %\draw[very thick] (\vertcut,-\cutshift) -- (\vertcut,-\cutshift-\cutlength);

  \begin{feynman}[inline=(m1)]
    \vertex (m1) at (1, 0);
    \vertex (m) at (\blobpos,\by );
    \vertex[blob] (m2) at (\blobpos, -\by) {};
    \vertex (a) at (\blobpos+\leglength,-\by+0.4);
    \vertex (b) at (\blobpos+ \leglength,\leglength);
    \vertex (c) at (\blobpos+ \leglength*1.2,- \by);
    \vertex (d) at (\blobpos+ \leglength,- \leglength);
    \diagram* {
       (m1) -- [fermion, quarter left,thick] (m),
       (m1) -- [anti fermion, quarter right,thick] (m2),
       (m) -- [gluon] (m2),
       (m) -- [fermion,thick] (b),
       (m2) -- [gluon] (a),
      (m2) -- [gluon] (c),
      (m2) -- [anti fermion,  thick] (d),
    };
  \end{feynman}
\end{tikzpicture}
+
\dots\\
&=&\begin{tikzpicture}[baseline=(m1)]
  \def\leglength{1}
  \def\blobpos{2.0}
  \def\cutshift{0.5}
  \def\cutlength{1.0}
  \def\by{0.6}
  \def\bshift{0.9}

  %The double cut lines
  \draw[very thick] (\cutshift,0) -- (\cutshift+\cutlength,0);
  %\draw[very thick] (\vertcut,-\cutshift) -- (\vertcut,-\cutshift-\cutlength);
  \def\hg{1.1}
  \def\wg{0.5}
  \def\bg{3.2}
  \draw[ultra thick,red] (\wg,\hg) -- (\wg+\bg,-\hg);
  \draw[ultra thick,red] (\wg,-\hg) -- (\wg+\bg,\hg);
  \begin{feynman}[inline=(m1)]
    \vertex (m1) at (1, 0);
    \vertex (m) at (\blobpos,\by );
    \vertex (m2) at (\blobpos, -\by) ;
    \vertex (m3)[blob] at (\blobpos+\bshift, -\by) {};
    \vertex (a) at (\blobpos+\bshift+\leglength,-\by+0.4);
    \vertex (b) at (\blobpos+ \leglength,\leglength);
    \vertex (c) at (\blobpos+\bshift+ \leglength*1.2,- \by);
    \vertex (d) at (\blobpos+\bshift+ \leglength,- \leglength);
    \diagram* {
       (m1) -- [fermion, quarter left,thick] (m),
       (m1) -- [anti fermion, quarter right,thick] (m2),
       (m) -- [gluon] (m2),
      (m2) -- [anti fermion,  thick] (m3),
       (m) -- [fermion,thick] (b),
       (m3) -- [gluon] (a),
      (m3) -- [gluon] (c),
      (m3) -- [anti fermion,  thick] (d),
    };
  \end{feynman}
\end{tikzpicture}+\sum\dots+\dots
\end{eqnarray*}
\vspace{-0.5cm}
\captionedequationset{The non-tadpole like part $N$ of the single cut contains contributions from external leg
  self-energy corrections that have to be removed. They are accounted
  for in the renormalization.\label{fig:singlecut2}} 
\end{minipage}


The second conceptual challenge is that the non-tadpole like term $N$
in Eq.~\eqref{eq:scparts} also contains singular terms. They correspond
to external leg self-energy Feynman diagrams. In
Sec.~\ref{sec:1legbub}, we saw that they contain a divergent
propagator. The part $N$ of Eq.~\eqref{eq:scparts} contains a sum over all possible ways of distributing the external
legs to the two tree-like subgraphs, represented by blobs. Amongst the
Feynman diagrams contained therein, we have external leg self-energy contributions, as is apparent in the second line of Fig.~\ref{fig:singlecut2}. We
apply the same treatment as in the previous Sec.~\ref{sec:1legbub} for
massive one-leg bubbles. The singular contributions are removed by
stopping the Berends-Giele recursion whenever such a configuration
appears during the computation of the tree amplitude. Again, we
explicitly break gauge invariance which is only restored after
combination with the appropriate counter terms. The removal of these
singular contributions from the single cut makes the direct
application of numerical unitarity feasible.

\section{Tree Amplitude Generation with Berends-Giele Recursions}
\label{sec:BGrec}
Generalized unitarity relates color-ordered primitive \ola s to the
product of color-ordered tree
amplitudes. An efficient and flexible way of computing tree amplitudes are
Berends-Giele recursion relations \cite{Berends:1987me} for off-shell
currents. They were originally
devised for pure gluon amplitudes but can easily be extended to an
arbitrary spectrum. In algorithms based on off-shell recursion
relations, so called currents with a single off-shell leg and $n$ on-shell legs
are related to those with fewer on-shell legs. The possible
interactions between the currents are determined by the
theory at hand. Berends-Giele recursions work for complex
momenta as well as in any integer-dimensional space-time. Therefore, they are well suited for the
application in generalized unitarity and we implemented a new tree
generator in the \BlackHat{} library for the computation of unitarity
cuts for massive particles. Note, that
the fine-grained control over the contributing Feynman diagrams to each
current is beneficial for our treatment of double and single cuts with
massive particles (cf.~Sec.~\ref{sec:massivetadpoles} and
Sec.~\ref{sec:1legbub}). In particular, the removal of certain Feynman
diagrams does not affect the
efficiency of the computation during the evaluation phase. Compared to on-shell recursions like BCFW
\cite{Britto2005c,Britto2005f}, that are based on factorization
properties of the tree amplitudes, Berends-Giele recursion relations have the
advantage of being flexible, e.g.\ with respect to the space-time
dimensionality, and robust since they rely on a small set of rules to be
implemented.

\subsubsection{A Toy Example for Off-Shell Tree Recursions}
\label{sec:toybg}
The basic principle of BG recursions can easily be demonstrated in a vectorial toy
example with a three-point interaction only. For simplicity we consider the example of a
four-point tree amplitude $\Amt(1,2,3,4)$ with a fixed order of
external legs. We start by defining the \textit{current}
$J^\mu(2,3,4)$, which is constructed from an ordered on-shell tree amplitude by
stripping one leg off. In this example we remove the external leg labeled
by $1$. To recover the tree amplitude, we multiply the current by both
the wave function $\epsilon^\mu(1)$ of a massless spin-1 particle and
the inverse propagator
$D_1$\footnote{Numerically, the divergent propagator of the external
  leg is dropped in the final step of the recursion.} of the
removed external leg
\begin{align}
 \Amt(1,2,3,4)=D_1 J^\mu(1,2,3)\epsilon_\mu(4) .
\end{align}
The recursive steps to construct the current $J^\mu(2,3,4)$ are graphically represented in
Fig.~\ref{fig:toy3pt}. The external on-shell legs join into the shaded
blob, representing a tree-like contribution. The main idea of the recursion is that the current
leg must interact via one of the interactions furnished by the theory. In our simplified example, we
insert the allowed three-point interaction. Then, we sum over all
possible ways to distribute the ordered external legs to the
subcurrents. The same reasoning is applied to the smaller
subcurrents, cf.~the second row. The endpoint of the recursion are the off-shell
one-point currents $J^\mu(i)\equiv \epsilon^\mu(i)$. 
\begin{eqnarray*}
\begin{tikzpicture}[baseline=(m)]
  \def\leglength{1.2}
  \def\blobpos{1.0}
  \def\rshift{0.6}
  \def\lshift{-0.6}

  \draw (\lshift,-\leglength) --
  (\blobpos+\leglength+\rshift,-\leglength) --
  (\blobpos+\leglength+\rshift,+\leglength) -- (\lshift,\leglength) -- (\lshift,-\leglength);

  \begin{feynman}[inline=(m)]
    \vertex (m) at (-0.3, 0) {\(\mu\)};
    \vertex[blob] (m1) at (\blobpos,0 ) {};
    \vertex (a) at (\blobpos+\leglength,\leglength*0.7) {\(1\)};
    \vertex (b) at (\blobpos+ \leglength*1.2,0) {\(2\)};
    \vertex (c) at (\blobpos+ \leglength,-\leglength*0.7) {\(3\)};
    \diagram* {
      (m)  -- [gluon,thick] (m1),
      (m1) -- [gluon,thick] (a),
      (m1) -- [gluon,thick] (b),
      (m1) -- [gluon,thick] (c),
    };
  \end{feynman}
\end{tikzpicture}
&=&
\begin{tikzpicture}[baseline=(m)]
  \def\leglength{1.2}
  \def\curlength{0.8}
  \def\blobpos{1.0}
  \def\curx{\blobpos+\curlength+\leglength*0.8}
  \def\cury1{\curlength*0.7+\leglength*0.5}
  \def\cu2{\curlength*0.7-\leglength*0.5}

  \def\rshift{0.8}
  \def\lshift{1.35}
  \def\dshift{0.95}
  \def\ushift{0.25}

  \draw (\lshift,-\leglength+\dshift) --
  (\blobpos+\leglength+\rshift,-\leglength+\dshift) --
  (\blobpos+\leglength+\rshift,+\leglength+\ushift) -- (\lshift,\leglength+\ushift) -- (\lshift,-\leglength+\dshift);


  \begin{feynman}[inline=(m)]
    \vertex (m) at (-0.3, 0) {\(\mu\)};
    \vertex (m1) at (\blobpos,0 );
    \vertex[blob] (a) at (\blobpos+\curlength,\curlength*0.7){};
    \vertex (a1) at (\curx,\cury1) {\(1\)};
    \vertex (a2) at (\curx,\cu2) {\(2\)};
    \vertex (c) at (\blobpos+ \leglength,-\leglength*0.7) {\(3\)};
    \diagram* {
      (m)  -- [gluon,thick] (m1),
      (m1) -- [gluon,thick] (a),
      (m1) -- [gluon,thick] (c),
      (a) -- [gluon,thick] (a1),
      (a) -- [gluon,thick] (a2),
    };
  \end{feynman}
\end{tikzpicture}
+
\begin{tikzpicture}[baseline=(m)]
  \def\leglength{1.2}
  \def\curlength{0.8}
  \def\blobpos{1.0}
  \def\curx{\blobpos+\curlength+\leglength*0.8}
  \def\cury1{-\curlength*0.7+\leglength*0.5}
  \def\cu2{-\curlength*0.7-\leglength*0.5}


  \def\rshift{0.8}
  \def\lshift{1.35}
  \def\dshift{-0.25}
  \def\ushift{-0.95}

  \draw (\lshift,-\leglength+\dshift) --
  (\blobpos+\leglength+\rshift,-\leglength+\dshift) --
  (\blobpos+\leglength+\rshift,+\leglength+\ushift) -- (\lshift,\leglength+\ushift) -- (\lshift,-\leglength+\dshift);


  \begin{feynman}[inline=(m)]
    \vertex (m) at (-0.3, 0) {\(\mu\)};
    \vertex (m1) at (\blobpos,0 );
    \vertex (a) at (\blobpos+\leglength,\curlength*0.7){\(1\)};
    \vertex (a1) at (\curx,\cury1) {\(2\)};
    \vertex (a2) at (\curx,\cu2) {\(3\)};
    \vertex (c)[blob] at (\blobpos+ \curlength,-\curlength*0.7) {};
    \diagram* {
      (m)  -- [gluon,thick] (m1),
      (m1) -- [gluon,thick] (a),
      (m1) -- [gluon,thick] (c),
      (c) -- [gluon,thick] (a1),
      (c) -- [gluon,thick] (a2),
    };
  \end{feynman}
\end{tikzpicture}\\[30pt]
\begin{tikzpicture}[baseline=(a)]
  \def\leglength{1.2}
  \def\curlength{0.8}
  \def\blobpos{1.0}
  \def\curx{\blobpos+\curlength+\leglength*0.8}
  \def\cury1{\leglength*0.5}
  \def\cu2{-\leglength*0.5}

  \def\rshift{0.8}
  \def\lshift{0.4}
  \def\dshift{0.95-\curlength*0.75}
  \def\ushift{0.25-\curlength*0.75}

  \draw (\lshift,-\leglength+\dshift) --
  (\blobpos+\leglength+\rshift,-\leglength+\dshift) --
  (\blobpos+\leglength+\rshift,+\leglength+\ushift) -- (\lshift,\leglength+\ushift) -- (\lshift,-\leglength+\dshift);


  \begin{feynman}[inline=(a)]
    %\vertex (m) at (-0.3, 0) {\(\mu\)};
    \vertex (m1) at (\blobpos-0.3,0 ) {\(\mu\)};
    \vertex[blob] (a) at (\blobpos+\curlength,0){};
    \vertex (a1) at (\curx,\cury1) {\(1\)};
    \vertex (a2) at (\curx,\cu2) {\(2\)};
    %\vertex (c) at (\blobpos+ \leglength,-\leglength*0.7) {\(3\)};
    \diagram* {
      %(m)  -- [gluon,thick] (m1),
      (m1) -- [gluon,thick] (a),
      %(m1) -- [gluon,thick] (c),
      (a) -- [gluon,thick] (a1),
      (a) -- [gluon,thick] (a2),
    };
  \end{feynman}
\end{tikzpicture}
&=&
\begin{tikzpicture}[baseline=(a)]
  \def\leglength{1.2}
  \def\curlength{0.8}
  \def\blobpos{1.0}
  \def\curx{\blobpos+\curlength+\leglength*0.8}
  \def\cury1{\leglength*0.5}
  \def\cu2{-\leglength*0.5}


  \begin{feynman}[inline=(a)]
    %\vertex (m) at (-0.3, 0) {\(\mu\)};
    \vertex (m1) at (\blobpos-0.3,0 ) {\(\mu\)};
    \vertex (a) at (\blobpos+\curlength,0);
    \vertex (a1) at (\curx,\cury1) {\(1\)};
    \vertex (a2) at (\curx,\cu2) {\(2\)};
    %\vertex (c) at (\blobpos+ \leglength,-\leglength*0.7) {\(3\)};
    \diagram* {
      %(m)  -- [gluon,thick] (m1),
      (m1) -- [gluon,thick] (a),
      %(m1) -- [gluon,thick] (c),
      (a) -- [gluon,thick] (a1),
      (a) -- [gluon,thick] (a2),
    };
  \end{feynman}
\end{tikzpicture}
\hspace{3.0cm}
\begin{tikzpicture}[baseline=(a)]
  \def\leglength{1.2}
  \def\curlength{0.8}
  \def\blobpos{1.0}
  \def\curx{\blobpos+\curlength+\leglength*0.8}
  \def\cury1{\leglength*0.5}
  \def\cu2{-\leglength*0.5}

  \def\rshift{0.8}
  \def\lshift{0.4}
  \def\dshift{0.95-\curlength*0.75}
  \def\ushift{0.25-\curlength*0.75}

  \draw (\lshift,-\leglength+\dshift) --
  (\blobpos+\leglength+\rshift,-\leglength+\dshift) --
  (\blobpos+\leglength+\rshift,+\leglength+\ushift) -- (\lshift,\leglength+\ushift) -- (\lshift,-\leglength+\dshift);


  \begin{feynman}[inline=(a)]
    %\vertex (m) at (-0.3, 0) {\(\mu\)};
    \vertex (m1) at (\blobpos-0.3,0 ) {\(\mu\)};
    \vertex[blob] (a) at (\blobpos+\curlength,0){};
    \vertex (a1) at (\curx,\cury1) {\(2\)};
    \vertex (a2) at (\curx,\cu2) {\(3\)};
    %\vertex (c) at (\blobpos+ \leglength,-\leglength*0.7) {\(3\)};
    \diagram* {
      %(m)  -- [gluon,thick] (m1),
      (m1) -- [gluon,thick] (a),
      %(m1) -- [gluon,thick] (c),
      (a) -- [gluon,thick] (a1),
      (a) -- [gluon,thick] (a2),
    };
  \end{feynman}
\end{tikzpicture}
=
\begin{tikzpicture}[baseline=(a)]
  \def\leglength{1.2}
  \def\curlength{0.8}
  \def\blobpos{1.0}
  \def\curx{\blobpos+\curlength+\leglength*0.8}
  \def\cury1{\leglength*0.5}
  \def\cu2{-\leglength*0.5}


  \begin{feynman}[inline=(a)]
    %\vertex (m) at (-0.3, 0) {\(\mu\)};
    \vertex (m1) at (\blobpos-0.3,0 ) {\(\mu\)};
    \vertex (a) at (\blobpos+\curlength,0);
    \vertex (a1) at (\curx,\cury1) {\(2\)};
    \vertex (a2) at (\curx,\cu2) {\(3\)};
    %\vertex (c) at (\blobpos+ \leglength,-\leglength*0.7) {\(3\)};
    \diagram* {
      %(m)  -- [gluon,thick] (m1),
      (m1) -- [gluon,thick] (a),
      %(m1) -- [gluon,thick] (c),
      (a) -- [gluon,thick] (a1),
      (a) -- [gluon,thick] (a2),
    };
  \end{feynman}
\end{tikzpicture}
\end{eqnarray*}
\vspace{-0.5cm}
\captionedequationset{Berends-Giele recursion for a current
  $J^\mu(1,2,3)$ in a vectorial toy example with only three-point interactions.\label{fig:toy3pt}} 
We can easily generalize the recursive relation of our vectorial toy example to any number of external particles. The
recursive relations between the current $J(n,\dots,m)$, containing $m-n$ external on-shell legs, and those currents containing
less external legs can be written as
\begin{align}\label{eq:rectoy}
  J^\mu(n,\dots,m) = \frac{g^{\mu\nu}}{D_{m,n}}\sum_{i=n}^{m-1}V_{\nu\rho\sigma}(-p_{m,n},p_{n,i},p_{i+1,m})J^\rho(n,\dots,i) J^\sigma(i+1,\dots,m),
\end{align}
with the momenta sums defined as $p_{i,j}=\sum_{k=i}^jp_k$. The
indices of the inverse propagator $D_{i,j}$ label the involved
momenta sum $D_{i,j}=ip_{i,j}^2$ and we assume a momentum-dependent three-point interaction
$V^{\nu\rho\sigma}(p_1,p_2,p_3).$ The Berends-Giele-like tree recursions thereby provide a
systematic prescription to collect all the Feynman diagram contributions to a given
color-ordered tree amplitude. The recursion relations are valid for complex momenta in any
integer space-dimension and the currents can be multi-component
objects representing vectors, spinors or scalars in any integer
space-time dimension.


\subsubsection{Tree-Level Recursions}
\label{sec:bgqcd}
The recursion relations of the previous subsection can easily be extended to the full QCD
spectrum and an additional \ew~gauge boson, requiring recursive relations for different types of currents. We will present those for
pure gluon currents and those for currents composed of a single quark
pair and additional gluons. Gluons interact via both three- and
four-point interactions and thus the recursion relations read
\begin{align}\label{eq:recglue}
\begin{split}
  J^\mu(n,\dots,m) &=
  \frac{g^{\mu\nu}}{D_{m,n}}\left[\sum_{i=n}^{m-1}V_{\nu\rho\sigma}^{ggg}(-p_{m,n},p_{n,i},p_{i+1,m})J^\rho(n,\dots,i)
  J^\sigma(i+1,\dots,m)\right.\\
 &\left.+\sum_{i=n}^{m-2}\sum_{j=i+1}^{m-1}V_{\nu\rho\sigma\delta}^{gggg}J^\rho(n,\dots,i)
   J^\sigma(i+1,\dots,j)J^\delta(j+1,\dots,m)\right].
\end{split}
\end{align}
The color-ordered momentum dependent three-point interaction
$V_{\nu\rho\sigma}^{ggg}(p_1,p_2,p_3)$ and the momentum independent
four-point interaction $V_{\nu\rho\sigma\delta}^{gggg}$ are given in App.~\ref{sec:cofr}. Momentum sums
and the gluon propagator are defined as in the previous section. The
recursion terminates with the polarization state of the gluon
$J^\mu(i)\equiv\epsilon^\mu(i)$.


\begin{minipage}[c]{\linewidth}
\begin{eqnarray*}
\begin{tikzpicture}[baseline=(m)]
  \def\leglength{1.2}
  \def\blobpos{1.0}
  \def\rshift{0.6}
  \def\lshift{-0.6}

  %\draw (\lshift,-\leglength) --
  %(\blobpos+\leglength+\rshift,-\leglength) --
  %(\blobpos+\leglength+\rshift,+\leglength) -- (\lshift,\leglength) -- (\lshift,-\leglength);

  \begin{feynman}[inline=(m)]
    \vertex (m) at (-0.3, 0) {\(\mu\)};
    \vertex[blob] (m1) at (\blobpos,0 ) {};
    \vertex (a) at (\blobpos+\leglength,\leglength*0.7) {\(n\)};
    \vertex (b) at (\blobpos+ \leglength*1.2,0) {\(\vdots\)};
    \vertex (c) at (\blobpos+ \leglength,-\leglength*0.7) {\(m\)};
    \diagram* {
      (m)  -- [gluon,thick] (m1),
      (m1) -- [gluon,thick] (a),
      (m1) -- [gluon,thick] (b),
      (m1) -- [gluon,thick] (c),
    };
  \end{feynman}
\end{tikzpicture}
&=&\sum_{j=n}^{m-1}
\begin{tikzpicture}[baseline=(m)]
  \def\leglength{1.2}
  \def\curlength{0.8}
  \def\blobpos{1.0}
  \def\curx{\blobpos+\curlength+\leglength*0.8}
  \def\cury1{\leglength*0.7+\leglength*0.5}
  \def\cu2{\leglength*0.7-\leglength*0.5}

  \def\rshift{0.8}
  \def\lshift{1.35}
  \def\dshift{0.95}
  \def\ushift{0.25}

  %\draw (\lshift,-\leglength+\dshift) --
  %(\blobpos+\leglength+\rshift,-\leglength+\dshift) --
  %(\blobpos+\leglength+\rshift,+\leglength+\ushift) -- (\lshift,\leglength+\ushift) -- (\lshift,-\leglength+\dshift);


  \begin{feynman}[inline=(m)]
    \vertex (m) at (0.2, 0);
    \vertex (m1) at (\blobpos,0 );
    \vertex[blob] (a) at (\blobpos+\curlength,\leglength*0.7){};
    \vertex (a1) at (\curx,\cury1) {\(n\)};
    \vertex (a3) at (\curx*1.2,\leglength*0.7) {\(\vdots\)};
    \vertex (a2) at (\curx,\cu2) {\(j\)};

    \vertex[blob] (c) at (\blobpos+ \curlength,-\leglength*0.7) {};
    \vertex (b1) at (\curx+0.4,-\leglength*0.7+\leglength*0.5) {\(j+1\)};
    \vertex (b3) at (\curx*1.2,-\leglength*0.7) {\(\vdots\)};
    \vertex (b2) at (\curx+0.2,-\leglength*0.7-\leglength*0.5) {\(m\)};
    \diagram* {
      (m)  -- [gluon,thick] (m1),
      (m1) -- [gluon,thick] (a),
      (m1) -- [gluon,thick] (c),
      (a) -- [gluon,thick] (a1),
      (a) -- [gluon,thick] (a2),
      (a) -- [gluon,thick] (a3),
      (c) -- [gluon,thick] (b1),
      (c) -- [gluon,thick] (b2),
      (c) -- [gluon,thick] (b3),
    };
  \end{feynman}
\end{tikzpicture}
+\sum_{i=n}^{m-2}\sum_{j=i+1}^{m-1}
\begin{tikzpicture}[baseline=(m)]
  \def\leglength{1.2}
  \def\curlength{0.8}
  \def\blobpos{1.0}
  \def\curx{\blobpos+\leglength+1.2}
  \def\cury1{-\curlength*0.7+\leglength*0.5}
  \def\cu2{-\curlength*0.7-\leglength*0.5}
  \def\cu{0.4}

  \begin{feynman}[inline=(m)]
    \vertex (m) at (0.2, 0);
    \vertex (m1) at (\blobpos,0 );
    \vertex (a)[blob] at (\blobpos+\leglength,\curlength*1.3){};
    \vertex (b)[blob] at (\blobpos+\leglength+0.8,0){};
    \vertex (c)[blob] at (\blobpos+\leglength,-\curlength*1.3){};
    \vertex (a1) at (\curx,\curlength*1.3+\cu) {\(n\)};
    \vertex (a2) at (\curx+0.2,\curlength*1.3){\(\vdots\)};
    \vertex (a3) at (\curx,\curlength*1.3-\cu) {\(i\)};
    \vertex (b1) at (\curx+1.0,+\cu) {\(i+1\)};
    \vertex (b2) at (\curx+0.2+0.8,0){\(\vdots\)};
    \vertex (b3) at (\curx+0.8,-\cu) {\(j\)};
    \vertex (c1) at (\curx+0.2,-\curlength*1.3+\cu) {\(j+1\)};
    \vertex (c2) at (\curx+0.2,-\curlength*1.3){\(\vdots\)};
    \vertex (c3) at (\curx,-\curlength*1.3-\cu) {\(m\)};
    \diagram* {
      (m)  -- [gluon,thick] (m1),
      (m1) -- [gluon,thick] (a),
      (m1) -- [gluon,thick] (b),
      (m1) -- [gluon,thick] (c),
      (a) -- [gluon,thick] (a1),
      (a) -- [gluon,thick] (a3),
      (b) -- [gluon,thick] (b1),
      (b) -- [gluon,thick] (b3),
      (c) -- [gluon,thick] (c1),
      (c) -- [gluon,thick] (c3),
    };
  \end{feynman}
\end{tikzpicture}
\end{eqnarray*}
\vspace{-0.5cm}
\captionedequationset{Berends-Giele recursion for a gluonic current
  $J^\mu(n,\dots,m)$. The sums run over all possible ways to
  distribute the on-shell legs with fixed cyclic ordering amongst the subcurrents.\label{fig:bggluoncur}} 
\end{minipage}


Similar Berends-Giele recursions can be devised for tree amplitudes
containing quark pairs. We have to distinguish currents with a
quark off-shell leg and those with an anti-quark off-shell leg. Fig.~\ref{fig:bgqcur} shows the
Berends-Giele recursions for a quark current with $n$-gluons and a
single quark line. The quark has two possibilities to interact. It can
radiate a gluon to the left or to the
right of the fermion line. These are related by a minus sign
$V_{\mu}^{qg\bar{q}}=-V_{\mu}^{q\bar{q}g}$, see color-ordered Feynman rules in App.~\ref{sec:cofr}. The recursion for the quark current
$Q(g_1,g_2,\dots,g_{n_1};\bar{q};g_{n_1+1},\dots,g_{n_1+n_2})$
therefore reads
\begin{align}\label{eq:recq}
\begin{split}
  Q =&
S^{q}(-p_{1,n_1+n_2})\left[\sum_{j=1}^{n_1}V_{\mu}^{qg\bar{q}}Q(g_{j+1},\dots,g_{n_1};\bar{q};g_{n_1+1},\dots,g_{n_1+n_2})J^\mu(g_1,\dots,g_j)\right.\\
 &\left.+\sum_{j=n_1}^{n_1+n_2-1}V_{\mu}^{q\bar{q}g}Q(g_{1},\dots,g_{n_1};\bar{q};g_{n_1+1},\dots,g_{j})J^\mu(g_{j+1},\dots,g_{n_1+n_2})\right],
\end{split}
\end{align}
with the quark propagator given by $S^{q}(p)=\frac{i(\slashed{p}+m)}{p^2-m^2}$ and momenta sums
$p_{i,j}=\sum_{k=i}^jp_k$ defined as above. Fermion currents, propagators
and interactions are objects in
spinor space, which means that their relative positioning matters. We suppressed the related spinorial indices in Eq.~\eqref{eq:recq}.

\begin{eqnarray*}
\begin{tikzpicture}[baseline=(m)]
  \def\leglength{1.2}
  \def\blobpos{1.5}
  \def\2blobpos{3.0}
  \def\rshift{0.6}
  \def\lshift{-0.6}
  \def\shift{0.6}
  \def\yshift{1}

  \begin{feynman}[inline=(m)]
    \vertex (m) at (0, 0) {\(\mu\)};
    \vertex[blob] (m1) at (\blobpos,0 ) {};
    \vertex (m2) at (\2blobpos,0 ) ;
    \vertex (a) at (\blobpos+\shift,\yshift) {\(n_1\)};
    \vertex (a1) at (\blobpos,\yshift+0.2) {\(\dots\)};
    \vertex (a2) at (\blobpos-\shift,\yshift) {\(1\)};
    \vertex (c) at (\blobpos+\shift+0.2,-\yshift) {\(n_1+1\)};
    \vertex (c1) at (\blobpos,-\yshift-0.3) {\(\dots\)};
    \vertex (c2) at (\blobpos-\shift-0.2,-\yshift) {\(n_1+n_2\)};
    \diagram* {
      (m)  -- [anti fermion,thick] (m1),
      (m1) -- [anti fermion,thick] (m2),
      (m1) -- [gluon,thick] (a),
      (m1) -- [gluon,thick] (a1),
      (m1) -- [gluon,thick] (a2),
      (m1) -- [gluon,thick] (c),
      (m1) -- [gluon,thick] (c1),
      (m1) -- [gluon,thick] (c2),
    };
  \end{feynman}
\end{tikzpicture}
&=&\sum_{j=1}^{n_1}
\begin{tikzpicture}[baseline=(m)]
   \def\leglength{1.2}
  \def\blobpos{1.5}
  \def\2blobpos{3.0}
  \def\rshift{0.6}
  \def\lshift{-0.6}
  \def\shift{0.6}
  \def\yshift{1}
\def\mup{1}

  \begin{feynman}[inline=(m)]
    \vertex (m0) at (-1, 0);
    \vertex (m) at (0, 0);
    \vertex[blob] (mup) at (0, \mup){};
    \vertex (b) at (\shift,\mup+\yshift) {\(j\)};
    \vertex (b1) at (0,\mup+\yshift+0.2) {\(\dots\)};
    \vertex (b2) at (-\shift,\mup+\yshift) {\(1\)};
    \vertex[blob] (m1) at (\blobpos,0 ) {};
    \vertex (m2) at (\2blobpos,0 ) ;
    \vertex (a) at (\blobpos+\shift,\yshift) {\(n_1\)};
    \vertex (a1) at (\blobpos,\yshift+0.3) {\(\dots\)};
    \vertex (a2) at (\blobpos-\shift,\yshift) {\(j+1\)};
    \vertex (c) at (\blobpos+\shift+0.2,-\yshift) {\(n_1+1\)};
    \vertex (c1) at (\blobpos,-\yshift-0.3) {\(\dots\)};
    \vertex (c2) at (\blobpos-\shift-0.2,-\yshift) {\(n_1+n_2\)};
    \diagram* {
      (m0)  -- [anti fermion,thick] (m),
      (m)  -- [anti fermion,thick] (m1),
      (m1) -- [anti fermion,thick] (m2),
      (m) -- [gluon,thick] (mup),
      (m1) -- [gluon,thick] (a),
      (m1) -- [gluon,thick] (a1),
      (m1) -- [gluon,thick] (a2),
      (m1) -- [gluon,thick] (c),
      (m1) -- [gluon,thick] (c1),
      (m1) -- [gluon,thick] (c2),
      (mup) -- [gluon,thick] (b),
      (mup) -- [gluon,thick] (b1),
      (mup) -- [gluon,thick] (b2),
    };
  \end{feynman}
\end{tikzpicture}
+\sum_{j=n_1}^{n_1+n_2-1}
\begin{tikzpicture}[baseline=(m)]
   \def\leglength{1.2}
  \def\blobpos{1.5}
  \def\2blobpos{3.0}
  \def\rshift{0.6}
  \def\lshift{-0.6}
  \def\shift{0.6}
  \def\yshift{1}
\def\mup{1}

  \begin{feynman}[inline=(m)]
    \vertex (m0) at (-1, 0);
    \vertex (m) at (0, 0);
    \vertex[blob] (mup) at (0, -\mup){};
    \vertex (b) at (\shift+0.3,-\mup-\yshift) {\(j+1\)};
    \vertex (b1) at (0,-\mup-\yshift-0.2) {\(\dots\)};
    \vertex (b2) at (-\shift-0.3,-\mup-\yshift) {\(n_1+n_2\)};
    \vertex[blob] (m1) at (\blobpos,0 ) {};
    \vertex (m2) at (\2blobpos,0 ) ;
    \vertex (a) at (\blobpos+\shift,\yshift) {\(n_1\)};
    \vertex (a1) at (\blobpos,\yshift+0.2) {\(\dots\)};
    \vertex (a2) at (\blobpos-\shift,\yshift) {\(1\)};
    \vertex (c) at (\blobpos+\shift+0.2,-\yshift) {\(n_1+1\)};
    \vertex (c1) at (\blobpos,-\yshift-0.3) {\(\dots\)};
    \vertex (c2) at (\blobpos-\shift-0.2,-\yshift) {\(j\)};
    \diagram* {
      (m0)  -- [anti fermion,thick] (m),
      (m)  -- [anti fermion,thick] (m1),
      (m1) -- [anti fermion,thick] (m2),
      (m) -- [gluon,thick] (mup),
      (m1) -- [gluon,thick] (a),
      (m1) -- [gluon,thick] (a1),
      (m1) -- [gluon,thick] (a2),
      (m1) -- [gluon,thick] (c),
      (m1) -- [gluon,thick] (c1),
      (m1) -- [gluon,thick] (c2),
      (mup) -- [gluon,thick] (b),
      (mup) -- [gluon,thick] (b1),
      (mup) -- [gluon,thick] (b2),
    };
  \end{feynman}
\end{tikzpicture}
\end{eqnarray*}
\vspace{-0.5cm}
\captionedequationset{Berends-Giele recursion for the quark
  current $Q$. The vertices for radiating the gluon to the right
  $V_{\mu}^{qg\bar{q}}$ and to the left
  $V_{\mu}^{q\bar{q}g}$ of the fermion line are related by a minus
  sign. The sums run over all possibilities to
  distribute the on-shell legs with fixed cyclic ordering amongst the
  subcurrents.\label{fig:bgqcur}} 
 The
tree amplitude $
\Amt(q;g_1,g_2,\dots,g_{n_1};\bar{q};g_{n_1+1},\dots,g_{n_1+n_2})$ can
thus be recovered from the quark current $Q(g_1,g_2,\dots,g_{n_1};\bar{q};g_{n_1+1},\dots,g_{n_1+n_2})$ by 
\begin{align}
 \Amt=\bar{u}(p_q)\left(-i(\slashed{p}_q-m)\right)Q.
\end{align}
Extending the Berends-Giele recursion relations to an arbitrary number of quark pairs poses
no problem in a numerical implementation but their graphical
representation can become quite cumbersome. Also the addition of a
color-neutral gauge boson poses no problem in a flexible numerical
implementation. Color-neutral particles are not strictly ordered, they
have to be inserted at all possible positions between the colored
particles with fixed cyclic ordering. Care has to be taken to consistently
treat color-neutral particles during both color decomposition and tree generation. Explicit expressions for propagators and vertices
for a $W$-boson can be found in App.~\ref{sec:cofr}.


\subsubsection{Multi-Cut Trees For Numerical Unitarity}
\label{sec:singlecutunit}
In the context of a numerical unitarity application, tree amplitudes
always appear as products of tree amplitudes in unitarity cuts. Adjacent tree amplitudes in the cut are connected
by an on-shell leg, whose momentum enters the amplitudes with opposite signs. The sum over helicity states on the cut corresponds to
the polarization state sum and is given by
\begin{align}\label{eq:projecpol}
\begin{split}
  P^{\mu\nu}_g \equiv -g^{\mu \nu} & =
  \sum_{s=\pm}\epsilon_s^{*\mu}(p)\epsilon_s^\nu(-p)\\
P_q(p) \equiv (\slashed{p} + m) &= \sum_{s=\pm} \bar{u}_s(p)v_s(-p),
\end{split}
\end{align}
Instead of inserting explicit on-shell states for the cut leg, we replace the sum over helicity states on the cut by the corresponding
polarization projectors $P^{\mu\nu}_g$ and $P_q(p)$ of Eq.~\eqref{eq:projecpol}. We refer to the product
of cut trees summed over helicities, with all but one state sum replaced by
the corresponding projectors, as a \textit{multi-cut tree}. In
Fig.~\ref{fig:multicut}, the graphical representation of such a
multi-cut tree appearing in the calculation of a quadruple cut is
shown, where four loop propagators are set on-shell. 
\begin{eqnarray*}
\begin{tikzpicture}[baseline=(m)]

\def\1up{1.0}
\def\up{2.0}
\def\hup{0.5}
\def\cutl{0.5}

\draw[ultra thick] (0,\1up-\cutl) -- (0,\1up+\cutl);
\draw[ultra thick] (0,-\1up-\cutl) -- (0,-\1up+\cutl);
\draw[ultra thick] (-\1up-\cutl,0) -- (-\1up+\cutl,0);
\draw[ultra thick] (-\1up-\cutl,0) -- (-\1up+\cutl,0);
\draw[ultra thick] (1.25,-0.25) -- (1.75,-0.75);
\draw[ultra thick] (1.25,0.25) -- (1.75,0.75);

\draw[->,thick] (\up+0.7,0.5) -- (\up+0.7,-0.5);
\node at (\up+1,0) {$\ell_1$};

\node at (0,\up) {$P_q$};
\node at (0,-\up) {$P_g$};
\node at (-\up,0) {$P_g$};
\node at (\up,0) {$\sum_s$};
  \begin{feynman}[inline=(m)]
    \vertex (v1) at (-\1up,\1up);
    \vertex (v2) at (\1up,\1up);
    \vertex (v3) at (\1up,-\1up);
    \vertex (v4) at (-\1up,-\1up);
    \vertex (o1) at (-\up,\up){\(3\)};
    \vertex (o2) at (\up,\up){\(4\)};
    \vertex (o3) at (\up,-\up){\(1\)};
    \vertex (o4) at (-\up,-\up){\(2\)};
    \vertex (o5) at (\up*0.75,\1up*0.5);
    \vertex (o6) at (\up*0.75,-\1up*0.5);
    \vertex (c1) at (\1up,0);
    \vertex (c2) at (\up,-\1up);
    \diagram* {
      (v1)  -- [fermion,thick] (v2) --  [fermion,thick] (o5),
      (o6) -- [fermion,thick] (v3)--  [gluon,thick] (v4) --  [gluon,thick] (v1), 
      (o1) -- [fermion,thick] (v1),
      (o2) -- [gluon,thick] (v2),
      (o3) -- [anti fermion,thick] (v3),
      (o4) -- [gluon,thick] (v4),
    };
  \end{feynman}
\end{tikzpicture}
\end{eqnarray*}
\vspace{-0.5cm}
\captionedequationset{A multi-cut tree appearing in a quadruple cut, where the sum over
  polarization states is replaced by the corresponding projectors on
  all but one cut line. \label{fig:multicut}} 

The multi-cut tree for the quadruple cut in Fig.~\ref{fig:multicut} thus corresponds to 
\begin{align}
\begin{split}
  &\sum_{s1,s_2,s_3,s_4=\pm}\Amt(-\ell_1,1_q,\ell_2)\Amt(-\ell_2,2_g,\ell_3)\Amt(-\ell_3,3_{\bar{q}},\ell_4)\Amt(-\ell_4,4_g,\ell_1)\\
  &=\sum_{s1=\pm}J_\mu(-\ell_1,1_q,\ell_2)P_g^{\mu\nu}\MC_{\nu\rho}(-\ell_2,2_g,\ell_3)P_g^{\rho\sigma}\MC_{\sigma}(-\ell_3,3_{\bar{q}},\ell_4)P_q(\ell_4)\bar{Q}(-\ell_4,4_{g},\ell_1),
\end{split}
\end{align}
where we introduced \textit{multi-currents} denoted by $\MC$, that is an amplitude
with two off-shell legs and therefore two open indices (we suppress spinorial indices for simplicity). The summation
in the first line is over all spin states of internal cut particles and
that in the second line only over the spin-states of the internal
particle, whose polarization sum is not replaced by the
corresponding projector. Depending on
the legs that are stripped off, the indices can be spinorial, vectorial, a
mixture of those or the current can be a scalar. The quark
projector $P_q$ and anti-fermion current $\bar{Q}$ live in spinor
space, whereas the gluon projector $P_g$ and gluon current $J^\mu$ are
vectors. Compared to computing tree amplitude by tree amplitude, the numerical
implementation of multi-cut trees simplifies the
book-keeping due to the smaller number of objects that are computed. Instead
of computing several trees for each helicity configuration of internal
cut legs, we only have to sum over the helicity configurations of the single
cut leg with explicit on-shell states. Furthermore, handling the polarization sum by a
projector reduces the number of operations for the computations in Feynman
gauge. This is particularly relevant for a spectrum with an increased number of
states. For example, we will see in the next chapter that our treatment of the
rational terms leads to a massive spin-1 particle with three
polarization states. The unphysical
degrees of freedom circulating in the loop that are introduced by the gluonic polarization sum in
Feynman gauge are canceled, since one cut-line is still computed with
on-shell states. We implemented a caching
mechanism for currents inside single tree amplitudes but also across
all tree amplitudes required for the computation of a \ola. In
particular, this allows for the efficient evaluation of the latter.
