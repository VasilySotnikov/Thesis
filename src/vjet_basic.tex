Electroweak vector boson production in association
with multiple light jets has been extensively studied during the 7 and
8 TeV runs of the LHC by both the ATLAS and CMS
experiments~\cite{ATLAS:ratio2017,Aad:2014qxa,ATLASRatioWZ14,ATLAS7Zjets13,Aad:2012en,Aad:2011qv,Aad:2011xn,Aad:2010ab,Khachatryan:2015ira,Khachatryan:2016fue,CMS8Zjets16,CMS7Zjets15,Khachatryan:2014uva,Chatrchyan:2013tna,Chatrchyan:2011ig,Chatrchyan:2011ne}. Recently,
the first
measurements of this process class at an energy of 13 TeV have been
presented by both
collaborations~\cite{Aaboud:2017hbk,Sirunyan:2017wgx}. These studies highlight the extent to which theoretical
predictions produced by either dedicated
calculations or by general Monte Carlo event generators can describe total rates
and differential distributions for signatures that involve leptons, missing
transverse energy and many light jets. Such characterizations are of key importance given that searches for new physics carried out at hadron colliders
target similar final states.

In recent years high precision results have become available with the NNLO QCD prediction for vector-boson
production with a single
jet~\cite{Boughezal:2015dva,Boughezal:2015ded,Gehrmann-DeRidder:2016zml}. Also other
refinements to NNLO QCD and NLO electroweak corrections for this process are developed which for
example have been used to describe backgrounds for dark matter
searches~\cite{Lindert:2017olm}. Calculations
for the high-multiplicity processes with four and five
jets~\cite{BH:Z4j,BH:W5j} in the final state remain the state
of the art at NLO QCD. The NLO electroweak corrections have been computed for up to
three light jets~\cite{Denner:2014ina,Kallweit:2014xda}, and combined
with QCD merging~\cite{Kallweit:2015dum}. The matching to parton showers
has been carried out for up to three jets in the final
state~\cite{Hoeche:2012ft}, and multi-jet merging has been studied
with up to two
jets~\cite{Hoeche:2012yf,Lonnblad:2012ix,Frederix:2012ps}.


In this part of the thesis we show dedicated
predictions for the 13~TeV LHC for $W+n$-jet and $Z+m$-jet production with
$n\leq 5$ and \mbox{$m\leq 4$}, as shown in \cite{Anger:2017nkq}. We use existing one-loop matrix elements from the original \BlackHat{} library~\cite{Berger:2008sj}\footnote{As opposed to the
new version of \BlackHat{} \cite{wbbpaper} which can handle massive quarks that was presented in the other parts of this thesis.} and provide results in publicly available Root~\cite{ROOT} \ntuple{} files. In particular we study theoretical uncertainties related to scale sensitivity and PDF
dependence of our predictions. To this end, we compare results obtained by using fixed-order dynamical
scales based on the total partonic transverse energy
as well as different variants of the \MINLO{} method~\cite{MINLO}, as implemented in \SHERPA{}~\cite{Sherpa} and a private implementation \cite{danielminlo}. A recent study of on-shell $t\bar t$ production in association with up to three light jets~\cite{ttjjj} has performed a similar comparison and the results obtained were largely consistent between the choices of dynamical scales. We extend the comparison of fixed-order scales
with the \MINLO{} method to high-multiplicity processes with four or
five light jets in the final state, where many scales are present. This type of comparison considers two very different approaches to the issue of scale choices and thereby helps to solidify the picture of uncertainties associated to the spurious dependence on unphysical renormalization and factorization scales.


In this Chapter~\ref{chap:vjet_basic} we summarize our
calculational setup, showing all kinematical information employed. We also present the implementation of dynamical
scales applied. In Chapter~\ref{chap:vjet_result} we present our results for total and
differential cross sections as well as for observable ratios and study their scale dependence and uncertainties
associated to PDFs.

%%%%%%%%%%%%%%%%%%%%%%%%%%%%%%%%%%%%%%%%%%%%

\section{Basic Setup}
\label{sec_setup}
The overall calculations are managed by the \SHERPA{}
package~\cite{Sherpa} and the required one-loop matrix elements are
provided by the original \BlackHat{}
library~\cite{Berger:2008sj}, as opposed to the new version of \BlackHat{} \cite{wbbpaper} which
  can handle massive quarks that was presented in the other
  parts of this thesis. Both Born and real-emission contributions
are computed by the matrix-element generator \COMIX{}~\cite{Comix}, which also
provides the necessary Catani-Seymour subtraction
terms~\cite{Catani:1996vz}. More details on the computational setup
can be found in Refs.~\cite{BH:W5j} and~\cite{BH:Z4j}. We include all
contributing subprocesses and confirm in
particular that 8-quark finite contributions to the real part in $W^\pm+5$-jet
production are negligible~\cite{BH:W5j}. Our results are stored
in Root~\cite{ROOT} \ntuple{} files and are publicly available. We use
an extension~\cite{Greiner:2016awe} of the \ntuple{} file format~\cite{BH:Ntuples} which allows for
extended reweighting procedures.


We use \texttt{CT14} LO (\texttt{CT14llo}) and NLO (\texttt{CT14nlo})
PDFs~\cite{CT14} at the respective orders, including the corresponding
definition of the strong coupling $\alpha_s$. To explore PDF
uncertainties, we use the corresponding \texttt{CT14nlo} error set and compare to predictions
generated with the PDF error sets of ABM~\cite{ABM}, MMHT~\cite{MMHT} and NNPDF 3.1~\cite{NNPDF}.
%
The lepton-pair invariant mass follows a relativistic Breit-Wigner distribution,
with $M_W=80.385$ GeV and $M_Z=91.1876$ GeV, and the widths are given by
$\Gamma_Z=2.4952$~GeV and $\Gamma_W=2.085$~GeV. We employ a diagonal CKM
matrix and use real values for the electroweak parameters.

We treat all light quarks ($u$, $d$, $s$, $c$, $b$) as massless
particles and do not include contributions from real or virtual top
quarks. We expect this to have
a percent-level effect on
cross-sections~\cite{BH:W4j,BH:Z4j,Campbell:2016tcu,wbbpaper}. We use
the leading-color approximation~\cite{Ita:2011ar} for the $V+4,5$-jets one-loop matrix elements which we find to be
precise at the level of 2\% of the total cross section in lower
jet-multiplicity calculations. We quote results for a single 
lepton (pair) flavor and both leptons are treated as massless, an approximation that can be 
applied to the electron or muon families.
%
Results presented are produced in fixed-order parton-level perturbation theory
and we do not apply any non-perturbative corrections to account for effects
associated to underlying event or hadronization.

\section{Kinematical Cuts}
\label{sec_kin}

We present NLO QCD results for $pp\rightarrow V+n$ jets at the
LHC with a center-of-mass energy of $\sqrt{s}=13$ TeV, with $n\le 5$ and $n\le 4$ for
$V=W^\pm$ and $Z$, respectively. We apply the following kinematical cuts:
\begin{eqnarray}
&& \pT^\jet > 30 \hbox{ GeV}\,, \hskip 1.5 cm 
|\eta^\jet|<3\,. 
\label{eq_jets}
\end{eqnarray}
Jets are defined using the anti-$k_{\textrm T}$
algorithm~\cite{antikT} with $R=0.4$. In general, we order jets in the transverse momentum $p_{\textrm T}$ and label them according to their hardness. For
all charged leptons we require:
\begin{eqnarray}
&& \pT^l > 20 \hbox{ GeV}\,, \hskip 1.5 cm 
|\eta^l|<2.5\,.
\label{eq_clep}
\end{eqnarray}
The transverse mass of $W^\pm$ bosons is defined by \mbox{$M_{\textrm
T}^W=\sqrt{2E_{\textrm T}^lE_{\textrm T}^\nu(1-\cos(\Delta
\phi_{l\nu}))}$} and we impose the following cuts on processes
involving a $W^\pm$ boson:
\begin{eqnarray}
&& \pT^\nu > 20 \hbox{ GeV}\,, \hskip 1.5 cm 
M_{\textrm T}^W>20 \hbox{ GeV}\,.
\label{eq_W}
\end{eqnarray}
For processes involving a $Z$ boson, we impose the following constraint on the
invariant mass of its decay products:
\begin{equation}
66 \hbox{ GeV}<M_{l^+l^-}<116 \hbox{ GeV}\,.
\label{eq_Z}
\end{equation}


\section{Dynamical Scale Choices}
\label{sec_scales}


In this section, we define the fixed-order scales employed,
and present a variant of the \MINLO{} \cite{MINLO} procedure, which we
will label \MINLOp{}. Finally we summarize the nomenclature used to label all dynamical scales considered.


\subsection{Fixed-Order Scales}
\label{sec_fo_scales}
We use two dynamical fixed-order scales. On the one hand, we use a scale based on the total partonic transverse energy $\mu_0=\HTpartonicp/2$, defined in
Eq.~\eqref{eq:htpart} and Section~\ref{sec:scale} of the previous part
of this thesis. This scale choice has proven to be a sensible
choice for weak-vector-boson production in association with jets as it tends to reduce the shape changes and global size of quantum corrections
when going from LO to NLO (see for
example~\cite{BH:W5j,BH:Z4j,BH:Wratios}). On the other hand, we use an additional scale denoted as
\begin{equation}
\HTpartonicpp=\frac{1}{2}\sum_j p_{\textrm T}^j+E_{\textrm T}^V\ ,
\label{eq_HTpp}
\end{equation}
where the sum runs over all final state partons. The scale $\HTpartonicpp$ differs from $\HTpartonicp/2$ by the prefactor of $E_{\textrm T}^V$. It is designed to match the invariant mass of the
lepton pair in kinematic configurations with very small hadronic transverse energy,
and in events of di-jet type the transverse momentum of the hardest QCD jet. We use a conventional variation of the central scale by factors
$(1/2,1/\sqrt{2},1,\sqrt{2},2)$ to assess the scale dependence, keeping factorization and renormalization
scales equal. In Subsection~\ref{sec_scale_notation} we provide our notation used to refer to the different scales employed.

\subsection{The \MINLOp{} Procedure}
\label{sec_minlo_gen}

In our study, we compare results obtained with the fixed-order scales described in the previous subsection to those obtained with a dynamical scale choice based on the recent \MINLO{} \cite{MINLO} reweighting procedure. We use the implementation in \SHERPA{}~\cite{Sherpa} that was validated in \cite{ttjjj} and denote it schematically as \MINLOp{} since it differs in some aspects from the original formulation. Additionally, we compare to the original formulation of the \MINLO{}
method~\cite{MINLO} for processes with fewer than three jets in the
final state by using the implementation of \cite{danielminlo}. In contract to the \MINLOp{} implementation, the original \MINLO{} variant
uses the Sudakov factors of~\cite{Catani:1991hj} and unordered clustering histories
are treated in a different manner.


The \MINLO{} reweighting procedure is inspired
by the next-to-leading logarithmic (NLL) branching formalism of Ref.~\cite{Catani1993kt}. It builds on an event-by-event identification
of the most likely branching history of the Born kinematics leading to the full $V+n-$parton final state using
a $k_T$-type clustering algorithm. In order to reflect the nature of QCD interactions, only $1\rightarrow 2$ branchings
consistent with elementary interaction vertices are allowed in the \MINLO{} procedure. Successive branchings take place at scales $q_1 , q_2,\ldots , q_N$, where the nodal scales $q_i$ correspond to the $k_T$ measure of the jet algorithm. The strong coupling associated to each node
in the branching tree is evaluated at the respective nodal scale $q_i$. In addition, we require a strong ordering of the nodal scales in the $k_T$ clustering in the \MINLOp{} implementation. If an inverted scale hierarchy is encountered the clustering is terminated. In this case we set the scale of the remaining $V+m$-parton (with $m<n$) ``core'' interaction,
$\mu_{\textrm core}$, to $\HTpartonicp/2$ (or when explicitly stated, to $\HTpartonicpp$) by default. This biases the scale choice for events with many hard scales towards $\HTpartonicp/2$
($\HTpartonicpp$), an effect that will be further discussed in Chapter~\ref{chap:vjet_result}. Note in particular that at very high energies there may be configurations where no clustering
can be performed at all: For example a $V+2$ jet event with
$p_{T,j1}\approx p_{T,j2}\gg m_{T,W}$ or a $V+5$ jet event where
$p_{T,j1}\approx p_{T,j2}\approx\ldots\approx p_{T,j5}$
and $y_{j1}\ll y_{j2}\ll\ldots\ll y_{j5}$. This is a considerable source of uncertainty because of the large available
phase space at the LHC. In general, the clustering procedure for a leading-order process of
$\mathcal{O}(\alpha_s^N)$ will yield a branching history with $M\le N$
ordered nodal scales $q_1\ldots q_M$ and a core interaction of
$\mathcal{O}(\alpha_s^{N-M})$ with scale $\mu_{\textrm core}>q_M$.
We then set the global renormalization scale $\mu_R$ to the geometric mean
$\mu_R^{N}=\mu_{\textrm core}^{N-M}\prod_{i=1}^{M}q_i$.

We then assign no-branching probabilities in the form of NLL
Sudakov form factors to both intermediate lines (connecting branching nodes $i$ and $j$) and external
lines to reflect the fact that no radiation above a resolution scale, given
by the lowest nodal
$k_T$ value in the clustering, should occur. External lines connected to the $i$-th branching are multiplied by a
factor $\Delta_a (q_{min},q_i)$, where the lowest branching scale $q_1=q_{min}$ is
identified as the resolution scale. Intermediate lines connecting nodes $j<i$
are dressed by factors $\Delta_a (q_{min},q_i)/ \Delta_a (q_{min},q_j)$.
Internal lines connected to the primary process are assigned form factors
between their respective scales and $\mu_{\textrm core}$. The factorization scale
$\mu_F$ used in the evaluation of the PDFs is set to the lowest scale, $\mu_F=q_1$. In the \MINLOp{} implementation, we use a physical definition of the Sudakov form factors, which is given by
\begin{equation}\label{eq:def_nll_sudakov}
  \begin{split}
    &\Delta_a(Q_0,Q)=\exp\left\{-\int_{Q_0}^Q\frac{{\textrm d} q}{q}
    \frac{\alpha_s(q)}{\pi}\sum_{b=q,g}\int_0^{1-q/Q}{\textrm d}z
    \left(z\,P_{ab}(z)+\delta_{ab}\frac{\alpha_s(q)}{2\pi}\frac{2C_a}{1-z}K\right)\right\}\;,
  \end{split}
\end{equation}
where~\cite{Catani:1990rr}
\begin{equation}
  K=\left(\frac{67}{18}-\frac{\pi^2}{6}\right)C_A-\frac{10}{9}T_R\,n_f\;,
\end{equation}
and $a=g,q$ corresponds to massless gluons and quarks, respectively.
Eq.~\eqref{eq:def_nll_sudakov} does not exceed unity and can therefore
be interpreted as a no-branching probability between the scales
$Q_0$ and $Q$, while maintaining the correct limiting behavior for $Q_0\ll Q$.

The \MINLO{} method generalized the above to NLO and thus requires some modifications~\cite{MINLO}. Virtual corrections and integrated IR-subtraction terms are treated identically
to the leading order case. Real-emission events have branching histories with
$M+1 \leq N+1$ ordered branchings, but they are treated as Born-like $M$-parton events
for the purpose of scale definition. This is achieved by discarding the softest branching,
i.e. if the $M+1$ step branching history is given by $q_0<q_1<\dots<q_K$, we set the
resolution scale to $q_1$. Consequently, the softest emission at NLO (with scale $q_0$)
is neither dressed with Sudakov factors nor does it enter the definitions of $\mu_R$
and $\mu_F$. The value of the additional strong coupling at NLO (the $N+1$-th power)
appearing in both real and virtual corrections is set to the average of all other values
of $\alpha_s$, i.e.\ $N\,\alpha_s^{(N+1)}=(N-M)\,\alpha_s(\mu_{\textrm core})+\sum_{i=1}^M\alpha_s(q_i)$.


In order to retain NLO accuracy of the full calculation, the Born configuration
receives correction terms that are proportional to the first-order expansion of the Sudakov
factors, Eq.~\eqref{eq:def_nll_sudakov}. Conventional scale uncertainties associated to the \MINLO{} method are estimated
using variations of $\mu_R$ and $\mu_F$ by constant factors of two. The scale of
the strong coupling in the Sudakov form factors remains fixed at the integration variable, $q$,
while it is varied in all other parts of the calculation, including the \MINLO{} counterterms
used to subtract the $\mc{O}(\alpha_s)$ expansion of the Sudakov factors.
Factorization scale variations in the \MINLOp{} procedure have been
discussed extensively in~\cite{ttjjj}. We perform them in the same manner,
i.e.\ we set $q_1$ equal to $\mu_F$.

\subsection{Nomenclature for Dynamical Scales Explored}
\label{sec_scale_notation}
Throughout this part of the thesis, we set renormalization and factorization scales equal
$\mu_R=\mu_F=\mu_0$ for fixed-order scales. The results labeled ``LO'' and ``NLO'' use the
central scale $\mu_0=\HTpartonicp/2$ by default, where $\HTpartonicp$ is defined
in Eq.~\eqref{eq:htpart}. When necessary, to distinguish the usage of the fixed-order
scales defined in Section~\ref{sec_fo_scales}, we write ``(N)LO
$\HTpartonicp/2$'' or ``(N)LO $\HTpartonicpp$'', where $\HTpartonicpp$ is
defined in Eq.~\eqref{eq_HTpp}. 

In the \MINLOp{} procedure described in Section~\ref{sec_minlo_gen} the default
core scale is $\mu_{\textrm core}=\HTpartonicp/2$. When considering variations of
this choice we explicitly write ``\MILNLOp{}~$\HTpartonicp/2$'' or
``\MILNLOp{}~$\HTpartonicpp$''. We also compare to the original formulation of the \MINLO{}
method~\cite{MINLO} for processes with fewer than three jets in the
final state. Compared to our implementation, this variant
uses the Sudakov factors of~\cite{Catani:1991hj} and unordered clustering histories
are treated in a different manner. Following the previous naming convention, we label those results as 
``\MILNLO{}~$\HTpartonicp/2$'' or ``\MILNLO{}~$\HTpartonicpp$'' depending on the
choice of core scale $\mu_{\textrm core}$ employed.
