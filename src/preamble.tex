%%% Definition der Dokumentklasse für A4-Papier und 12 Punkte Schriftgröße.
%%% Ränder sind für Klebebindung sowohl im A4- als auch im herunterskalierten
%%% A5-Ausdruck optimiert.
\documentclass[a4paper,titlepage,bibliography=totoc,12pt,BCOR=17mm,DIV=12,headinclude,footinclude=false]{scrbook}

\usepackage[utf8]{inputenc}

\usepackage{easy-todo}

\usepackage{tikz}
\usetikzlibrary{decorations.markings,decorations.text,calc,arrows,positioning,shapes}
\tikzstyle{decision} = [diamond, draw, fill=blue!20, text width=4.5em, text badly centered, node distance=2cm, inner sep=0pt]
\tikzstyle{block} = [rectangle, draw, fill=blue!20, text width=0.8\textwidth, text centered, rounded corners, minimum height=3em]
\tikzstyle{block2} = [rectangle, draw, text centered, rounded corners, minimum height=2em]
\tikzstyle{line} = [draw, -latex']
\tikzstyle{cloud} = [draw, ellipse,fill=red!20, node distance=3cm, minimum height=2em]
\tikzstyle{process} = [rectangle, minimum width=3cm, minimum height=2em, text centered, draw=black, fill=orange!30]

\usepackage{minibox}
\usepackage{mdframed}
\mdfsetup{
  skipabove=\baselineskip,
  skipbelow=\baselineskip
}


%%%%%%%%%%%%%%%%%%%%%%%%%%%%%%
%%% Math symbols etc
\usepackage{amsmath,amsfonts,amssymb}
\usepackage{bbold}
\usepackage{slashed}
\usepackage{physics}
\usepackage{booktabs}
\usepackage{tabularx,bigdelim}
\usepackage{multirow}
\usepackage{xfrac}
\usepackage{siunitx}
\usepackage{mathtools}

%%%%%%%%%%%%%%%%%%%%%%%%%%%%%%
%%% Grafics
\usepackage{graphicx}
\usepackage{caption}
%\usepackage{subcaption}
%%% Sidecap: Bildbeschriftung seitlich neben dem Bild
%\usepackage{sidecap}
\usepackage[caption=false]{subfig}

\usepackage{placeins}
\usepackage{adjustbox}

%%%%%%%%%%%%%%%%%%%%%%%%%%%%%%
%%% Header
%%% Fancyhdr: Nette Kopf- und Fusszeilen, werden weiter unten definiert.
\usepackage{fancyhdr}
%%% Definition der Kopf- und Fussleisten
\pagestyle{fancy}
\renewcommand{\chaptermark}[1]{% Kopfzeilen in Groß-/Kleinschreibung
\markboth{%
\chaptername\ \thechapter.%
\ #1}{}}
\renewcommand{\sectionmark}[1]{% Kopfzeilen in Groß-/Kleinschreibung
\markright{%
\thesection.%
\ #1}}


\usepackage{titlesec, blindtext, color} 

%\definecolor{gray75}{gray}{0.75}
%\newcommand{\hsp}{\hspace{20pt}}
%\titleformat{\chapter}[hang]{\Huge\bfseries}{\thechapter\hsp\textcolor{gray75}{|}\hsp}{0pt}{\Huge\bfseries}

%\renewcommand{\thechapter}{\Roman{chapter}}
\titleformat{\chapter}[display]
{\bfseries\Large}
{\filleft\MakeUppercase{\chaptertitlename} \Huge\thechapter}
{2ex}
{\titlerule
  \vspace{1ex}%
\filright}
[\vspace{2ex}%
\titlerule]

\titlespacing{\chapter}{0cm}{-1cm}{0.5cm}[0cm]

%\titleformat{\chapter}[display]
%  {\bfseries\Large}
%  {\filright\MakeUppercase{\chaptertitlename} \Huge\thechapter}
%  {1ex}
%  {\titlerule\vspace{1ex}\filleft}
%  [\vspace{1ex}\titlerule]

%%% Seitennumerierung vor dem eigentlichen Inhalt in römischen Zahlen
\pagenumbering{roman}

%%% Vermeidung von Hurenkindern und Schusterjungen
\clubpenalty = 10000  % schliesst Schusterjungen aus
\widowpenalty = 10000 % schliesst Hurenkinder aus


%%%%%%%%%%%%%%%%%%%%%%%%%%%%%%
%%% Einige Pakete für eine schöne Titelseite
\usepackage{everyshi,eso-pic,calc,ifthen,sty/wallpaper}
\usepackage{sty/diplomarbeit}

%%%%%%%%%%%%%%%%%%%%%%%%%%%%%%

%%% Eineinhalbfachen Zeilenabstand nutzen (Vorgabe Prüfungsamt)
\usepackage{setspace}
%\onehalfspacing
\typearea[current]{last}



\usepackage[%
  pdftitle={Scattering Amplitudes with Multi-Loop Generalized Numerical Unitarity Method},%
  pdfauthor={Vasily Sotnikov}%
]{hyperref}

\hypersetup{
    colorlinks=true,
    allcolors={blue!60!black}
    %linkcolor={red!50!black},
    %citecolor={blue!50!black},
    %urlcolor={blue!80!black}
}

%%% Mehrere aufeinanderfolgend numerierte Zitate mit einem Bindestrich zusammen
%%% fassen (z.B. [1-5] statt [1,2,3,4,5]).
\usepackage[numbers,sort&compress]{natbib}
\usepackage{doi}
%\usepackage{mcite}

%%% Hypernat beseitigt Fehler durch das
%%% Paket Hyperref in Zusammenhang mit dieser Kompression.
%\usepackage{hypernat}


%%%%%%%%%%%%%%%%%%%%%%%%%%%%%%

%-- Defines a CAPTION for a set of equations and numbers
%   the set of equations is numbered
% note: the caption adapts the text alignment to the length. 
\newcounter{captionedequationset} %numbering
\newdimen\captionlength
\newcommand{\captionedequationset}[1]{
    \refstepcounter{figure}% Step counter
    \setlength{\captionlength}{\widthof{#1}} %
    \addtolength{\captionlength}{\widthof{Figure~\thefigure: }}
    %If the caption is shorter than the line width then
    % the caption is centred, otherwise is flushed left.
    \ifthenelse{\lengthtest{\captionlength < \linewidth }} %
    {\begin{center}
            Figure~\thefigure: #1
        \end{center}} 
    { \begin{flushleft} 
        Figure~\thefigure: #1 %
        \end{flushleft}}}




%%%%%%%%%%%%%%%%%%%%%%%%%%%%%%
%%%%%%%%%%%%%%%%%%%%%%%%%%%%%%
%  NOT USE CURRENTLY
%%%%%%%%%%%%%%%%%%%%%%%%%%%%%%
%%%%%%%%%%%%%%%%%%%%%%%%%%%%%%
%%% Fancyref: Vereinfachte Referenzen. Labels werden im Format "PRÄFIX:LABEL"
%%%           erstellt. Dann wird mit \fref{PRÄFIX:LABEL} ein Verweis der Form
%%%           z.B. "Kapitel 4" oder "Abschnitt 4.5" statt nur "4" oder "4.5"
%%%           erstellt.
%%%           ACHTUNG: Fehlt im Label im \fref-Kommando das Präfix und der
%%%           Doppelpunkt, kommt es zu Fehlermeldungen, die ein fehlendes "}"
%%%           beklagen!
%%%           Liste der Präfixe und zugehöriger Objekte:
%%%           Object      Prefix
%%%           Chapter     chap
%%%           Section     sec
%%%           Equation    eq
%%%           Figure      fig
%%%           Table       tab
%%%           Enumeration enum
%%%           Footnote    fn
%%%           Anhang*     apx
%%%           *) Selbst definiertes Präfix

%%%\usepackage[german,plain]{fancyref}
%%% 
%%%\newcommand*{\fancyrefapxlabelprefix}{apx}
%%%\frefformat{plain}{\fancyrefapxlabelprefix}{%
%%%    Anhang\fancyrefdefaultspacing#1%
%%%}%
%%%\Frefformat{plain}{\fancyrefapxlabelprefix}{%
%%%    Anhang\fancyrefdefaultspacing#1%
%%%}%

%%%%%%%%%%%%%%%%%%%%%%%%%%%%%%
%%% URL: Zur besseren Darstellung von URLs z.B. in Literaturquellen (einbindbar
%%%      z.B. über "howpublished = \url{URL}")
%\usepackage{url}

%%%%%%%%%%%%%%%%%%%%%%%%%%%%%%
%%% Hyphenat: Problemlöser für Trennungsprobleme: TeX trennt keine Wörter, die
%%%           bereits einen Bindestrich beinhalten (z.B. Wasserstoff-Atom).
%%%           Um die automatische Worttrennung auch bei solchen Wörtern zu 
%%%           aktivieren, ersetzt man einfach den Bindestrich durch das 
%%%           Kommando \hyp{}, also z.B. "Wasserstoff\hyp{}Atom".
%\usepackage{hyphenat}

%%%%%%%%%%%%%%%%%%%%%%%%%%%%%%

%%% Mit diesen Kommandos kann man eigene Kapitel- und Sektionskopfzeilen
%%% einbinden. Anwendung:
%%%    \chapter{Langer Kapitelname}
%%%    \shortchapter{Kurzer Kapitelname für Kopfzeile}
%\newcommand{\shortchapter}[1]{% Erstellt eine eigene Kapitel-Kopfzeile
%\markboth{%
%\chaptername\ \thechapter.%
%\ #1}{}}
%\newcommand{\shortsection}[1]{% Erstellt eine eigene Sektions-Kopfzeile
%\markright{%
%\thesection.%
%\ #1}}

%%%%%%%%%%%%%%%%%%%%%%%%%%%%%%

\usepackage{cmap}
\usepackage{indentfirst}

% for ranges of equations
\usepackage{cleveref}

\graphicspath{{figures/}}

