NLO QCD predictions require the computation of virtual one-loop corrections. They appear as interference terms of
$n$-parton loop amplitudes $\Amp_n$ with $n$-parton tree level contributions $\Ampt_n$
\begin{align}\label{eq:interference}
 \sum_{c,h}2 \Re{{\Ampt_n}^*\Amp_n},
\end{align}
summed over color ($c$) and helicity ($h$) of the involved partons. We handle the dynamical
degrees of freedom of color and kinematics separately, which has several
advantages. A fixed cyclic ordering of external particles results in less
kinematical invariants for each color-ordered amplitude and thereby
makes the analytic structure more transparent, which was in particular
relevant for the development of
unitarity methods. In numerical
approaches, the smaller building blocks allow for a fine-grained
numerical precision monitoring and control. Furthermore, using a color decomposition allows for an
expansion of the virtual cross-section in powers of $1/N_C$, which gives a significant speedup for
the numerical phase space integration. The typically smaller subleading-color
contributions have to be evaluated on fewer phase space points in
order to reach the same overall integration error. 

In order to disentangle kinematical and color degrees of freedom of
QCD \ola s,
we apply a trace-based color decomposition \cite{Mangano1991}. As a
result, we can express \ola s as the
product of color factors and purely kinematic functions, so called
partial amplitudes. At one-loop, partial amplitudes are further
decomposed into the basic building blocks of gauge-invariant primitive amplitudes \cite{Bern1995b,Bern:1997sc} with a fixed
cyclic ordering of external particles. The handling of
color is independent of whether the involved quarks are massless or
massive, which is of particular relevance for our application. It allows us to use an existing algorithm to obtain the color
decomposition of one-loop amplitudes involving multiple quark lines as
described in Ref.~\cite{Ita:2011ar}. This algorithm has been successfully applied to massless
high-multiplicity processes involving an \ew~gauge boson \cite{BH:W4j,BH:W5j,BH:Z4j}.

\section{Partial Amplitudes}
\label{sec:partamp}
The vertices of QCD Feynman
rules are equipped with either fundamental generators
$(T^a)_i^{\bar{j}}$ or structure constants $f^{abc}$, where the indices of quarks are given by $\{i,j,\dots\}$ and those
of anti-quarks by $\{\bar{i},\bar{j},\dots\}$. In order to treat both fundamental and adjoint color indices on the
same footing, we rewrite the structure constants $f^{abc}$ appearing in gluon
interactions in terms of the fundamental representation $(T^a)_i^{\bar
  j}$ using 
\begin{align}\label{eq:repcs}
f^{abc} = - \frac{i}{\sqrt{2}}\text{Tr}\left([T^a,T^b]T^c\right).
\end{align}
with $\text{Tr}(T^aT^b) = \delta^{ab}$ such
that $T^a=\frac{\lambda^a}{\sqrt{2}}$. Whenever contracted adjoint
indices appear, we can apply the $SU(N_c)$ Fierz identity
\begin{align}\label{eq:fierz}
\sum_a (T^a)_i^{\bar j} (T^a)_k^{\bar l} = \delta_i^{\bar l} \delta_k^{\bar j} - \frac{1}{N_c}  \delta_i^{\bar j} \delta_k^{\bar l},
\end{align}
and thereby reduce the possible forms of appearing color factors. They
are constructed out of powers of $N_C$, traces
of generators $\Tr(T^aT^b\dots T^n)$, strings of generators with open
fundamental indices $(T^aT^b\dots T^n)_i^{\bar{j}}$ and Kronecker delta functions
$\delta_i^{\bar{j}}$. Here the gauge group parameter $N_C$ is left
unspecified and can be set to $N_C=3$ for physical QCD predictions. A \textit{partial amplitude} is defined as the kinematic terms multiplying a
specific color structure, by which we mean a combination of color generators. The generic expression for amplitudes involving up to six-quarks can
be found in \cite{Ita:2011ar}. As an example, for a two-quark amplitude at
tree level, we have
\begin{align}
  \Ampt_n  (1_q,2_{\bar q},3,\dots,n)=\sum_{\sigma \in
   S_{n-2}}(T^{a_{\sigma(3)}} \dots
 T^{a_{\sigma(n)}})_{i_1}^{\bar{j_2}}\Amt_n(1_q,\sigma(3),\dots,
 \sigma(n),2_{\bar{q}}),
\end{align}
where the coupling can be thought of as included in $\Amt_n$ and we sum over all permutations
$S_{n-2}$ of the $n-2$ gluons labels appearing in the partial tree
amplitudes $\Amt_n$. The permutations in $\sigma$ are also used for
the mapping of the adjoint gluonic
color indices. Note that for clarity, we suppress both helicity and color indices of
the involved particles in the expressions for the amplitudes
$\Ampt_n$. The color-decomposition of the corresponding
\ola~additionally contains traces of fundamental generators and is given by
\begin{align}\label{eq:c2q}
  \begin{split}
  \Amp_n  (&1_q,2_{\bar q},3,\dots,n)=\\
&  \sum_{j=1}^{n-1}\sum_{\substack{\sigma \in \mathcal{P}\\}}
  \text{Gr}_{2;n-2;j}(\sigma)A_{n;j}\left(\sigma(3),\dots,\sigma(j+1);1_q,\sigma(j+2),\dots,\sigma(n),2_{\bar
  q}\right),
\end{split}
\end{align}
with the following color-structure groupings
\begin{align}\label{eq:groupings}
   \text{Gr}_{2;n-2;j}(\sigma) = \Tr(T^{a_{\sigma(3)}} \dots
 T^{a_{\sigma(j+1)}})(T^{a_{\sigma(j+2)}} \dots
 T^{a_{\sigma(n)}})_{i_1}^{\bar{j_2}}.
\end{align}
The combined sum in Eq.~\eqref{eq:c2q} runs over all distinct color
structures: the outer sum over all different group
theoretical factors and the inner sum over independent gluon orderings
$\mathcal{P}=S_{n-2}/S_{n-2;j}$ that leave the trace in
Eq.~\eqref{eq:groupings} invariant. As an explicit example, the one-loop color
decomposition of $\Amp_4(1_q,2_{\bar q},3_g,4_g)$ in terms of partial amplitudes yields
\begin{align}\label{eq:partialdecomp}
\begin{split}
  \Amp_4(1_q,2_{\bar q},3_g,4_g) =&
  (T^3T^4)_{i_1}^{\bar{i}_2}\Am_{4;1}(1_q,3_g,4_g,2_{\bar q})+
  (T^4T^3)_{i_1}^{\bar{i}_2}\Am_{4;1}(1_q,4_g,3_g,2_{\bar
    q})\\
&+\Tr(T^3T^4)\delta_{i_1}^{\bar{i}_2}\Am_{4;3}(3_g,4_g;1_q,2_{\bar q}),
\end{split}
\end{align}
where we have used the conventions $(T^aT^b\dots T^n)_i^{\bar
  j} \rightarrow \delta_i^{\bar j}$ for $n=0$ and $\Tr(T^aT^b\dots
 T^n) \rightarrow 1$ for $n=0$ as well as the property $\Tr (T^a) =
 0$. The first list of particles in front of the semicolon is omitted if the color trace contribution is absent.


\section{Primitive Amplitudes}
\label{sec:primitives}
Partial amplitudes can be further decomposed into gauge invariant,
color-ordered basic building blocks, so called \textit{primitive amplitudes}
\cite{Bern1995b}, with a fixed cyclic ordering of external
legs. All partial amplitudes can be expressed by the set of primitive amplitudes
of a given process. Since primitive amplitudes can contribute to
several partial amplitudes, a caching mechanism for the primitives is
appropriate. Explicit examples of how to generate the set of primitives
and their relations to the partial amplitudes can be found in \cite{Bern1995b,Bern:1997sc}, with a general algorithm of how to
obtain them given in \cite{Ita:2011ar}. The Feynman rules used for the
computation of amplitudes have to be changed due to
our replacements (cf Eqs.~\eqref{eq:repcs} and \eqref{eq:fierz}). The
color-ordered Feynman rules can for example be found in
\cite{Mangano1991} and in Appendix \ref{sec:cofr}.

\begin{eqnarray*}
\begin{tikzpicture}[baseline=(m)]
\def\1up{1.0}
\def\up{2.0}
\def\opa{0.5}
\node at (0,-\up) {$L$};
  \begin{feynman}[inline=(m)]
    \vertex (m) at (0,0);
    \vertex (v1) at (-\1up,\1up);
    \vertex (v2) at (\1up,\1up);
    \vertex (v3) at (\1up,-\1up);
    \vertex (v4) at (-\1up,-\1up);
    \vertex (o1) at (-\up,\up){\(1\)};
    \vertex (o2) at (\up,\up){\(3\)};
    \vertex (o3) at (\up,-\up){\(4\)};
    \vertex (o4) at (-\up,-\up){\(2\)};
    \vertex (c1) at (\1up,0);
    \vertex (c2) at (\up,-\1up);
    \diagram* {
      (v1)  -- [gluon,thick,opacity=\opa] (v2) -- [gluon,thick,opacity=\opa] (v3)--  [gluon,thick,opacity=\opa] (v4) --  [fermion,thick] (v1), 
      (o1) -- [anti fermion,thick] (v1),
      (o2) -- [gluon,thick,opacity=\opa] (v2),
      (o3) -- [gluon,thick,opacity=\opa] (v3),
      (o4) -- [fermion,thick] (v4),
    };
  \end{feynman}
\end{tikzpicture}
\begin{tikzpicture}[baseline=(m)]
\def\1up{1.0}
\def\up{2.0}
\def\opa{0.5}
\node at (0,-\up) {$R$};
  \begin{feynman}[inline=(m)]
    \vertex (m) at (0,0);
    \vertex (v1) at (-\1up,\1up);
    \vertex (v2) at (\1up,\1up);
    \vertex (v3) at (\1up,-\1up);
    \vertex (v4) at (-\1up,-\1up);
    \vertex (o1) at (-\up,\up){\(1\)};
    \vertex (o2) at (\up,\up){\(3\)};
    \vertex (o3) at (\up,-\up){\(4\)};
    \vertex (o4) at (-\up,-\up){\(2\)};
    \vertex (c1) at (\1up,0);
    \vertex (c2) at (\up,-\1up);
    \diagram* {
      (v1)  -- [anti fermion,thick] (v2) -- [anti fermion,thick] (v3)--  [anti fermion,thick] (v4) --  [gluon,thick,opacity=\opa] (v1), 
      (o1) -- [anti fermion,thick] (v1),
      (o2) -- [gluon,thick,opacity=\opa] (v2),
      (o3) -- [gluon,thick,opacity=\opa] (v3),
      (o4) -- [fermion,thick] (v4),
    };
  \end{feynman}
\end{tikzpicture}
\begin{tikzpicture}[baseline=(m)]
\def\1up{1.0}
\def\up{2.0}
\def\opa{0.5}
\node at (0,-\up) {$n_fL$};
  \begin{feynman}[inline=(m)]
    \vertex (m) at (0,0);
    \vertex (v1) at (-\1up,\1up);
    %\vertex (v2) at (\1up,\1up);
    %\vertex (v3) at (\1up,-\1up);
    \vertex (v4) at (-\1up,-\1up);
    \vertex (o1) at (-\up,\up){\(1\)};
    \vertex (o4) at (-\up,-\up){\(2\)};
    \vertex (c1) at (\1up,0);
    \vertex (c2) at (\up,-\1up);
    \vertex (n1) at (\1up,0);
    \vertex (n0) at (-\1up0,0);
    \vertex (n2) at (0.5,0.5);
    \vertex[below=1cm of n2] (n3);
    \vertex[above right=1cm of n2] (v2){\(3\)};
    \vertex[below right=1cm of n3] (v3){\(4\)};
    \diagram* {
      (n0) --  [gluon,thick,opacity=\opa] (m), 
      (m) --  [fermion,thick,quarter left] (n2) --  [fermion,thick,half
      left] (n3) -- [fermion,thick,quarter left] (m),
      (v4) --  [fermion,thick] (v1), 
      (o1) -- [anti fermion,thick] (v1),
      (n2) -- [gluon,thick,opacity=\opa] (v2),
      (n3) -- [gluon,thick,opacity=\opa] (v3),
      (o4) -- [fermion,thick] (v4),
    };
  \end{feynman}
\end{tikzpicture}
\end{eqnarray*}
\vspace{-0.5cm}
\captionedequationset{Primitive amplitudes are represented by parent
  diagrams specified by a routing label for the fermion
  lines. Here we show the parent diagrams that contribute to the
  partial amplitude $\Am_{4;1}(1_q,3_g,4_g,2_{\bar q})$, that is left-turner ($L$), right
  turner ($R$) and closed fermion loop with the line passing to the left
  of the loop ($n_fL$).\label{fig:parentdiag}} 

We represent primitive amplitudes by \textit{parent diagrams},
that is color-ordered Feynman diagrams with the maximal number of loop
propagators. Additionally, the routing of
each fermion line which can either appear as a left- ($L$) or
right-turner ($R$) or as a closed
fermion loop ($n_f$), has to be specified. Amplitudes with
multiple fermion lines therefore have multiple routing labels. As an explicit example, we give the decomposition of
the partial amplitudes $\Am_{4;1}(1_q,3_g,4_g,2_{\bar q})$ and $\Am_{4;3}(3_g,4_g;1_q,2_{\bar q})$ which
appeared in the color decomposition in Eq.~\eqref{eq:partialdecomp}, in terms of primitive amplitudes
\begin{align}
\begin{split}
 \Am_{4;1}(1_q,3_g,4_g,2_{\bar{q}})&=\nc
 A^{[L]}(1_q,3_g,4_g,2_{\bar{q}})-\frac{1}{\nc}
 A^{[R]}(1_q,3_g,4_g,2_{\bar q})\\
&+ n_fA^{[n_fL]}(1_q,3_g,4_g,2_{\bar q}).
\end{split}
\end{align}
The right hand side is written in terms of primitive amplitudes only, where the superscripts $[L]$, $[R]$ and $[n_f]$ specify the quark routing in the primitive amplitude. The $[n_f]$ primitive contains contributions from both light flavors and heavy flavors turning
in the loop and the external quark line passes to the left ($n_fL$) of
the closed loop. In Fig.~\ref{fig:parentdiag}, we show the
graphical representation of the parent diagrams contributing to this partial
amplitude. The other partial amplitude in Eq.~\eqref{eq:partialdecomp} can be decomposed as
\begin{align}
\begin{split}
\Am_{4;3}(3_g,4_g;1_q,2_{\bar q})= \sum_{P(3,4)}& \left( A^{[L]}(1_q,3_g,4_g,2_{\bar{q}})+A^{[L]}(1_q,3_g,2_{\bar{q}},4_g) \right.\\
&\left.+ A^{[L]}(1_q,2_{\bar{q}},3_g,4_g)\right),
\end{split}
\end{align}
where we sum over the two permutations of gluon labels.

For more general topologies, we apply the decomposition algorithm
presented in \cite{Ita:2011ar}, which is applicable to massive quarks. By
using symmetry properties amongst the primitive amplitudes, one can
further reduce the number of fundamental building blocks that have to
be computed. In computations with massive fermions, care has to be
taken because not all symmetry properties of massless quarks hold for
massive ones as well.\footnote{For example, the reversal of a fermion's arrow (cf.~Eq.~(2.23) in
\cite{Ita:2011ar}) does not lead to a flip of the routing label for massive
quark lines.}



\section{Assembly of Virtual Matrix Elements}
\label{sec:assembly-virt-matr}
In order to obtain the full virtual cross-section $\dd \sigma_{\text{V}}$, we need to
interfere tree- and loop-level amplitudes as noted in
 Eq.~\eqref{eq:interference}. Therefore, open fundamental indices of
 tree and loop color factors are contracted and the resulting color factors are made up of products of
 traces of generators with repeated adjoint indices stemming from both tree and loop contributions. We then use the Fierz identity, Eq.~\eqref{eq:fierz}, on the
 repeated adjoint indices
 \begin{align}
\begin{split}
\Tr(T^aT^a) &= (T^a)_j^{\bar
     i}(T^a)_i^{\bar
     j}=\delta_{i}^{\bar{i}}\delta_{j}^{\bar{j}}-\frac{1}{N_C}\delta_{j}^{\bar{i}}\delta_{i}^{\bar{j}}=N_C^2-1\\
   \Tr(T^aAT^aB)&=\Tr(A)\Tr(B)-\frac{1}{N_C}\Tr(AB),
\end{split}
 \end{align}
where $A$ and $B$ are products of generators. Therefore, the
color factors reduce to positive and negative powers of $N_C$ that
multiply the products of tree and one-loop
 partial amplitudes.

 For the \textit{assembly} of virtual matrix
 elements, that is multiplying tree and one-loop partial amplitudes
 with the corresponding color factors, we use the approach of previous
 work \cite{Ita:2011ar,BH:W4j,BH:W5j,BH:Z4j}. The color-decomposition therefore has to
 be done only once. For the remainder of this thesis, we work with color-ordered primitive amplitudes with a fixed cyclic ordering of
 external legs.



