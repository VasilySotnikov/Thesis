Understanding the fundamental principles of nature has been one of the most longstanding and remarkable adventures of the human civilization.
%One of the most important concepts which makes it possible is cumulative knowledge.

Can be defined as continuously asking "why?" 

So far, this lead us to the picture of particles and their interactions.
More precisely, particles are excitations of the underlying fields.

QFT is the most successful formalism for understanding  phenomena
And the SM is the most successful QFT to date.
It is not, however complete.

QFT provides a more quantitative explanation to the notion that something needs to be smashed harder to probe it's structure.
One of the  energy allows to probe shorter distances, hence the need for colliders.
Today the most powerful machine is the \emph{Large Hadron Collider} (LHC).
One of the highlights is the discovery of the Higgs, the last missing piece for self-consistency of SM,
which, in principle, extends it's validity all the way to the Planck scale.

%The foremost objective of this thesis is to contribute to advance the knowledge further, however minuscule the contribution is.


Quantitative predictions match to experiment with incredible precision.



Interplay between theory and expertiment
To uncover the full potential of the LHC

Precision measurements, discovery via precision,
stringent constraints on physica beyond the SM

Sources of uncertainties: statistical decreases with data.

Theoretical predictions 

Modeling of hadron collisions: effort of many people.

Improvement of predictions from higher-order corrections

%Scattering amplitudes in QCD provide the basic building blocks for hadron
%collider phenomenology. Integrated over phase space, they yield theoretical
%predictions that can be compared to experimental measurements like those
%performed at the CERN Large Hadron Collider (LHC). These theoretical
%predictions can be systematically improved through inclusion of higher-order
%QCD corrections, which require higher-loop scattering amplitudes.  The analytic
%computation of these multi-loop amplitudes is a non-trivial task.

Complexity of multi-loop computations

Modern computational algorithms.

Analytical vs numerical algorithms: scaling of complexity

Role of understanding of structure of the theory to the copmutations/






1) Content of thesis, extend of thesis
2) scientific relevance of results
3) evaluation of independent scientific contribution of candidate


