First and foremost I would like to thank my supervisors Harald Ita and Fernando Febres Cordero.

Ben Page, Samuel Abreu for mentoring and showing me the ways


Reading chapters:

Ekta Chaubey


Anna Gantimurova first years

GRK for framework, seminars and events


%First and foremost I would like to thank my supervisor Harald Ita, for his scientific guidance and his supportive attitude. Your advice and help has been invaluable! I would also like to express my gratitude to my second supervisor Fernando Febres Cordero, for the good collaboration and his outstanding commitment. I really enjoyed working with you! 

%I would like to thank my collaborators Stefan H{\"o}che and Daniel Ma{\^i}tre, for the things I have learnt from them.

%I would like to thank my colleague and office mate Vasily Sotnikov, for all the discussions and good collaboration on many aspects of this work. We have been a good team! To all the current and former members of the 8\textsuperscript{th} (and 7\textsuperscript{th}) floor for supporting me in one or another way but also for the good times shared: Kicker tournaments, Christmas parties and the ``Betriebsausflug'' come to my mind. A special thanks goes to Samuel Abreu for reading the manuscript of this thesis, to Matthijs van der Wild for forming the winning team of the kicker tournament '17 and to Jerry Dormans for tolerating our long discussions in the office but also for the fun in ``office 809''. To the members of the $\hbar$-racing team for the shared memories on snow and beyond: Lukas Altenkamp, Michele Boggia, Michael Kordovan.

%I would like to thank Hannah Arnold, Giulia Gonella and Gernot Knippen for the shared time as student representatives of the ``Graduiertenkolleg 2044''. From organizing BBQs to increasing the ``educational value'' of the seminar series: I think we provided some innovative impulses!

%Finally, I would like to thank Stefanie for all the things one cannot express in words!

%%\textit{blablabla}
%%\textit{Wissenschaft scheint die geistreiche Verkn\"upfung von Altbekanntem auf neue Weise.} (Anonym, 8.Stock)
