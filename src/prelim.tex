\section{Colour decomposition}
We decompose \ola s into purely kinematic functions, the so called partial amplitudes, that multiply group-theoretic factors, so called \cf s. We work in \SNc, where \Nc$=3$ is set for QCD. The group generators are normalized according in the fundamental representation, such that $\text{Tr}(T^aT^b) = \delta^{ab}$. We then rewrite the appearing structure constants $f^{abc}$ by 
\begin{align}
f^{abc} = - \frac{i}{\sqrt{2}}\text{Tr}\left([T^a,T^b]T^c\right)
\end{align}
and apply the Fierz identity for the \SNc~group
\begin{align}
\sum_a (T^a)_k^{\bar i} (T^a)_l^{\bar j} = \delta_k^{\bar j} \delta_l^{\bar i} - \frac{1}{N_c}  \delta_k^{\bar i} \delta_l^{\bar j} 
\end{align}
which leads to a \cd. We then collect terms according to their \cf~and obtain the following expressions for \ola s.

\begin{align}
\Amp_n (\{k_i,h_i,a_i\}) = \sum_{c=1}^{[n/2]+1}\sum_{\sigma \in S_n/S_{n;c}} \text{Gr}_{n;c}(\sigma)A_{n;c}(\sigma)
\end{align}
The leading \cf~GR$_{n;1}(1)=N_c \text{Tr}(T^{a_1}\dots T^{a_2})$ is given by the product of \Nc~and the tree \cf. Sub-leading \cs s are given by products of trace expressions.

Partial amplitudes can then be further decomposed into gauge invariant
building blocks, the primitive amplitudes. They do have a more
transparent anlytic structure due to the fixed order of the appearing
external legs, which results in a reduction of both factorization
channels as well as invariants that can appear. In particular, this
makes their usage amenable to methods based on factorization and
branch cut singularity properties, such as numerical unitarity. We do
use a \cd~based on the algorithm proposed in \cite{Ita:2011arx}


\section{The Loop Integrand}

Decomposition in terms of master integrals (massive basis), The generic form of a loop integrand at 1-loop, loop integrations,
transverse / physical space, tensor basis
A generic N-particle one-loop amplitude is given by
\begin{align}
 A_N^{\text{1-loop}}= \int \text{d}^4 l
 \frac{N(p_1,p_2,\dots,p_N;l)}{D_1D_2\dots D_N}
\end{align}

\section{Tree amplitudes}
We generate the required \co~tree amplitudes using an off-shell recursion due to Berends-Giele. This represents an efficient and flexible ways to obtain tree amplitudes. The recursion is based the idea of summing over all possible ways to connect the appearing building blocks, off-shell currents, to each other. The interactions of the theory as well as all possible way to distribute the \co~particles determine one recursive step. 








